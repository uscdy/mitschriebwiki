\documentclass[a4paper,11pt]{book}

\usepackage{amssymb}
\usepackage{amsmath}
\usepackage{amsfonts}
\usepackage{ngerman}
%\usepackage{graphicx}
\usepackage{fancyhdr}
\usepackage{euscript}
\usepackage{makeidx}
\usepackage{hyperref}
\usepackage[amsmath,thmmarks,hyperref]{ntheorem}
\usepackage{enumerate}
\usepackage{url}
\usepackage{mathtools}
\usepackage[arrow, matrix, curve]{xy}
%\usepackage{pst-all}
%\usepackage{pst-add}
%\usepackage{multicol}

\usepackage[latin1]{inputenc}

%%Zahlenmengen
%Neue Kommando-Makros
\newcommand{\R}{{\mathbb R}}
\newcommand{\C}{{\mathbb C}}
\newcommand{\N}{{\mathbb N}}
\newcommand{\Q}{{\mathbb Q}}
\newcommand{\Z}{{\mathbb Z}}
\newcommand{\K}{{\mathbb K}}
\newcommand{\ssL}{{\mathcal L}}
\newcommand{\sn}[1]{||#1||_{\infty}}
\newcommand{\eps}{{\varepsilon}}
\newcommand{\begriff}[1]{\textbf{#1}} %das sollte man noch ändern!
\newcommand{\eb}{\begin{flushright} \rule{1ex}{1ex} \end{flushright}}
\newcommand{\ind}{1\hspace{-0,9ex}\raisebox{-0,2ex}{1}}

% Seitenraender
\textheight22cm
\textwidth14cm
\topmargin-0.5cm
\evensidemargin0,5cm
\oddsidemargin0,5cm
\headheight14pt

%%Seitenformat
% Keine Einrückung am Absatzbeginn
\parindent0pt

\DeclareMathOperator{\unif}{Unif}
\DeclareMathOperator{\var}{Var}
\DeclareMathOperator{\cov}{Cov}


\def\AA{ \mathcal{A} }
\def\PM{ \EuScript{P} } 
\def\EE{ \mathcal{E} }
\def\BB{ \mathfrak{B} } 
\def\DD{ \mathcal{D} } 
\def\NN{ \mathcal{N} } 

% Komische Symbole
\def\folgt{\ensuremath{\implies}}
\newcommand{\folgtnach}[1]{\ensuremath{\DOTSB\;\xRightarrow{\text{#1}}\;}}
\def\equizu{\ensuremath{\iff}}
\def\d{\mbox{d}}
\def\fs{\stackrel{f.s.}{\rightarrow }}

%Nummerierungen
\newtheorem{Def}{Definition}[chapter]
\newtheorem{Sa}[Def]{Satz}
\newtheorem{Theo}[Def]{Theorem}
\newtheorem{Lem}[Def]{Lemma}
\newtheorem{Kor}[Def]{Korollar}
\theorembodyfont{\normalfont}
\newtheorem{Bsp}[Def]{Beispiel}
\newtheorem{Bem}[Def]{Bemerkung}
\newtheorem*{BemNO}{Bemerkung}
\theoremstyle{nonumberplain}
\theoremsymbol{\ensuremath{_\blacksquare}}
\newtheorem{Bew}{Beweis}
\setcounter{chapter}{1}
\setcounter{Def}{33}

% Kopf- und Fusszeilen
\pagestyle{fancy}
\fancyhead[LE,RO]{\thepage}
\fancyfoot[C]{}
\fancyhead[LO]{\rightmark}

\title{15.11.06}
\author{Das \texttt{latexki}-Team\\[8 cm]}

\date{Stand: \today}
\begin{document}

\maketitle

F"ur $A\in \ssL_{d}$ und $f:A^{d}\rightarrow \C \qquad f:A\rightarrow[-\infty,\infty]$ beschr"ankt betrachtet man\\
$\tilde{f}(x)= \begin{cases}
f(x)&,x\in A\\ 
0&, x\in \R^{d}\backslash A \end{cases}$\\
$f$ hei\ss t \underline{integrierbar "uber A} wenn $\tilde{f}$ integrierbar ist ("uber $\R^{d})$ und wir setzen 
\begin{displaymath}
\int\limits_{A}{f}(x)dx = \int\limits_{\R^{d}}\tilde{f}(x)dx.
\end{displaymath}
Beachte: F"ur integrierbare $f$ auf $\R{d}$ gilt
\begin{displaymath}
\int\limits_{\R^{d}}\mathbb{1}_{A}(x){f}(x)dx = \int\limits_{A}{f}(x)dx.
\end{displaymath}
%Bemerkung 1.34
\begin{Bem}$\\$
\begin{enumerate}
\item [a)] Die obigen Def. sind unabh"angig von der Darstellung der einfachen Funktionen und der Wahl der aprox. Fkt. $\varphi_{k}$ (siehe A§,X,2.1,2.7,3.2).
Obiges Integral und das in K"onigsberger II stimmen "uberein.
\item [b)] Sei $f:\R^{d} \rightarrow \C$ messbar. Dann ist $f$ int'bar $\equizu|f|$ int'bar. 
\begin{Bew}:
\glqq$\Rightarrow$\grqq verwende zu $\varphi_{n}$ aus Def. 1.33 die Folge $|\varphi_{n}|$ (AE,X,2.8).\\
\glqq$\Leftarrow$\grqq Es gilt $f=f_{1}-f_{2}+if_{3}-if_{4}$ mit $0\leq f_{k}\leq |f|$. Verwende c).
\end{Bew}
\item [c)] F"ur $A\in\ssL_{d}$ setzt man $\ssL^{1}(A):\{f:A\rightarrow\C : f \textrm{integrierbar}\}$. \\
Dann gilt: $\ssL^{1}(A)$ ist VR und $||f||_{1}=\int\limits_{A}|f|dx$ ist Halbnorm auf $\ssL^{1}(A)$.\\
(Beachte $||\mathbb{1}_{N}||_{1} = \lambda (N) = 0$ f"ur jede NM $N$.)\\
Seien $f,g\in\ssL^{1}(A)$, $ \alpha $,$\beta \in \K$. Dann:
\begin{enumerate}
\item [i)] 
\begin{displaymath}
\int\limits_{A}(\alpha f+\beta g)dx=\alpha \int\limits_{A}f dx+\beta \int\limits_{A}gdx
\end{displaymath}
\item [ii)] 
\begin{displaymath}
\K = \R , f\leq g \folgt \int\limits_{A}fdx\leq \int\limits_{A}gdx.
\end{displaymath}
\item [iii)]
\begin{displaymath}
|\int\limits_{A}fdx|\leq\int\limits_{A}|f|dx\qquad \textrm{(nach b))}
\end{displaymath}
\item [iv)]
$h$ messbar, $|h|\leq |f|$ dann $h\in \ssL^{1}(A)$. Analoge Eigenschaften f"ur:\\ $f:A\rightarrow[0,\infty]$.
(Folgt leicht per Approximation. Siehe AE 2.4,2.9,2.11,3.3)
\end{enumerate}
\item [d)]
Sei $\varphi :A\rightarrow [0,\infty]$ messbar mit $\int\limits_{A}\varphi dx <\infty$. Dann ist $\varphi(x)<\infty$ ffa. $x\in A$.
\end{enumerate}
\begin{Bew}
Anm.: $\varphi(x)=\infty$ f"ur $x\in B$ mit $\lambda(B)>0$. $B\subseteq A$ messbar. Dann folgt f"ur alle $n\in N$: $\int\limits_{A}\varphi dx\geq \int\limits_{A}n\cdot \mathbb{1}_{B}dx = n\cdot \lambda(B)\rightarrow_{n\rightarrow\infty} \infty$ Wid.!
\end{Bew}
\end{Bem}
%Theorem 1.35
\begin{Theo}(Lemma von Fatou, mojorisierte konvergenz)(Beppo Levi)\\
Seien $f_{n}:A\rightarrow [0.\infty]$ messbar $(n\in \N)$, $A\in\ssL_{d}$.\\
Dann gelten:
\begin{enumerate}
\item [a)]
\begin{displaymath}
\int\limits_{A} \underline{\lim_{n\rightarrow \infty}} f_{n}dx \leq \underline{\lim_{n\rightarrow \infty}} \int\limits_{A} f_{n} dx. \textrm{(AE,X,3.7)}
\end{displaymath}
\item [b)]
\begin{displaymath}
f_{n} \leq f_{n+1} \folgt \int\limits_{A} \lim_{n\rightarrow \infty}f_{n}dx=\lim_{n\rightarrow \infty} \int\limits_{A} f_{n}dx
\end{displaymath}
\end{enumerate}
\end{Theo}
%Theorem 1.36
\begin{Theo}(majorisierte Konvergenz)\\
Seien $f_{n},g \in \ssL^{1}(A), A\in \ssL_{d}$ $f_{n}(x)\rightarrow f(x) (n\rightarrow \infty)$ ffa. $x\in A$\\
 \underline{und} $ |f_{n}(x)|\leq g(x)$ ffa.
$x\in A$ und alle $n\in \N$.Dann gilt:
\begin{displaymath}
f\in \ssL^{1}(A) \textrm{ und } ||f_{n}-f||_{1} \rightarrow 0 (n\rightarrow \infty)\\
\textrm{ somit } \int\limits_{A} f_{n}dx \rightarrow \int\limits_{A} fdx.
\end{displaymath}
\end{Theo}
\begin{BemNO}Majorante bzw. Monotonie sind oben \underline{wesentlich}.\\
\underline{Bsp:} hier steht eine Zeichnung!\\
$\folgt f_{n}(x) \rightarrow 0 \quad\forall x\geq 0, n\rightarrow \infty$ aber 
$||f_{n}||_{1} = n \quad\rightarrow\infty$\\
F"ur $A\in\ssL_{d}$, $p\in[1,\infty]$ messbare $f:A\rightarrow \C$ definiert man
\begin{displaymath}
||f||_{p} = (\int\limits_{A}|f(x)|^{p}dx)^{\frac{1}{p}} \quad p<\infty \textrm{(betrachte $|f|^{p}$ ist messbar.)}
\end{displaymath}
\begin{displaymath}
\sn{f}=\inf\{c\geq 0: |f(x)| \leq c,\textrm{ f"uer }x\notin N=N(c)\}= ess \sup_{x\in A} |f(x)| \quad (\textrm{wobei } \inf \varnothing = \infty )
\end{displaymath}
$\ssL^{p}(A) = \{f:A\rightarrow \C : \textrm{messbar und }  ||f||_{p} <\infty \}$\\
Schon gesehen $\ssL^{1}(A)$ ist VR, $||\cdot||_{1}$ ist Halbnorm.
\end{BemNO}
%Satz 1.37
\begin{Sa}
$\mathcal{L}^{\infty}(A) $ ist VR mit Halbnorm $ ||\cdot||_{\infty}$
\end{Sa}
\begin{Bew}$\\$
Seien $f_{j} \in \mathcal{L}^{\infty}(A), |f_{j}(x)| \leq c_{j} \qquad x \notin N_{j} j=1,2,... N_{j}=NM$
\\Dann: $|f_{1}(x)+f_{2}(x)|\leq c_{1} + c_{2}\qquad \forall x\notin N_{1}\bigcup N_{2}$, wobei $N=N_{1}\bigcup N_{2}=NM.$
\\$\folgt f_{1}+f_{2} \in \mathcal{L}^{\infty}(A), ||f_{1}+f_{2}||_{\infty} \leq ||f_{1}||_{\infty} + ||f_{2}||_{\infty}$
''$||\alpha f_{1}|| = |\alpha| ||f_{1}||_{\infty}$'' zeigt man genau so.
\end{Bew}
%Satz 1.38
\begin{Sa}
Seien $p\in [1,\infty], A \in \mathcal{L}_{d}, f,g \in \mathcal{L}^{p}(A), h\in \mathcal{L}^{p}(A).$ Dann: $\\$
\begin{enumerate}
\item [a)] $fh \in \mathcal{L}^{1}(A).\qquad \int\limits_{A}|fh|dx = ||fh||_{1} \leq ||f||_{p} \cdot ||h||_{p}$
\item [b)] $||f+g||_{p} \leq ||f||_{p} + ||g||_{p}\\$
und $\mathcal{L}^{p}(A)$ ist $VR$ mit Halbnorm $||\cdot||_{p}$ (Minkowski).
\item [c)] $f_{n} \in \mathcal{L}^{p}(A), f_{n}\rightarrow\varphi$ für $(n\rightarrow\infty)$. $|f_{n}| \leq g$ fast "uberall, $g\in\mathcal{L}^{p}(A).$\\
Dann $\varphi\in\mathcal{L}^{p}(A), f_{n}\rightarrow\varphi$ in $\mathcal{L}^{p}(A) \quad (n\rightarrow\infty).$
\end{enumerate}
\end{Sa}
\begin{Bew}
F"ur Fall $p=1$ ist klar, verwende f"ur H"older (Bem 1.40)
$\\$ Sei also $p\in(1,\infty)$ und $p' = \frac{p}{(p-1)}$.
\begin{enumerate}
\item [a)] Da $f,g$ messbar sind, nur Absch"atzung zu zeigen. Wie in (1.26) liefert(1.4)\\
$\int\limits_{A}|f(x)| |h(x)|dx \leq \frac{t^{p}}{p} \underbrace{\int\limits_{A}|f(x)|^{p}dx}_{=||f||_{p}^{p}} + \frac{t^{-p'}}{p'}\int\limits_{A}|h(x)|^{p'}dx \qquad \forall t>0$\\
$\underbrace{inf}_{t>0}$ liefert Behauptung mit (1.4)
\item [b)] $|f(x)+g(x)|^{p}\leq 2^{p} (|f(x)|^{p}+|g(x)|^{p})\folgt f+g\in \mathcal{L}^{p}(A) \\
\underbrace{||f+g||_{p}^{p}}_{endlich}=\int\limits_{A}|f+g||f+g|^{p-1}dx \leq \int\limits_{A} |f|+|f+g|^{p-1}dx + \int\limits_{A}|g||f+g|^{p-1}dx \leq ||f||_{p} (\int\limits_{A}|f+g|^{(p-1)\frac{p}{p-1}} dx)^{\frac{p-1}{p}}
\leq ||f||_{p}(\int\limits_{A}|f+g|^{p-1}\frac{p}{p-1} dx)^{\frac{p-1}{p}} + ||g||_{p} (\int\limits_{A}|f+g|^{p} dx)^{\frac{p-1}{p}}= (||f||_{p}+||g||_{p}) ||f+g||_{p}^{p-1} \folgt$ Behauptung (b).
\item [c)] Haben $|f_{n}|^{p} \leq g^{p}$ fast "uberall $(n\in \mathbb{N})$\\ und $|f_{n}|^{p} \rightarrow |f|^{p}$ fast "uberall $(n \rightarrow \infty)$ \\
major.konv. $\folgt |f|^{p} \in \ssL^{1}(A) \folgt f \in \ssL^{p}(A) \\$
Setze $h_{n}=|f_{n}-f|^{p} \in\ssL^{1}(A)\folgt h_{n}\rightarrow 0 \quad$ f."u. $(n\rightarrow\infty)$.\\
$0\leq h_{n} \leq 2^{p} (|f_{n}|^{p}+|f|^{p})\leq 2^{p}(g^{p}+|f|^{p})\in\ssL^{1}$. majorisierte Konvergenz: $||f-f_{n}||_{p}=||h_{n}||_{1}\rightarrow 0 (n\rightarrow\infty)$
\end{enumerate} 
\end{Bew}
\end{document}