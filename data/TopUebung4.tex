\documentclass{article}
\usepackage[utf8]{inputenc}
\usepackage{mathrsfs}
\usepackage{stmaryrd}

\usepackage{mathe}
\usepackage{enumerate}

\title{4. Topologie-Übung}
\author{Joachim Breitner}
\date{14. November 2007}

\begin{document}
\maketitle
\section*{Aufgabe 1}

Es gibt auf der Menge $X \da \{1,2,3\}$ folgende Topologien, geordnet nach Zahl der Elemente:
\begin{itemize}
\item $\{\emptyset, X\}$
\item $\{\emptyset, X, \{a\}\}$, für $a\in X$ (3 Möglichkeiten)
\item $\{\emptyset, X, \{a,b\}\}$, für $a\ne b\in X$ (3 Möglichkeiten)
\item $\{\emptyset, X, \{a\}, \{a,b\}\}$, für $a\ne b\in X$ (6 Möglichkeiten)
\item $\{\emptyset, X, \{a\}, \{b,c\}\}$, für $a,b,c\in X$ paarweise verschieden (3 Möglichkeiten)
\item $\{\emptyset, X, \{a\}, \{b\}, \{a,b\}\}$, für $a\ne b\in X$ (3 Möglichkeiten)
\item $\{\emptyset, X, \{a\}, \{a,b\}, \{a,c\}\}$, für $a,b,c\in X$ paarweise verschieden (3 Möglichkeiten)
\item $\{\emptyset, X, \{a\}, \{b\}, \{a,b\}, \{a,c\}\}$, für $a,b,c\in X$ paarweise verschieden (6 Möglichkeiten)
\item $\mathcal P(X)$
\end{itemize}
Insgesamt gibt es also 29 verschiedene Topologien auf $X$.

\section*{Aufgabe 2}

\paragraph{Behauptung:} Sei $X$ ein topologischer Raum, $A\subseteq X$. Dann gilt: $A$ ist offen und abgeschlossen genau dann, wenn $\partial A=\emptyset$.

$\partial A = \bar A \setminus \mathring A$, $\mathring A = \bigcup_{U\subset A,\text{ $U$ offen}} U$, $\bar A = \bigcap_{A\subset U,\text{ $U$ abg.}} U$, 

„$\Longrightarrow$“: $A$ offen, also $A = \mathring A$, $A$ abgeschlossen, also $A= \bar A$, also gilt $\partial A = \bar A \setminus \mathring A = A \setminus A = \emptyset$.

„$\Longleftarrow$“: $\bar A \setminus \mathring A = \emptyset \implies \bar A = \mathring A \implies A\subseteq \bar A = \mathring A \subseteq A \implies A$ ist offen und abgeschlossen.

\paragraph{Behauptung:} $x\in \partial A$ genau dann, wenn für jede Umgebung $U$ von $X$ gilt: $U\cap A \ne \emptyset$ und $U \cap (X\setminus A) \ne \emptyset$.

„$\Longrightarrow$“: $x\in \partial A = \bar A \setminus \mathring A$. Sei $U$ eine Umgebung von $x$, die o.B.d.A offen ist.

\begin{enumerate}[1. {Fall}:]
\item $x\in A$, also $U\cap A \ne \emptyset$.

Annahme: $U\cap (X\emptyset A) \ne \emptyset \implies U \subseteq A \implies x\in \mathring A \implies x\in \bar A \setminus \mathring A \wedge x \in \mathring A$ $\lightning$

\item $x\notin A$, also $U \cap (X \setminus A)\ne \emptyset$

Annahme: $U \cap A = \emptyset \implies A \subseteq X\setminus U$, also $X\setminus U$ ist abgeschlossene Teilmenge von $X$, die $A$ enthält, also $x\in X\setminus U$, im Widerspruch zu $x\in U$.

\end{enumerate}

„$\Longleftarrow$“: $x\notin \mathring A$, denn wäre $x\in \mathring A$, so wäre $\mathring A$ eine Umgebung von $x$, also nach Vorraussetzung $\mathring A\cap (X\setminus A) \ne 0$, im Widerspruch zu $\mathring A \subseteq A$.

$x\in \bar A$, denn wäre $x\notin \bar A$, so wäre $X\setminus \bar A$ offen und eine Umgebung von $x$, also gälte $(X\setminus \bar A) \cap A \ne \emptyset$, im Widerspruch zu $\bar A \supseteq A$.

Also gilt: $x\in \bar A \setminus \mathring A = \partial A$.


\section*{Aufgabe 3}

$A \subseteq \MdC^n$ heißt Zariski-abgeschlossen, wenn es $P_i\in \MdC^n[X_1,\ldots,X_n]$, $i\in I$ gibt mit $A = \{z\in \MdC^n \mid \forall i\in I: P_i(z)=0\}$.

$A \subseteq \MdC^n$ heißt Zariski-offen, genau dann, wenn $\MdC^n\setminus A$ Zariski-abgeschlossen ist.

\paragraph{Behauptung:} Das ist eine Topologie auf $\MdC^n$.
\begin{itemize}
\item $\MdC^n$ und $\emptyset$ sind Zariski-offen, da $\emptyset$ Nullstellenmenge von $P(z)\da1$ und $\MdC^n$ Nullstellenmenge von $P(z)\da0$ ist.

\item Sei $(U_i)_{i\in I}$ eine Familie Zariski-offener Mengen. dann ist $\bigcup_{i\in I} U_i$ auch Zariski-offen:

Für jedes $i\in I$ gilt: $U_i$ ist Zariski-offen, also gibt es Polynome $P_{ij}\in \MdC^n[X_1,\ldots,X_n]$, $i\in I$, $j\in J_i$, mit
\[\MdC^n\setminus U_i = \{ z\in \MdC^n \mid \forall j\in J_i: P_{ij}(z) = 0\}\,.\]
Also ist 
\[\MdC^n\setminus \bigcup_{i\in I} U_i = \bigcap_{i\in I}(X\setminus U_i) = \{x\in \MdC^n \mid \forall i\in I \ \forall j\in J_i : P_{ij}(z) = 0 \}\]
Zariski-abgeschlossen, und damit $\bigcup_{i\in I} U_i$ Zariski-offen.
\item Seien $U,V$ Zariski-offene Teilmengen. Dann ist $U\cap V$ auch Zariski-offen:

$U$ ist Zariski-offen, also ist $\MdC^n\setminus U$ ist Nullstellenmenge einer Familie von Polynomen $P_i$, $i\in I$: $U = \MdC^n \setminus \{ z\in \MdC^n \mid \forall i \in I: P_i(z)=0\} = \MdC^n\setminus \bigcap_{i\in I}U_i = \bigcup_{i\in I}(\MdC^n \setminus U_i)$, wobei $U_i = \{z \in \MdC^n \mid P_i = 0\}$.

Analog ist $V = \bigcup_{i\in J}(\MdC^n \setminus V_j)$, wobei $V_j = \{z \in \MdC^n \mid Q_j(z) =0\}$.

Damit ist $\MdC^n \setminus (U\cap V) = \bigcap_{i\in I, j\in J}(U_i \cup V_j) = \{z\in \MdC^n \mid \forall (i,j)\in I\times J: P_{ij}(z)=0 \}$, wobei $P_{ij} = P_i\cdot Q_j$. Also ist $\MdC^n\setminus(U\cap V)$ abgeschlossen und $U\cap V$ offen.
\hfill$\blacksquare$
\end{itemize}

Auf $\MdC$ sind Zariski-offene Mengen sind dann gerade die Komplemente endlicher Mengen, das heißt: $\MdC$ ist nicht hausdorff’sch bezüglich dieser Topologie.

\paragraph{Behauptung:} $\mathcal B \da \{U \subset \MdC^n \mid U$ ist Komplement einer Nullstellenmenge eines einzelnen Polynoms$\}$

Sei $U$ offen, dann ist $\MdC^n \setminus U = \{ z\in \MdC^n \mid \forall i\in I: P_i(z) = 0 \}$ mit $P_i\in \MdC[X_1,\ldots,X_n]$, $i\in I$. Dann ist
\[
\MdC^n\setminus U = \bigcap_{i\in I} \underbrace{\{z\in \MdC \mid P_i(z)=0\}}_{B_i\da} = \bigcup_{i\in I}(\MdC^n\setminus B_i)
\]
mit $(\MdC^n\setminus B_i)\in \mathcal B$, also ist $U$ Vereinigung von Mengen aus $\mathcal B$.


\section*{Aufgabe 4}

Betrachte die Topologie auf $\mathbb Z$, die $\{a + b\mathbb Z\mid a, b\in \MdZ, b\ne 0\}$ als Subbasis besitzt.



\paragraph{Behauptung:} Jede Menge der Form $a+b\MdZ$, $b\ne 0$ ist abgeschlossen bezüglich dieser Topologie.

Es gilt o.B.d.A: $a + b \MdZ$ = $\MdZ \setminus \bigcup_{i=1}^{b-1}((a+i) + b \MdZ)$, also ist $a + b\MdZ$ komplement einer offenen Menge, also abgeschlossen.

\paragraph{Behauptung:} $\{-1,1\}$ ist abgeschlossen.

Es gilt: $\MdZ \setminus \{-1,1\} = \bigcup_{p\in \mathbb P} (0 + p\MdZ)$, denn jedes $n\in \MdZ$ hat eine Primzahl $p$ als Teiler, wenn $n \notin \{-1,1\}$, also $n\in p\MdZ$. Daher ist $\MdZ\setminus \{-1,1\}$ offen und $\{-1,1\}$ abgeschlossen.

\paragraph{Behauptung:} Es gibt unendlich viele Primzahlen $\mathbb P$.

Annahme: $\mathbb P$ ist endlich. Dann wäre $\mathbb Z\setminus\{-1,1\}$ als endliche Vereinigung abgeschlossener Mengen abgeschlossen, also wäre $\{-1,1\}$ offen. Das ist ein Widerspruch, denn alle offenen Mengen $\ne \emptyset$ sind in dieser Topologie unendlich.
\phantom{Anfang}\hfill{$\blacksquare$}
\end{document}
