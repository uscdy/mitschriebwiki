\documentclass[a4paper,twoside,DIV15,BCOR12mm]{scrbook}
\usepackage{ztheo}
\usepackage{enumerate} 
%Eingefügt von Stephan
%Liefert Optionen für enumerate-Umgebung, wie z.b. für eine "römische
%Liste": \begin{enumerate}[(i)] \item ...

\lecturer{Prof. Rehm}
\semester{Sommersemester 2006}
\scriptstate{unknown}


\pdfinfo{
        /Author (Die Mitarbeiter von http://mitschriebwiki.nomeata.de/)
        /Title   (Elementare Zahlentheorie)
        /Subject (Elementare Zahlentheorie)
        /Keywords (Zahlentheorie)
}

\author{Die Mitarbeiter von \url{http://mitschriebwiki.nomeata.de/}}
\title{Elementare Zahlentheorie}
\makeindex

\begin{document}
\maketitle

\renewcommand{\thechapter}{\roman{chapter}}
%\chapter{Inhaltsverzeichnis}
\addcontentsline{toc}{chapter}{Inhaltsverzeichnis} \tableofcontents


\chapter*{Bezeichnungen und Vorraussetzungen}
\begin{itemize}
\item Logische Zeichen: $\implies$, $\equizu$, $\forall$, $\exists$, $\exists^1$ (es gibt genau ein), $\wedge$ (und), $\vee$ (oder)
\item Zeichen der Mengenlehre: z.B. $\cup$, $\cap$, $\MdN := \{x\in \MdZ| x \ge 0 \}$
\item Induktion als Beweistechnik
\item $\#M$ Kardinalität der Menge $M$, z.B. $\#\MdN = \infty$
\item $\MdN = \{0,1,2,3,\ldots\}$, $\MdNp = \{1,2,3,4,\ldots\}$ (natürliche Zahlen)
\item $\MdZ = \{0, \pm1, \pm2, \pm3, \ldots \}$ (Ring der ganzen Zahlen)
\item $\MdQ = \{\frac z n| z\in \MdZ, n\in \MdN_+\}$ (Körper der rationalen Zahlen)
\item $\MdR$ Körper der reelen Zahlen
\item $\mathbb{F}_q$ Körper mit $q<\infty$ Elementen (= $GF(q)$ in der Informatik)
\item $\MdP = \{ 2,3,5,7,11,13,17,19,23,\ldots\}$ Menge aller Primzahlen
\end{itemize}

\setcounter{chapter}{0}
\renewcommand{\thechapter}{\arabic{chapter}}

\chapter{Primzerlegung}

\section[Einführung und Motivation]{Faszination Primzahlen: Primzahlsatz (o.Bew.), gelöste
und ungelöste Probleme über Primzahlen}

\begin{satz}[Euklid, ca. 300 v. Chr.]
\[ \#\MdP = \infty \]
\end{satz}

\begin{bemerkung}
Analysis:
\begin{align*}
\sum_{n\in\MdN} \frac{1}{n} &= \infty \\
\sum_{n\in\MdN} \frac{1}{n^2} &< \infty \\
\intertext{Euler:} \sum_{p\in\MdP} \frac{1}{p}  &= \infty
\end{align*}
\end{bemerkung}

\begin{definition}
$p\in\MdP$ heiße Zwillingsprimzahl $\equizu p, p+2 \in \MdP$

$\{p,p+2\}$ heißt Primzahlzwilling
\end{definition}

\textbf{Frage:} Gibt es unendlich viele Primzahlzwillinge? Kein
Mensch hat eine Idee, wie das zu zeigen ist.

\begin{satz}[Primzahlzwillingsatz von Viggo Brun, ca. 1915]
\[ \sum_{p \text{ Zwillingsprimzahl}} \left(\frac1p + \frac 1 {p+2}\right) < \infty \]
\end{satz}

\textbf{Pierre de Fermat} (1601 -- 1665) schreibt auf den Rand
seines Exemplars von Arithmetica des Diophant: \glqq Die Gleichung
$x^n+y^n=z^n$ (mit $n\in\MdN$, $n>2$) hat keine Lösung mit $x,y,z
\in \MdNp$\grqq. \textbf{Heute}: Fermat hat recht. (Wiles 1995/96)

Fermat schrieb auch: Die Zahlen $F_n = 2^{(2^n)}+1$ sind prim. Die
Aussage ist ok für $n=0,1,2,3,4$. \textbf{Euler} konnte zeigen, dass
$F_5  = 4294967297 = 641 \cdot 6700417$. Noch 2000 ist unbekannt, ob
$F_{24}$ prim ist.

Möglichkeiten:
\begin{enumerate}
\item Kein $F_n$ mit $n>24$ ist prim.
\item Nur endlich viele $F_n$ sind prim.
\item $\#\{F_n|F_n\in \MdP\} = \infty$
\item $\#\{F_n|F_n\notin \MdP\} = \infty$
\end{enumerate}
Niemand weiß oder vermutet, was richtig ist, keine Beweisideen!

\begin{definition}
$M_p = 2^p - 1 \text{ heißt $p$-te Mersenne-Zahl}$
\end{definition}

\begin{satz}
$M_p$ ist höchsten dann prim, wenn $p\in\MdP$
\end{satz}
\begin{beweis}
Übungsaufgabe
\end{beweis}

Die größte bekannte Primzahl ist seit längerem eine
Mersenne-Primzahl, da es gute Tests gibt, z.B. Lucas/Lehmer,
verbessert von Grandall. Heute: $M_p\in\MdP$ für $p=3021327$,
$M_p>10^{2000000}$.

Eine weitere Frage an Primzahlen ist die nach der Verteilung von
$\MdP$ in $\MdN$. Bei dieser Frage spielt die Analysis eine Rolle.
\begin{satz}[Elementarer Primzahlsatz]
Sei $\Pi(x) = \#\{p\in\MdP|p\le x\}$ ($x\in\MdR$). Dann gilt:
\[ \Pi(x) \sim \frac{x}{\log x} \text{ (fast asymptotisch gleich)} \]
\end{satz}
Der Satz wurde 1792 von Gauß vermutete und 1896 von Hadamard und von
de la Vaille-Poussin nach Vorarbeiten von Riemann bewiesen

\begin{folgerung}
Sei $p_n$ die $n$-te Primzahl der Größe nach ($p_1=2, p_2=3,
p_3=5,\ldots$). Dann gilt:
\[ p_n \sim n\cdot \log n \quad (n\to\infty) \]
\end{folgerung}
\begin{beweis}
$p_n = x \implies n=\Pi(x)$
\begin{align*}
\lim_{n\to\infty} \frac{n\cdot \log n}{p_n} &= \lim_{n\to\infty} \frac{\Pi(x) \log \Pi(x)}{x}\\
&= \lim_{n\to\infty} \frac{\Pi(x)}{x/\log x} \cdot \frac{x}{\log x} \cdot \frac{\log \Pi(x)}{x} \\
&= \lim_{n\to\infty} \frac {\log  \Pi(x)}{\log x}\\
&= \lim_{n\to\infty} \frac1{\log x} \cdot \log \frac{\Pi(x)}{x/\log x} x/\log x\\
&= \lim_{n\to\infty} \frac{1}{\log x} \left( \log \frac{\Pi(x)}{x/\log x} + (\log x - \log \log x) \right) \\
&= 1 - \lim_{x\to\infty} \frac{\log(\log x)}{\log x}\\
&= 1 - \lim_{t\to\infty} \frac{\log t}t\\
&= 1 - \lim_{n\to\infty}\frac n {e^n} = 1
\end{align*}
\end{beweis}

\begin{folgerung}
$\forall \ep > 0 \ \exists N\in\MdN \ \forall x\ge N \ \exists
p\in\MdP$:
\[ x \le p \le x(1+\ep) \]
\end{folgerung}

\textbf{Riemann} (1826--66): \glqq Über die Anzahl der Primzahlen
unter einer gegebenen Größe\grqq{}  stellt Zusammenhang mit Riemanns
$\zeta$-Funktion her.
\[ \zeta(s) = \sum_{n\in\MdNp} \frac{1}{n^s}, s\in\MdC \]
$\zeta(s)$ konvergiert für $\Re s>1$ und hat eindeutige Fortsetzung
zum analytischer Funktion $\MdC\setminus1\to\MdC$ mit Pol in $s=1$.
Man kann zeigen: Primzahlsatz $\equizu$ $\zeta$ hat keine Nullstelle
mit $\Re \ge 1$.

\textbf{Vermutung}: Alle nichtreellen Nullstellen von $\zeta$ liegen
auf $\frac12+i\MdR$. Gauß vermutet: Besser als $x/\log x$
approximiert
\[ \text{li}(x) = \int_2^x \frac {du}{\log u} \quad \text{(Integrallogarithmus).} \]
Man will möglichst gute Abschätzung des Restglieds $R(x) = |\Pi(x) -
\text{li}(x)|$.

\textbf{Fakt}: Je größer die nullstellenfreien Gebiete von $\zeta$,
desto bessere Restgliedabschätzung möglich. Demnach: Beste
Restgliedabschätzung möglich, wenn Riemanns Vermutung stimmt.
\[ R(x) \le \text{Const} \cdot x^{\frac 12} \log x \]
\textbf{Fakt 2:} Von der Qualität der Restgliedabschätzung hängen in
der Informatik viele Aussagen über die theoretische Effektivität von
numerischen Algorithmen ab.

%2. Vorlesung!!!
\section{Elementare Teilbarkeitslehre in integren Ringen}
In dieser Vorlesung gilt die Vereinbarung, dass ein Ring
definitionsgemäß genau ein Einselement $1_R$ besitzt.

\begin{definition}
Ein Ring $R$ heißt \emph{integer}, wenn gilt:
\begin{enumerate}
\item $R$ ist kommutativ.
\item $\forall a,b \in R:\ ab=0 \iff a=0 \vee b=0$.
\end{enumerate}
\end{definition}
\begin{beispiel}
    Jeder Unterring eines Körpers ist integer.
\end{beispiel}
\begin{definition}
    Die Menge
    $$
    R^{\times}:=\{a \in R|\ \exists x \in R:\ ax=1=xa\}
    $$
    heißt \emph{Einheitengruppe} $R^{\times}$ des (allgemeinen)
    Ringes $R$.\\
    Leicht zu sehen ist, dass $R^{\times}$ eine Gruppe ist, $x$ ist
    das eindeutig bestimmte Inverse $a^{-1}$ von $a$.
\end{definition}
\begin{beispiel}
    $\MdZ^{\times}=\{\pm 1\}$ (klar!)\\
    $\MdZ^{n \times n}$ ist der Ring der ganzzahligen $n\times
    n$-Matrizen, $GL(\MdZ)=(\MdZ^{n \times n})^{\times}$.
    Beispielsweise für $n=2$:\\
    $$
    A=\left(
        \begin{array}{cc}
          2 & 5 \\
          1 & 3 \\
        \end{array}
      \right),\ A^{-1}=\left(
        \begin{array}{cc}
          3 & -5 \\
          -1 & 2 \\
        \end{array}
      \right),\ A\,A^{-1}=I=A^{-1}\,A \Rightarrow A \in GL_2(\MdZ).
    $$\\
    $R=K[X]$ ist der Ring der Polynome in $X$ über dem Körper $K$.
    $R^{\times}=\{\alpha \in K^{\times}=K\backslash \{0\}\}$ (Konstante,
    von $0$ verschiedene Polynome)\\
    $\MdZ, K[X]$ sind integere Ringe.
\end{beispiel}
Ab jetzt sei $R$ ein integerer Ring, $a,b,c,d,x,y,u,v,w \in R$.

\textbf{Problem}: Gleichung $ax=b$ mit der Variablen $x$.
Beispielsweise ist $3x=5$ in $R=\MdZ$ nicht lösbar, $3x=6$ hingegen
schon.

\begin{definition}
    \[ a|b \iff \exists x\in R:\ ax=b \]
    Sprechweise: $a$ teilt $b$,
    $b$ ist Vielfaches von $a$, $a$ ist Teiler von $b$.

    $\neg a|b \iff a \not |\ b$ ($a$ teilt nicht $b$).
\end{definition}
\begin{beispiel}
    $R=\MdZ$: $3 \not|\ 5,\ 3|0,\ \pm 3,\ \pm 6 \dotsc$\\
    $R=K[X]$: $(X-1)|(X^2-1)$.\\
    In jedem $R:\ \forall a \in R: 1|a \text{ (denn $a=a\cdot 1$ )}\
    \wedge\
    a|0$ (denn $0=0\cdot a$).
\end{beispiel}
\begin{satz}[Elementare Teilbarkeitseigenschaften]
\begin{enumerate}
    \item $|$ ist mit $\cdot$ verträglich:\\
          $a|b\ \wedge\ c|d \Rightarrow ac|bd$.
    \item $|$ ist mit Linearkombinationen verträglich:\\
          $a|b\ \wedge\ a|c \Rightarrow\ \forall x,y\in R:\
          a|xb+yc.$
    \item $|$ ist eine transitive und reflexive Relation und für
    $a\neq 0$ gilt:\\
    $ a|b\ \wedge\ b|a \iff \exists e \in R^{\times}:\ a=eb$.
\end{enumerate}
\end{satz}
\begin{beweis}
    Treppenbeweis $\copyright$ Dr. Rehm.
\end{beweis}
\begin{bemerkung}
$(2)$ hat einen häufigen Spezialfall: $a|b\ \wedge\ a|c \Rightarrow
a|b\pm c$.\\
\textbf{Anwendungsbeispiel}: $a|b^2\ \wedge\ a|b^2+1 \Rightarrow
a|\underbrace{b^2+1-b^2}_{=1}$.\\
\textbf{Folgerung}: $e\in R^{\times}:\ a|b \iff ea|b \iff a|eb$.\\
\textbf{Grund}: $b=xa=(xe^{-1})ea$.\\
Merke: Einheitsfaktoren ändern Teilbarkeit nicht!\\
\textbf{Folge 2}: $R$ ist disjunkte Vereinigung aller Mengen
$R^{\times}a=\{ea|e\in R^{\times}\}$.\\
\textbf{Grund}: $u\in R^{\times}a\cap R^{\times}b\iff u|a\ \wedge\
a|u\ \wedge\ u|b\ \wedge b|u$, also
$R^{\times}a=R^{\times}u(=R^{\times}b$, $eu\in R^{\times}a
\Rightarrow R^{\times}u\subset R^{\times}a$, genauso zeigt man
$R^{\times}a \subset R^{\times}u$.
\end{bemerkung}
\begin{definition}[Normierung]Auswahl je eines festen $a_{nor}$ in
$R^{\times}a$. Man wählt immer $e_{nor}=1,\ 0_{nor}=0$.\\
Standard-Normierung: $R=\MdZ,\ R^{\times}a=\{\pm a\},\
a_{nor}=\max\{R^{\times}a\}=|a|$.\\
$R=K[X],\ 0\neq f=\alpha_0+\alpha_1 X + \dotsb + \alpha_n X^n$ mit
$\alpha_n \neq 0$. Dann ist $f_{nor}=\frac{1}{\alpha_n}f$.
\end{definition}
Klar ist: Jedes $a\in R$ hat die trivialen Teiler $e\in R^{\times}$
und $ea,  e \in R^{\times}$. Nichttriviale Teiler heißen auch echte
Teiler.
\begin{beispiel}
 $R=\MdZ$, triviale Teiler von $6$ sind $\pm 1,\ \pm 6$. Echte
 Teiler sind $\pm 2,\ \pm 3$.
\end{beispiel}
\begin{definition}
    \begin{enumerate}
        \item $a\in R$ heißt unzerlegbar oder irreduzibel, falls
            $a\neq 0,\ a \notin R^{\times}$ und $a$ hat nur triviale Teiler.
        \item $R=\MdZ$. $p\in\MdZ$ heißt Primzahl $\iff$ p normiert
            und irreduzibel.
        \item $R=K[X]$. $f \in R$ heißt Primpolynom $\iff$ $f$
            irreduzibel.
    \end{enumerate}
\end{definition}

\subsubsection*{Größter gemeinsamer Teiler und kleinstes Gemeinsames
Vielfaches}

\begin{definition}
    $d$ heißt ein größter gemeinsamer Teiler von
    $a_1,a_2,\dotsc,a_n$ $:\iff$
    \begin{enumerate}
        \item $d|a_1\ \wedge\ d|a_2\ \wedge\ \dotsb \ \wedge\ d|a_n$
        (d ist gemeinsamer Teiler)
        \item $u|a_1\ \wedge\ u|a_2\ \wedge\ \dotsb \ \wedge\ u|a_n \Rightarrow u|d$
    \end{enumerate}
\end{definition}
\begin{bemerkung}
    \begin{enumerate}
        \item Bei $R=\MdZ$ ist ein bezüglich $\leq$ größter
        gemeinsamer Teiler ein normierter ggT.
        \item Eindeutigkeit des ggT: Ist $d$ ein ggT von
        $a_1,a_2,\dotsc,a_n$, so ist auch $d_{nor}$ ein ggT und $d_{nor}$
        ist durch $a_1,a_2,\dotsc,a_n$ eindeutig bestimmt:
        $d=d_{nor}=\ggt(a_1,a_2,\dotsc,a_n)$\\
        \textbf{Grund}: $e \in R^{\times}$ spielt bei Teilbarkeit keine Rolle, und
        $d_{nor}=ed$ für ein $e \in R^{\times}$. Sind $d,d'$ ggTs von $a_1,a_2,\dotsc,a_n$
        $\Rightarrow d|d'\ \wedge\ d'|d \iff d'=ed\text{, da
        normiert}\Rightarrow d=d'$.
    \end{enumerate}
\end{bemerkung}
Der kgV wird analog zum ggT unter Umkehrung aller
Teilbarkeitsrelationen definiert:

\begin{definition}
    $k$ heißt ein kgV von $a_1,a_2,\dotsc,a_n$ $:\iff$
        \begin{enumerate}
        \item $a_1|k\ \wedge\ a_2|k\ \wedge\ \dotsb \ \wedge\ a_n|k$
        (k ist gemeinsames Vielfaches)
        \item $a_1|u\ \wedge\ a_2|u\ \wedge\ \dotsb \ \wedge\ a_n|u \Rightarrow k|u$
    \end{enumerate}
\end{definition}
Die Eindeutigkeitsaussage des ggT gilt für den kgV ebenfalls.

\begin{satz}[Euklids Primzahlsatz]
    Für $R=\MdZ$ gilt:
    \[\#\MdP=\infty\]
\end{satz}
\begin{beweis}
    Es seien $p_j,\ j=1,2,\dotsc,n$ paarweise verschiedene Primzahlen.
    Betrachte $1+\prod p_i>0$.\\
    \textbf{Aussage}: Ist $a\in \MdN,\ a > 1$, so ist $\min\{d\in \MdN:
    d|a\}$ eine Primzahl und das Minimum existiert wegen $a|a$.
    Benutzt, dass jede Teilmenge der natürlichen Zahlen eine
    kleinste Zahl enthält $\Rightarrow$ Behauptung, da ein echter Teiler
    kleiner wäre  $\Rightarrow \exists p \in R:\ p|1+\prod p_j$.\\
    Wäre $p=p_j$ für ein $j\in\{1,2,\dotsc,n\}$, so $p|\prod p_j$.
    $p|\underbrace{1+\prod p_j-1(\prod p_j)}_{=1} \iff p|1
    \Rightarrow p\in \MdZ^{\times} \Rightarrow$ Widerspruch.
\end{beweis}
\section{Primzerlegung in Euklidischen Ringen, Faktorielle Ringe}
In diesem Abschnitt sei $R$ integerer Ring, $a,b,c,d,\dotsc \in
R$.\\
\textbf{Sprechweise}: $a=qb+r$. Man sagt $r$ ist der Rest bei
Division von $a$ durch $b$, $q$ ist der Quotient (Division mit
Rest).\\
\textbf{Mathematischer Wunsch}: Rest $r$ soll im geeigneten Sinn
kleiner sein als der Divisor $b$. Man benötigt dafür eine
Größenfunktion $gr:\ R\mapsto \MdN$.
\begin{definition}
    Ein Ring $R$, beziehungsweise ein Paar $(R,gr)$ heißt euklidisch
    $:\iff$
    \begin{enumerate}
        \item $R$ ist integer\\
        \item Man hat Division mit Rest, das heißt:\\ $\forall a,b \in
        R,\ b\neq 0,\ \exists q,r\in R: a=qb+r$, wobei $r=0$ oder
        $gr(r)<gr(b)$.
    \end{enumerate}
\end{definition}

Es ist $(\MdZ, |\cdot|)$ ein euklidischer Ring.
%Dritte Vorlesung!!!
\begin{beweis}
O.b.d.A: $b>0$, da $|b| = gr(b) = gr(-b)$.

$q=\lfloor \frac a b \rfloor$ ist geeignet: $0\le \frac a b - q < 1
|\cdot b \Rightarrow 0 \le a - qb  = r < b \Rightarrow gr(r) = |r| =
r<b = |b| =gr(b)$
\end{beweis}

Viele Porgrammiersprachen, etwa MAPLE, bieten einen modulo-Operator:\\
\texttt{r := (a mod b) }$= a - \lfloor \frac a {|b|} \rfloor$.

Im $K[X]$ ist die Division mit Rest möglich bezüglich $gr(f) :=
\grad f = n$, ($f\ne 0$).

Der Ring $R = \MdZ + \MdZ i \subset \MdC$, also $R =
\{x+iy|x,y\in\MdZ\}$ heißt "`Ring der ganzen Gaußschen Zahlen"'. $R$
ist euklidisch mit $gr(x,iy) = |x+iy| = \sqrt{x^2+y^2}$. Die Idee
für die Division mit Rest ist: Suche einen Gitterpunkt nahe $\frac a
b$. (siehe Übung)

\begin{lemma}
$R$ integer, $a= qb+r$, $a,b,q,r\in R$. Dann gilt
\[ \ggt(a,b) = \ggt(b,r)\,, \]
und falls eine Seite existiert, so auch die andere.
\end{lemma}

\begin{beweis}
Sind $u,v \in R$, so kann Existenz und $\ggt(u,v)$ abgelesen werden
an
\[ T(u,v) = \{d\in R \big| d|u \wedge d|v \}\,,\]
der Menge der gemeinsamen Teiler. Es ist aber $T(a,b) = T(b,r)$:\\
\glqq$\subseteq$\grqq: $d|a \wedge d|b \implies d|r$ (Linearkombination)\\
\glqq$\supseteq$\grqq: $d|r \wedge d|b \implies d|a$
(Linearkombination)
\end{beweis}

Euklids glänzende Idee ist nun: Bei der Division mit Rest
verkleinert der Übergang von $(a,b)$ zu $(b,r)$ das Problem. Sein
Algorithmus ist wie folgt:

\lstset{morecomment=[l]{\#}, texcl=True, commentstyle=\textrm}
\begin{lstlisting}
ggT := proc(a,b);      # Prozedur, die \texttt{ggT} $ = \ggt(a,b)$
aus $a$,$b$ berechnet if b = 0
  then normiere(a)     # es ist immer $\ggt(a,0) = a_\text{nor}$
  else ggT(b, a mod b) # terminiert wegen $gr(a \mod b) < gr(b)$
fi
\end{lstlisting}

Idee: $r$ ist Linearkombination von $a$ und $a,b$. Die Hoffnung
dabei ist: Auch $d:= \ggt(a,b)$ lässt sich linear kombinieren.

\begin{satz}[Satz der Linearkombination des ggT]\label{satz:LinKom}
Sei $R$ ein euklidischer Ring. Dann existiert $d=\ggt(a,b)$ für alle
$a,b\in R$ und ist als $R$-Linearkombination von $a,b,$ darstellbar:
\[ \exists x,y\in R: d= \ggt(a,b) = xa + yb \]
\end{satz}

\begin{beweis}
\begin{itemize}
\item[I] Falls $b=0$ ("`Induktionsanfang"') gilt $d= a_\text{nor}= e\cdot a + 0 \cdot b$ mit geeignetem $e\in R^\times$
\item[II] Falls $b\ne 0$: Division mit Rest $a  = qb + r$ \\
Falls $r=0$ ist $d=b_\text{nor}$, fertig!\\
Falls $r\ne0$, so gilt $\ggt(a,b) = \ggt(b,r) = d$ und $gr(r)<gr(b)$

Induktionshypothese: $\exists x_0,y_0\in R$: $d=x_0b + y_0 r = x_0 b + (a-qb)y_0  = y_0 a + (x_0-qy_0)b = xa+yb$\\
Induktionsschritt geleistet.
\end{itemize}
\end{beweis}


Die Idee ist, dass ein Ring \emph{faktoriell} heißt, wenn man in ihm
eine eindeutige Primzerlegung, wie aus $\MdZ$ bekannt, hat. Ein Ziel
der Vorlesung ist die Feststellung, dass euklidische Ringe
faktoriell sind (Euler-Faktoriell-Satz).

\begin{definition}
Ein Ring $R$ heißt faktoriell (älter: "`ZPE-Ring"') wenn gilt:
\begin{itemize}
\item[(i)] $R$ ist integer
\item[(ii)] Es gibt eine Menge $P\subseteq R$, bezüglich der jedes $a\in R$ mit $a\ne 0$ eine "`eindeutige Primzerlegung"' hat, also:

$\exists e(a)\in\MdR^\times \ \exists v_p(a) \in \MdN$, mit nur
endlich vielen $v_p(a)\ne 0$ mit
\[a = e(a) \cdot \prod_{p\in P} p^{v_p(a)} \text{ "`Primzerlegung von $a$"'}\]
Eindeutigkeit heißt: Durch $a$ sind $e(a)$ und alle $v_p(a)$
eindeutig bestimmt.
\end{itemize}
\end{definition}

Der Fall $R=\MdZ$ ist aus der Schule bekannt, und wird nicht
bewiesen. Ein Beispiel ist $-100 = -1 \cdot 2^2 \cdot 5^2$, also
$e(-100)=-1$, $v_2(-100)=v_5(-100)=2$ und $\forall p\in P, p\ne 2,
p\ne5: v_p(-100)=0$

Im Fall $R=K$, wobei $K$ ein Körper ist, gilt
$R^\times=\MdK\setminus\{0\}$ und $ P=\emptyset$.

Ist $R$ faktoriell, so ist die Standardnormierung \[a_\text{nor} =
\prod_{p\in P} p^{v_p(a)}\,.\]

\begin{bemerkung}
$P$ besteht aus unzerlegbaren Elementen. Hätte man nämlich $p=uv$
mit echten Teilern $u,v$, so gilt $u,v\notin R^\times$, also
$\forall p_1,p_2 \in P$: $v_{p_1}>0,v_{p_2}>0$. Nun haben wir zwei
Primzerlegungen, da $v_p(p) = 1$, $\forall q\in P, q\ne p, v_q(p)=0$
und damit $p=1 \cdot p^1 = 1\cdot p_1^1 \cdot p_2^1$
\end{bemerkung}

Ein Zweck der Primfaktorzelegung ist, dass die Multiplikation in $R$
auf die $R^\times$ und die Addition in $\MdN$ zurückgeführt werden
kann. Denn mit $a=e(a) \cdot \prod_{p\in P} p^{v_p(a)}$, $b=e(b)
\cdot \prod_{p\in P} p^{v_p(b)}$ gilt:
\begin{align*}
ab &= e(a) \cdot e(b) \cdot \prod_{p\in P} p^{v_p(a) + v_p(b)} \\
   &= e(ab)\cdot \prod_{p\in P} p^{v_p(ab)}
\end{align*}
Aus der Eindeutigkeit folgt nun: $e(ab) = e(a) \cdot e(b)$ und
$v_p(ab) = v_p(a)+ v_p(b)$. $v_p(a)$ heißt "`additiver $p$-Wert
von $a$"'. $v_p$ heißt (additive) $p$-adische Bewertung von $R$.

Ein weiterer Zweck liegt in der Rückführung der Teilbarkeit auf
$\le$ in $\MdN$: Für $a,b\ne 0$ gilt \[b|a \iff \ \forall p\in P:
v_p(b) \le v_p(a) \] Begründung: $nb=a \implies v_p(b) \le v_p(b) + \underbrace{v_p(n)}_{\ge 0} = v_p(a)$

Eine Folgerung davon ist, dass $\forall p\in P$ gilt:
$v_p(\ggt(a,b)) = \min\{v_p(a), v_p(b)\}$ und allgemeiner:
$v_p(\ggt(a_1,\ldots,a_n)) = \min\{v_p(a_1),\ldots,v_p(a_n)\}$.
(Damit das auch bei $a = 0$ Sinn macht, kann man $v_p(0) = \infty$
definieren, was auch üblich ist.) Ebenso gilt: $\forall p\in P$:
$v_p(\kgv(a,b)) = \max\{v_p(a), v_p(b)\}$.

Allerdings ist zur Bestimmung von $\kgv(a,b)$ folgener Algorithmus
besser als der Weg über die Primfaktorzelegung:
\begin{enumerate}
\item Berechne $\ggt(a,b)$ mit Euklids Algorithmus
\item Verwende: Sind $a,b$ normiert, so gilt:
\[ \ggt(a,b) \cdot \kgv(a,b) = ab \]
\end{enumerate}
Begründung: $\min\{v_p(a), v_p(b)\} + \max\{v_p(a), v_p(b)\} =
v_p(a) + v_p(b)$ und $ab = \prod_{p\in P} p^{v_p(a) + v_p(b)}$

Anwendungsbeispiel: Ist $m,n\in\MdN_+$, so gilt $\ggt(a^m,b^n)=1
\iff \ggt(a,b)=1$

Zusammenfassung: Für alle $a,b\in R$, $a,b\ne 0$ gilt:
\begin{itemize}
\item $v_p(ab) = v_p(a) + v_p(b)$
\item $a\in R^\times \iff \forall p\in P: v_p(a) = 0$
\item $v_p(a+b) \ge \min\{v_p(a),v_p(b)\}$
\item $v_p(\ggt(a,b)) = \min\{v_p(a),v_p(b)\}$
\end{itemize}

Noch zu zeigen: $v_p(a+b) \ge \min(v_p(a),v_p(b))$.\\
O.B.d.A: $v_p(a) \le v_p(b)$, also $\min(v_p(a),v_p(b)) = v_p(a)$.
$a = p^{v_p(a)} \cdot a_0$, $b = p^{v_p(b)}b_0$ mit $a_0, b_0 \in \MdR$.\\
$a+b = p^{v_p(a)}(a_0 + p^{v_p(b)-v_p(a)}b_0)$ $\Rightarrow
p^{v_p(a)} | a+b \Rightarrow v_p(p^{v_p(a)}) = v_p(a) \le v_p(a+b)$

\begin{bemerkung}
Ist $R$ (integrer Rang) enthalten in einem Körper, so ist $K = \{\frac{a}{b} = x | a, b \in R, b \not=0\}$ ein Körper.\\
Man kann $v_p$ auf $K$ ausdehnen: $v_p(x) = v_p(a) - v_p(b)$ ($x \ne
0$) Ist $R$ faktoriell, so hat man die "`Primzerlegung"' von $x =
\frac{a}{b}:$
\[ x = e(x) \cdot \prod_{p \in P} p^{v_p(x)} \]
mit $e(x) \in R^\times, v_p(x) \in \MdZ$. Nur endlich viele $v_p(x)$
sind $\not=$ 0.\\$x \in R \Leftrightarrow v_p(x) \ge 0\ (\forall p
\in P)$. Die Rechenregeln 1-4 gelten auch auf $K$ (siehe $R$ [Beweis
leicht]).
\end{bemerkung}

\begin{beispiel}
$v_7(\frac{7}{25}) = 1, v_5(\frac{7}{25}) = -2, v_p(\frac{7}{25}) =
0$ sonst
\end{beispiel}

\begin{lemma}
\label{lemma1} Sei $R$ euklidisch, dann gibt es eine
"`Größenfunktion"' $gr: R \to \mathbb{N}$ für die (zusätzlich) gilt:
\begin{itemize}
\item Ist $e \in R^\times, a \in R, a\not= 0: gr(ea) = gr(a)$
\item Ist $b$ ein \emph{echter} Teiler von $a \not= 0$, so ist $gr(b) < gr(a)$
\end{itemize}
\end{lemma}

\begin{beweis}
\textbf{Idee:} Ist $gr$ die gegebene Größenfunktion, so erfüllt
$$gr^*(a) = \min\{gr(ea) | e \in R^\times\}$$
die beiden Punkte des Lemmas. (Beweis wird auf die Homepage
gestellt!)
\end{beweis}

Für $R = \MdZ$ und $R = K[X]$ sind beide ohnehin richtig.\\
(z.B. $\MdZ, gr(a) = |a|, b$ echter Teiler. $a = bu, u \in
\MdZ^\times = \{\pm 1 \} \Rightarrow |a| > 1 \Rightarrow gr(a) = |a| =
|b||u|, gr(b) = |b| = \frac{|a|}{|u|} < |a| = gr(a)$. Ähnlich in
$K[x]$)

\begin{lemma}
\label{lemma2} $R$ sei euklidisch, $p \in R$ irreduzibel, $a,b \in
R$. Dann gilt:
\[p | ab \implies p | a \text{ oder } p|b\]
\end{lemma}

\begin{beweis}
O.B.d.A.: $p$ normiert, die normierten Teiler von $p$ sind $1$ und $p$.\\
\underline{Annahme:} $p \nmid a \wedge p \nmid b$\\
Falls $p \nmid a \Rightarrow \ggt(p,a) = 1$ \\
(anderenfalls $\ggt(p,a) = p$, damit $p | a$, Widerspruch!). \\
$p \nmid b \Rightarrow \ggt(p,b) = 1$. \\
Nach dem Linearkombinations-Satz: \\
$$\exists x_0, y_0, x_1, y_1 \in R: 1 = x_0p + y_0a = x_1p + y_1b$$
$$1 = 1 \cdot 1 = \underbrace{(...)}_{\in R}p + y_0y_1ab$$
$p | ab \Rightarrow p | 1 \Rightarrow p \in R^\times$, also nicht
irreduzibel, Widerspruch!
\end{beweis}

\begin{beweis}
Des Euler-Faktoriell-Satzes: $R$ euklidisch $\Rightarrow R$ faktoriell.\\
$P = \{p_{\text{nor}} | p \text{ irreduzibel}\}$ (z.B. $P = \MdP$ für $R = \MdZ$).\\
\textbf{Existenz} der Primzerlegung für $a\in R$ ($a \ne 0$)
\begin{itemize}
\item[I] Fall: $a \in R^\times$, Primzerlegung $a = e(a), \forall p \in P$: $v_p(a) = 0$
\item[II] Fall: $a$ irreduzibel $\Rightarrow p = a_{\text{nor}} \in P$, $a = ea_{\text{nor}} = ep, e \in R^\times, e(a) := e, v_p(a) = \begin{cases}1 & q = p\\0 & q \not= p\end{cases}$
\end{itemize}
Allgemeiner Fall wird durch Induktion nach $gr(a)$ bewiesen.\\
Es ist nur noch $a \in R, a \not= 0, a \not\in R^\times, a$ nicht
unzerlegbar zu betrachten $\Rightarrow a = u \cdot v$ mit $u,v$
echte Teiler. Induktions-Hypothese mit Hilfe des Lemma \ref{lemma1}
$\Rightarrow gr(u) < gr(a) \wedge gr(v) < gr(a)$, also haben $u,v$
Primzerlegung $\Rightarrow$ (Durch Ausmultiplizieren) $a$ hat
Primzerlegung: $e(a) = e(u) \cdot e(v) \in R^\times$, $v_p(a) =
v_p(u) + v_p(v)$

\textbf{Eindeutigkeit:}
$a = e(a) \cdot \prod p^{v_p(a)} = e'(a) \cdot \prod p^{v_p'(a)}$ seien zwei Primzerlegungen.\\
Zu zeigen: $e(a) = e'(a)$, $\forall p \in P: v_p(a) = v_p'(a)$\\
Induktion nach $n =: \sum_{p \in P}(v_p(a) + v_p'(a)) \in \MdN$\\
Induktionsanfang: $n=0 \Rightarrow \forall p: v_p(a) = 0 = v_p'(a) \Rightarrow e(a) = e'(a)$\\
Induktionsschritt: $n > 0 \Rightarrow \exists p: v_p(a) > 0 \vee v_p'(a) > 0$, O.B.d.A.: $v_p(a) > 0 \Rightarrow p|a = e'(a) \prod_{q\in P}q^{v_q'(a)}$\\
Aus Lemma \ref{lemma2} leicht induktiv: $p|a_1 \cdot ... \cdot a_n \Rightarrow \exists j: p | a_j \Rightarrow \underbrace{p|e'(a)}_{\text{geht nicht}} \vee \exists q\in\MdP: p|q^{v_q'(a)} \Rightarrow p|q$\\
$\Rightarrow p$ ist normierter Teiler von $q \Rightarrow p = q$ ($p = 1$ geht nicht) $\Rightarrow p | p^{v_p'(a)} \Rightarrow v_p'(a) > 0$\\
$\tilde{a} = e(a) p^{v_p(a)-1} \prod_{q \not= p}p^{v_p(a)} = e'(a)p^{v_p'(a)-1} \prod_{q \not= p}q^{v_p'(a)}$\\
Zwei Primzerlegungen von $\tilde{a}$ mit $n-2$ statt $n$.
Induktionshypothese anwendbar auf $\tilde{a} \Rightarrow e(a) =
e'(a), \forall q\not= p: v_p(a) = v_q'(a)$. $v_p(a) -1 = v_p'(a) -1
\Rightarrow$ Induktionsschritt geleistet.
\end{beweis}

Primzerlegung hat viele Anwendungen, z.B.: $\ggt(a,b) = 1
\Rightarrow \ggt(a^n, b^m) = 1$

\begin{satz}[Irrationalitätskriterium]
Sei $\alpha \in \MdC$ eine Nullstellen von $f = X^m + \gamma_1
X^{m-1} + ... + \gamma_{m-1}X + \gamma_m \in \MdZ[X]$ (d.h.
$\gamma_1,...,\gamma_m \in \MdZ$) Ist dann $\alpha \notin \MdZ$, so
$\alpha \notin \MdQ$.
\end{satz}

\begin{beweis}
Annahme $\alpha \in \MdQ$, $\alpha = \frac{z}{n}, z \in \MdZ, n \in \MdN_+$, $\ggt(z, n) = 1$\\
$0 = f(\frac{z}{n}) = \frac{z^m}{n^m} + \gamma_1\frac{z^{m-1}}{n^{m-1}} + ... + \gamma_{m-1}\frac{z}{n} + \gamma_m$, multiplizieren mit $n^m \Rightarrow$\\
$0 = z^m + n\underbrace{(...)}_{\in \MdZ} \Rightarrow n|z^m \Rightarrow n|\ggt(z^m, n) = 1$, da $\ggt(z, n) = 1$ (s.o.)\\
$n|1 \Rightarrow \alpha = \frac{z}{n} = z\in \MdZ$.
\end{beweis}

\textbf{Anwendung:} z.B. auf $f = X^k - a, a \in \MdZ (k > 1)$. Ist
$a$ keine $k$-te Potenz in $\MdZ, \alpha$ eine Nullstelle von $f$ in
$\MdC$ (sozusagen $\alpha = \sqrt[k]{a}$), so ist $\alpha$
irrational.

[$\alpha \in \MdZ: a = \alpha^k$ ist $k$-te Potenz in $\MdZ$] Tritt
zum Beispiel ein, wenn $\exists p \in \MdP: k \nmid v_p(a)$ (denn $a
= z^k \Rightarrow v_p(a) = k \cdot v_p(z)$. Etwa $\sqrt[k]{q}, q \in
\MdP$ ist immer irrational, z.B. $\sqrt{2}$.

\paragraph{Die erste Grundlagenkrise der Mathematik}
Die Pythagoräer glaubten, alle Naturwissenschaften seien durch $\MdN$ "`mathematisierbar"'. Zum Beispiel wurde Folgendes als selbstverständlich betrachtet:\\
Man kann kleinen Einheitsmaßstab $e$ (verdeutlicht durch einen gezeichneten Streckenstab mit kleinen Einheiten) wählen, so dass die Strecke $a$ und die Strecke $b$ in der Form $a = n \cdot e, b = m \cdot e$ ist, mit $n, m \in \MdN \Leftrightarrow \frac{b}{a} \in \MdQ$.\\
Modern ist die Aussage $\frac{b}{a} = \sqrt{2} \Rightarrow $ Seite und Diagonale erfüllen nicht dem Glauben.\\
Der Glaube besagt: Nur natürliche und rationale Zahlen sind Zahlen.
$\Rightarrow$ Die Länge einer Strecke ist keine Zahl.

Der Dozent glaubt, dies hat die Griechen daran gehindert "`reelle Zahlen"' zu erfinden, d.h. mit Längen von Strecken wie in einem Körper zu rechnen (wirkt über 1000 Jahre, relle Zahlen exakt erst seit ca. 1800 exakt erklärt!).\\
Heute bekannt: Die Proportionenlehre von Eudoxos von Knidos ist
logisch äquivalent zu der Konstruktion der rellen Zahlen.

\chapter{Arithmetische Funktionen}

\section{Einführung}
\textbf{Erklärung:} Eine zahlentheoretische Funktion ist eine Abbildung $\alpha: \MdN \to \MdC$, also nichts anderes als eine Folge $\alpha_n  = \alpha(n)$ komplexer Zahlen ($n \in \MdN$).\\

\begin{beispiel}
$p_n$: $n \to p_n$ ($n$-te Primzahl)  ist eine zahlentheoretische
Funktion.
\end{beispiel}

Kurzbezeichnung: $\sum_{d|n} = \sum_{\{d \in \MdN_+ \big| d|n\}}$\\
Standardbezeichnungen (\emph{in vielen Büchern}):
\begin{itemize}
    \item $\varphi(n) = \#\{x \in \MdN | 1 \le x \le n \wedge \ggt(x,n) = 1\}$ ("`\emph{Eulersche Funktion}"')
    \item $\tau(n) = \sum_{d|n}1 = \#\{x \in \MdN ; x|n\}$
    \item $\sigma(n) = \sum_{d|n}d $ \glqq{}Teilersumme\grqq
    \item $\sigma_k(n) = \sum_{d|n}d^k$, $k\in\MdN$, also $\sigma_0 = \tau$, $\sigma_1=\sigma$
    \item $\omega(n) = \#\{p\in\MdP\big|p|n\}$
    \item $\mu(n) =
\begin{cases}
0 & \exists p\in\MdP : p^2|n \\
(-1)^{\omega(n)} & \text{sonst, d.h. "`$n$ quadratfrei"'}
\end{cases}$\quad\quad "`Möbiusfunktion"'
\end{itemize}

\textbf{Zeichen in dieser Vorlesung}:
\begin{itemize}
\item $c_a$: Konstante Funtion, also $\forall n\in\MdN: c_a(n) = a$
\item $\delta$: $\delta(n) =
\begin{cases}
1 & n=1 \\
0 & \text{sonst}
\end{cases} = \delta_{1,n}$\quad "`Kronecker-Delta"'
\item $\Pi_k(n) = n^k$ "`Potenzfunktion"'
\end{itemize}

Sprechweise für den Fall $\ggt(x,n) = 1 \iff$ $x$ und $n$ sind
"`relativ prim"'.

\begin{beispiel}
\begin{enumerate}
\item $\varphi(12) = \#\{1,5,7,11\} = 4$
\item $p\in\MdP$, $n\in\MdNp,$ $\varphi(p^n) = ?$

$\ggt(x,p^n) = 1 \iff p\not|x$\\
$\{x\in\MdNp|\ggt(x,p^n) = 1 , x \le p^n\} = \{x\in\MdNp| p \not| x, x \le p^n\}$\\
$= \{1,\ldots,p^n\} \setminus \{p,2p,\ldots,p^n\} = \{1,\ldots,p^n\}\setminus p \{1,2,\ldots,p^{n-1}\}$\\
$\varphi(p^n) = p^n-p^{n-1} = p^{n-1}(p-1) = p^n(1-\frac 1 p )$

\end{enumerate}
\end{beispiel}

% Vorlesung Di. 10.5., TeXer: nomeata

\section{Dirichlet-Reihen}
Benannt nach Peter Gustav Lejeune Dirichlet, 1805-59.

\begin{definition}
Sei $\alpha$ eine zahlentheoretische Funktion. Ist $s\in\MdR$ oder
besser $s\in\MdC$, so definiert man:
\[ L(s,\alpha) = \sum_{n\in\MdNp} \frac{\alpha(n)}{n^s} \]
\end{definition}

\begin{beispiel}
$L(s,c_1) = \zeta(s)$ ("`Riemanns $\zeta$-Funktion"')
\end{beispiel}

Wir rechnen nun formal. $\alpha,\beta$ seien zahlentheoretische
Funktionen:
%Stimmt das so?
\begin{align*}
L(s,\alpha)\cdot L(s,\beta) &= \sum_{n\in\MdNp} \frac{\alpha(n)}{n^s} \cdot \sum_{n\in\MdNp} \frac{\beta(n)}{n^s} \\
&=  \sum_{n,u\in\MdNp} \sum_{n,u; nu=m} \frac{\alpha(n) \cdot \beta(u)}{(nu)^s} \\
&= \sum_{m\in\MdNp} \frac{(\alpha * \beta)(m)}{m^s}
\end{align*}
mit der \emph{Dirichlet-Faltung}: \[ (\alpha * \beta)(n) =
\sum_{u,v\in\MdNp; uv=n} \alpha(u) \beta(v) = \sum_{d|n} \alpha(d)
\beta \left(\frac n d\right) \] Als Ergebnis erhalten wir jetzt
(formal):
\[ L(s,\alpha)\cdot L(s,\beta) = L(s,\alpha * \beta) \]

\section{Arithmetische Funktionen allgemein}

$R$ sei jetzt ein faktorieller Ring.
\begin{definition}
\[ R_\text{nor} = \{ q_{\text{nor}} | q \ne 0 \} \]
\end{definition}
(z.B.: $\MdZ_\text{nor} = \MdNp)$)

\begin{bemerkung}
$\{d|n \big | d\in R_\text{nor}\}$, $(n\ne 0)$, ist \emph{endlich}.

$n= e(n) \cdot \prod_{p\in\MdP} p^{v_p(n)}$ hat endlich viele
$v_p(n)\ne 0$, etwa $p=p_1,\ldots,p_l$

$d|n, d=\prod_{p\in\MdP} p^{m_p}$ mit $m_p\le v_{p_1}(n), \ldots,
m_{p_l} \le v_{p_l}(n)$, $m_p=0$ sonst.
\end{bemerkung}

\begin{definition}
\begin{enumerate}
\item Jede Abbildung $\alpha: R_\text{nor} \to K$ ($K$ ein Körper) heißt in dieser Vorlesung ($K$-wertige) arithmetische Funktion (auf $R$). Die Menge dieser Funktoin wird hier mit $\text{Arfun} = \text{Arfun}_{R,K}$ bezeichnet.
\item Für $\alpha,\beta \in \text{Arfun}$ wird definiert:
\begin{itemize}
\item $\alpha + \beta$ durch $(\alpha + \beta)(n) = \alpha(n) + \beta(n)$
\item $c\alpha$, $(c\in K)$, durch $(c\alpha)(n) = c \cdot \alpha(n)$
\end{itemize}
\item Dirichlet-Faltung $\alpha * \beta$ durch
\[ (\alpha * \beta) (n)  = \sum_{d|n} \alpha(d) \cdot \beta(\frac nd) \]
(Das Inverse wird mit $\alpha^{-1}$ bezeichnet, also $\alpha *
\alpha^{-1} = 1$)
\end{enumerate}
\end{definition}

\begin{satz}[Arfun-Ring-Satz]
\begin{itemize}
\item (Arfun,$+$,$*$) ist \emph{integrer} Ring und $K$-Vektorraum.
\item $\alpha\in\text{Arfun}^\times \iff \alpha(1) \ne 0$. \end{itemize}
\end{satz}

\begin{beweis}
Die Vektorraumeigenschaft wird wie in der Analysis gezeigt. Wir
zeigen die Ringeigenschaft:

Einselement ist $1_\text{Arfun} = \delta$:
\[ (\delta*\alpha)(n) = \sum_{d|n} \delta(d)\alpha(\frac nd) = \delta(1) \cdot \alpha(\frac n1) = \alpha(n) \]
Die Kommutativität von $*$ ist offensichtlich. Die Distributivregel
gilt auch:
\begin{align*}
\alpha * (\beta + \gamma)(n)
&= \sum_{d|n} \alpha(d) \cdot (\beta + \gamma) \left(\frac nd\right)\\
&= \sum_{d|n} \alpha(d) \cdot \left(\beta\left(\frac nd\right) + \gamma(\frac nd)\right)\ (\cdot\text{ ist distributiv in } \MdC) \\
&= \sum_{d|n}\left( \alpha(d) \cdot \beta\left(\frac nd\right) + \alpha(d) \cdot \gamma\left(\frac nd\right)\right)\\
&= \sum_{d|n} \alpha(d) \cdot \beta\left(\frac nd\right) + \sum_{d|n} \alpha(d) \cdot \gamma\left(\frac nd\right)\\
&= (\alpha * \beta)(n) + (\alpha * \gamma)(n)\\
&= \left( (\alpha * \beta) + (\alpha * \gamma) \right)(n)
\end{align*}
Bemerkung:\[ (\alpha * \beta) (n) = \sum_{u,v\in R_\text{nor};\
u\cdot v = n} \alpha(u) \beta(v) \]


Nun zeigen wir noch die Assoziativregel:
\begin{align*}
((\alpha * \beta) * \gamma )(n)
&= \sum_{\mathclap{u,v;\ uv=n}} (\alpha * \beta)(u) \gamma (v) \\
&= \sum_{\mathclap{uv=n;\ xy=u}} (\alpha(x)\beta(y))\gamma(v) \\
&= \sum_{\mathclap{xyv=n}} \alpha(x)\beta(y)\gamma(v) \\
&= \sum_{\mathclap{xu=n;\ yv=u}} \alpha(x)(\beta(y)\gamma(v)) \\
&= \sum_{\mathclap{xu=n}} \alpha(x)((\beta * \gamma)(u)) \\
&= (\alpha * (\beta * \gamma))(u)
\end{align*}

Den Beweis, dass Arfun ein integrer Ring ist, führen wir nur für
$R=\MdZ$, lässt sich aber mit etwas Scharfsinn auf beliebige $R$
übertragen.

$\alpha \ne 0$, $\beta \ne 0$ $\implies$ $\exists u =
\min\{x\in\MdNp| \alpha(x) \ne 0\}$, $v=\min\{y\in\MdNp|\beta(y)\ne
0\}$. $n:= uv$.

$(\alpha * \beta)(n) = \sum_{xy=n} \alpha(x)\beta(y)$. $x<u\implies
\alpha(x)=0$, $x>u \implies y = \frac nx < \frac nu = v \implies
\beta(y)=0$.

Also: $(\alpha * \beta)(n) = \alpha(u) \beta(\frac nu) = \alpha(u)
\beta(v) \ne 0$, da $K$ integer $\implies \alpha * \beta \ne 0 $

Die Existenz von Inversen: $\alpha \in \text{Arfun}^\times \iff
\exists \beta \in \text{Arfun}: \beta * \alpha = \delta
(=1_\text{Arfun})$

$\beta$ existiere $\implies 1 = \delta(1) = (\beta * \alpha)(1) =
\sum_{d|1} \beta(1) \alpha(\frac 1 d) = \beta(1) \alpha(1) \implies
\alpha(1) \ne 0$

Sei $\alpha(1) \ne 0$. Setze $\beta(1) = \frac 1 {\alpha(1)}$ (geht,
da $K$ ein Körper ist und $\alpha(1) \ne 0$). $\beta$ ist so zu
definieren, dass für $n\in R_\text{nor}$, $n\ne1$, gilt:
\begin{equation}\label{eq:2.3Stern}
    (*) \quad 0 = \delta(n) = (\beta * \alpha)(n) = \sum_{d|n}
\beta(d) \alpha(\frac nd)
\end{equation}

Induktion nach $\text{len}(n) = \sum_{p\in\MdP} v_p(n)$,
$\text{len}(n) = 0$, dann $n=1$, also OK.

Bemerkung: $d|n, d\ne n$ $(d=d_\text{nor}) \implies \text{len}(d) <
\text{len}(n)$

Induktiv darf man $\beta(d)$ schon als definiert annehmen.
$$(\ref{eq:2.3Stern}) \iff \beta(n) = -\frac 1 {\alpha(1)}
\sum_{d|n,\ d\ne m} \beta(d)\alpha(\frac d n).$$ Die rechte Seite
ist schon erklärt, die linke Seite dadurch gewonnen. $\beta$ also
rekursiv, also definiert, so dass $\beta * \alpha = \delta$. Im
Prinzip wird $\beta$ als "`Programm"' realisiert.

%Stimmt das?
%\begin{lstlisting}
%Arinv = proc (alpha, n)
%    if n = 1 then 1/(alpha(1))
%         else - 1/(alpha(1)) sum
%\end{lstlisting}
\end{beweis}
\section{Multiplikative arithmetische Funktionen}

\begin{definition}
$\alpha \in \text{Arfun}_{R,K}$, $(\alpha\ne 0)$, heiße
\emph{multiplikativ} $\iff$
\[ \forall m,n \in R_\text{nor}\text{ mit }\ggt(m,n)=1 :\quad  \alpha(mn) = \alpha(m) \alpha(n) \]
\end{definition}
$\alpha$ multiplikativ $\implies \alpha\left(\prod_{p\in\MdP}
p^{v_p(n)}\right) = \prod_{p\in\MdP} \alpha(p^{v_p(n)})$

Ein Beispiel für eine Anwendung folgt aus der Multiplikativität der
Eulerfunktion $\varphi$, welche wir später zeigen werden:
\[ \varphi(p^{v_p(n)}) = p^{v_p(n)}(1-\frac 1 p)\text { für } p\in\MdP \implies
\varphi(n) = n\cdot \prod_{\mathclap{p\in\MdP, p|n}} \left(1-\frac
1p\right)\quad\text{ "`Eulers Formel"'} \]

\begin{beispiel}
$\Pi_k$ ist multiplikativ. ($\Pi_k(n) = n^k$)
\end{beispiel}

\begin{satz}[Multiplikativitätssatz für Arfun]
\begin{enumerate}
\item Ist $\alpha \in \text{Arfun}$ multiplikativ, so ist $\alpha(1)=1$
\item Die multiplikativen Funktionen bilden eine Untergruppe von (Arfun$^\times$, $*$), also $\alpha, \beta$ multiplikativ, so auch $\alpha * \beta$ und $\alpha^{-1}$.
\end{enumerate}
\end{satz}

\begin{beweis}
    \begin{enumerate}\item
        $\alpha$ ist multiplikativ $\implies$ $\alpha(1)=\alpha(1
        \cdot 1)\stackrel{\ggt(1,1)=1}{=} \alpha(1) \cdot \alpha(1)
        \stackrel{\text{Körper!}}{\implies} \alpha(1)=1 \text{ oder }
        \alpha(1)=0$. Falls $\alpha(1)=0$, so $\forall n\in R_{nor}\
        \alpha(n)=\alpha(n \cdot
        1)\stackrel{\ggt(n,1)=1}{=}\alpha(n)\cdot
        \underbrace{\alpha(1)}_{=0}=0\ \implies\ \alpha \equiv 0$ und
        das ist nach Definition
        \emph{nicht} multiplikativ, also gilt $\alpha(1)=1$.

        \item Zu zeigen: $\alpha,\beta$ multiplikativ $\implies$ $ \alpha*\beta$
        multiplikativ und $\alpha^{-1}$ ist ebenfalls multiplikativ.
        \begin{equation}(\alpha * \beta)(n_1 n_2)=(\alpha * \beta) (n_1) \cdot
        (\alpha * \beta)(n_2)\label{Vorl.11.5.Stern1},\end{equation}
        falls $\ggt(n_1,n_2)=1$. $(\alpha *
        \beta)(1)=\sum_{d|n} \alpha(d) \beta(\frac 1 d )=\alpha(1)
        \beta(1)\stackrel{\alpha,\beta \text{ mult.}}{=}1\cdot 1$
        $\implies$ (\ref{Vorl.11.5.Stern1}) ist ok, wenn $n_1=1$ oder
        $n_2=1$. Sei nun $n_1 \neq 1,\ n_2\neq 1$.\\
        \textbf{Behauptung}: $n=n_1 n_2:$ Jeder Teiler $d|n$ ist
        eindeutig in der Form $d=d_1,d_2$ mit $d_1|n_1$ und
        $d_2|n_2$ darstellbar.\\
        Folgende Funktion $f$ ist bijektiv:
        $$f:\left\{\begin{array}{rcl}
            \{(d_1,d_2) \big| d_1|n_1,\ d_2|n_2\} &\to&
            \{d\big|d|n\}\\
            (d_1,d_2) & \mapsto & d_1 d_2
        \end{array}\right.$$
        Die Behauptung ist klar, wenn man die Primzahlzerlegung
        anschaut ($n_1,\ n_2 \neq 1$):\\
        $n_1=\prod_{i=1}^t p_i^{v_i}$,\ $n_2=\prod_{i=1}^l q_i^{w_i}$,
        die $p_i$ sowie die $q_i$ sind jeweils paarweise
        verschiedene Primzahlen. $\ggt(n_1,n_2)=1 \iff
        \{p_1,p_2,\dotsc,p_t\} \cap
        \{q_1,q_2,\dotsc,q_l\}=\emptyset$.\\
        $d|n, d=\underbrace{\prod_{i=1}^t p_i^{u_i}}_{=d_1} \cdot \underbrace{\prod_{i=1}^l
        q_i^{y_i}}_{=d_2}$ mit $u_j \leq v_j,\ y_k \leq w_k$.\\ Es
        gilt weiterhin $\ggt(d_1,d_2)=1=\ggt\left(\frac {n_1}{d_1},\frac{n_2}{d_2}
        \right)$.
        \begin{eqnarray*}
            (\alpha * \beta)(\underbrace{n}_{=n_1 n_2}) &=&
            \sum_{d|n} \alpha(d) \beta(\frac n d)\\
            &=&\sum_{d_1|n_1,\ d_2|n_2} \alpha(d_1 d_2) \beta\left(\frac
            {n_1}{d_1} \frac{n_2}{d_2}\right)\\
            &\stackrel{\alpha,\ \beta \text{
            mult.}}{=}&\sum_{d_1|n_1,\ d_2|n_2} \alpha(d_1) \alpha(d_2)
            \beta\left(\frac{n_1}{d_1}\right)
            \beta\left(\frac{n_2}{d_2}\right)\\
            &=&\sum_{d_1|n_1,\ d_2|n_2} \left(\alpha(d_1)
            \beta\left(\frac{n_1}{d_1}\right)\right) \cdot
            \left(\alpha(d_2)\beta\left(\frac{n_2}{d_2}\right)\right)\\
            &\stackrel{\text{distributiv}}{=}&\sum_{d_1|n_1} \alpha(d_1)
            \beta\left(\frac{n_1}{d_1}\right) \cdot \sum_{d_2|n_2}
            \alpha(d_2)\beta\left(\frac{n_2}{d_2}\right)\\
            &=& (\alpha * \beta)(n_1) \cdot (\alpha * \beta)(n_2).
        \end{eqnarray*}
        Zeige nun noch: $\alpha$ multiplikativ $\implies$
        $\beta=\alpha^{-1}$ ist multiplikativ. In der Vorlesung wird
        nur die Idee gezeigt, der Rest bleibt als Übung. Sei also
        $\gamma$ die multiplikative Funktion mit $\gamma(1)=1$ und
        $\gamma(p^k)=\beta(p^k),\ (p\in P, k \in \MdN_+$ (nach (3)))
        Mit Hilfe der Multiplikativität von $\gamma$ leicht
        nachzuweisen: $\alpha * \gamma = \delta \implies
        \gamma=\alpha^{-1}=\beta \implies \beta$ ist multiplikativ.
    \end{enumerate}
\end{beweis}

\begin{beispiel}
Anwendungsbeispiele für diesen Satz: $\Pi_k$ ist multiplikativ, $c_1
= \Pi_0$ auch. Daraus folgt, dass $\Pi_k * c_1$ auch multiplikativ
ist. Wegen $(\Pi_k * c_1)(n) = \sum_{d|n} \Pi_k(d) c_1(\frac n d) =
\sum_{d|n} d^k = \sigma_k(n)$ ist also auch $\sigma_k$, insbesondere
$\sigma$ und $\tau$, multiplikativ.

Zum Beispiel: $\sigma_k(p^t) = \sum_{d|p^t} d^k = \sum_{j=0}^t
(p^j)^k = \frac{p^{k(t+1)}}{p^k-1}$. \\
Das liefert die Formel
$\sigma_k(n) = \prod_{p\in\MdP, p|n} \frac{p^{k(v_p(n)+1)} - 1}{p^k-1}$ \\
sowie $\tau(p^t) = t+1 \implies \tau(n) = \prod_{p|n}(v_p(n) + 1)$ und
\begin{equation}\label{eq:Teilersumme}
    \sigma(n) = \prod_{p|n} \frac {p^{v_p(n) +1} -1 }{p-1}.
\end{equation}

Eine konkrete Berechnung ist $\sigma(100) = \frac{2^3 - 1}{2-1}
\cdot \frac{5^3 -1}{5-1} = 7\cdot 31$.
\end{beispiel}


\subsection*{Historischer Exkurs}
$\sigma(n)=\sum_{d|n} d$ (Teilersumme), $\sigma^*(n)=\sum_{d|n,\
d\neq n} d=\sigma(n)-n$.\\
\textbf{Benennung (Griechen)}: $n\in \MdN_+$ heißt
$\left\{\begin{matrix}\text{defizient}\\\text{abundand}\\\text{vollkommen}
\end{matrix}\right\}\iff \sigma^* (n) \left\{\begin{matrix}
<\\>\\=\end{matrix}\right\}n$.\\
Beispielsweise ist jede Primzahl defizient, 12 abundant und 6 ist die kleinste vollkommene Zahl.

\begin{satz}[Euklid, Euler]
    Die geraden vollkommenen Zahlen sind genau die der Form
    $$
    n=2^{p-1} M_p\quad p\in \MdP,\ M_p=2^p-1 \in \MdP \text{
    Mersenne-Primzahl}.
    $$
\end{satz}
Unbekannt: Gibt es unendlich viele Mersenne-Primzahlen? Gibt es
unendlich viele vollkommene Zahlen? Gibt es wenigstens \emph{eine}
ungerade vollkommene Zahl (Es gibt mindestens 100 Arbeiten zu den
Eigenschaften der ungeraden vollkommenen Zahlen, aber leider hat
noch niemand eine gefunden)?
\begin{beweis}\glqq $\Leftarrow$\grqq\space Sei $n=2^{p-1}M_p$ wie oben.
    \begin{eqnarray*}
        \sigma(n)&=&\sigma(2^{p-1}) \cdot
        \sigma(M_p)=\left(\underbrace{\frac{2^{p-1+1}-1}{2-1}}_{\text{vgl. }(\ref{eq:Teilersumme})}\right) \cdot \underbrace{(1+M_p)}_{M_p\text{ ist prim}}\\
        &=& (2^p-1)2^p=2\cdot 2^{p-1} \cdot M_p = 2n \implies \sigma^*(n)=n \implies n \text{
        vollkommen.}
    \end{eqnarray*}
    \glqq $\Rightarrow$
    %NICHT prüfungsrelevant, aber Hr. Rehm hält den Beweis für ein Juwel.
    \grqq\space $n$ sei vollkommen und $2|n$, also
    $\sigma(n)=2n$. $n=2^r \cdot x, x \in \MdN_+,\ 2 \not|\ x \implies
    \ggt(2^r,x)=1$.
    \begin{equation}\label{Vorl.11.5.Stern2}
        \sigma(n)\stackrel{\text{mult.}}{=}\sigma(2^r)\sigma(x)=\frac{2^{r+1}-1}{2-1}\sigma(x)
        \stackrel{n \ vollkommen}{=}2n=2^{r+1}x
    \end{equation}
    $\ggt(2^{r+1},2^{r+1}-1)=1 \implies 2^{r+1}|\sigma(x)$, also
    $\sigma(x)=2^{r+1}y$ mit $y\in \MdN_+$\\$
    \stackrel{(\ref{Vorl.11.5.Stern2})}{\implies} x=\underbrace{(2^{r+1}-1)}_{=:b}y
    =by$. $T(x) \subseteq \{1,y,b,by\}$ mit $b>1$ wegen $r>0$.
    $\sigma(x)=(b+1)y=y+by,\ y<by$ wegen $b>1$.\\
    $\implies T(x)=\{y,by\}\implies y=1,\ x=b,\ T(x)=\{1,b\}=\{1,x\}
    \implies x=2^{r+1}-1$ ist prim.\\
    Mit Aufgabe 3a, Übungsblatt 1 $\implies r+1=p \in \MdP,\ x=M_p\implies$ Behauptung.
\end{beweis}
% Änderung 31.05.2006 | Definition von befreundet | Robert Geisberger
\begin{satz}[ohne Beweis, nach Abdul Hassan Thâ bit Ibn Kurah, ca. 900]
Sind $u=3\cdot 2^{n-1}-1,\ v=3\cdot 2^n -1,\ w=9\cdot 2^{n-1}$ alle
prim, so sind $2^n uv$ und $2^n w$ befreundet. Zwei Zahlen $n,m$ aus
$\MdN_+$ heißen befreundet, genau wenn $\sigma(n)=\sigma(m)$ gilt (zum Beispiel 220 und 284).
% Prüf das mal bitte jemand: In meinem Mitschrieb steht, zwei Zahlen
% seien befreundet, wenn \sigma(n)=\sigma(m), aber das macht irgendwie
% keinen Sinn.
% Nachgefragt in der Übung
\end{satz}
Zur Eulerschen Funktion $\varphi$:
$\relp(n,d):=\{x\in\MdN_+\big|x\leq
n,\ \ggt(n,x)=d\}$.\\
$\varphi(n)=\#\relp(n,1).$
\begin{lemma}[Gauß]
    $$n=\sum_{d|n}\varphi(d)$$
\end{lemma}
\begin{beweis}
    Die Abbildung $f:\left\{\begin{array}{rcl}\relp(\frac nd,1)
    &\to& \relp(n,d)\\ x &\mapsto& dx\end{array} \right.$ ist
    bijektiv.\\
    $\ggt(\frac nd,x)=1,\ d=d\cdot 1=\ggt(d \frac n d, d\cdot
    1)=\ggt(n,d),\ x\leq \frac nd \iff dx\leq n$. $\bigcup_{d|n}
    \relp(n,d)=\{1,2,\dotsc,n\}$ (wenn $\ggt(y,n)=d$, so $y \in
    \relp(n,d),\ y\leq n$.\\
    $n=\#\{1,2,\dotsc,n\}=\sum_{d|n}\# \relp(n,d)\stackrel{\text{wg. obiger Bijektion}}{=}\sum_{d|n}
    \#\relp(\frac nd,1)=\sum_{d|n}\varphi\left( \frac{n}{d}
    \right)=\sum_{d'|n}\varphi(d'),\quad \left(d'=\frac
    {n}{d'}\right).$
\end{beweis}
Lemma von Gauß sagt: $\Pi_1=\varphi * c_1$, $\Pi_1(n)=n^1=n$. Da
$\Pi_1$ und $c_1$ multiplikativ sind $\implies \varphi=\Pi_1 *
c_1^{-1}$ ebenfalls multiplikativ (aus Multiplikativitätssatz)
$\implies$ $\varphi(n)=n \Pi_{p_n}(1-\frac 1 p)$ (früher).
\begin{definition}
    Ist $\alpha \in \text{Arfun}$, dann heißt $\hat{\alpha}$
    Möbiustransformierte von (oder Summatorische Funktion zu)
    $\alpha$, wenn:
    $$
        \hat{\alpha}(n):=\sum_{d|n}\alpha(d)
    $$
    (Das heißt: $\hat{\alpha}=\alpha * c_1$.)
\end{definition}
Problem: Wie kann man $\alpha$ aus $\hat{\alpha}$ gewinnen (bzw. berechnen)?\\
Lösung: $\hat{\alpha}=\alpha * c_1 \implies \alpha=\hat{\alpha} * \mu$, mit $\mu=c_1^{-1}$.\\
$\mu=c_1^{-1}$ heißt Möbiusfunktion.\\
Rest: Bestimmung von $\mu$, da $\mu$ multiplikativ ist, reicht es aus,\\
$\mu(p^l)=c_p,\ p\in P, l \in \MdN_+$ zu ermitteln.\\
$\mu(1)=1$\\
$0=\delta(p^l)=\mu*c_1(p^l)=\sum_{d|p^l}\mu(d)=\sum_{j=0}^l\mu(p^j)$\\
$l=1:\quad 0=\mu(1)+\mu(p)\implies \mu(p)=-1$\\
$l=2:\quad 0=\mu(1)+\mu(p)+\mu(p^2) \implies \mu(p^2)=0$\\
$\dotsc$\\
$\mu(p^i)=0$ für $j\geq 2$. Also folgt, weil $\mu$ multiplikativ
ist:
$$ \mu(n)=\begin{cases}0 & \exists p\in\MdP:\ p^2|n, \text{ d.h. $n$
ist nicht quadratfrei}\\(-1)^t & \text{falls $n=p_1\cdot p_2\cdot
\dotsb \cdot p_t$ mit $t$ verschiedenen Primzahlen}
\end{cases}$$
Ergebnis:
\begin{satz}[Umkehrsatz von Möbius]
    Sei $\alpha$ arithmetische Funktion, $\hat{\alpha}$ die
    Möbiustransformierte von $\alpha$, dann gilt $\alpha=\hat{\alpha}*\mu$ mit
    der Möbiusfunktion $\mu$, das heißt:
    $$\alpha(n)=\sum_{d|n}\hat\alpha(d)\mu\left(\frac{n}{d}\right)\quad
    \text{Möbiussche Umkehrformel}$$ und $\mu$ wie oben.
\end{satz}


Lineraturhinweise zu den Arithmetischen Funktionen:
\begin{enumerate}
\item Für Algebra-Freunde: "`Der Ring Arfun ist selbst faktoriell"', siehe Cashwell, Everett: The Ring of Numbertheoretic Functions, Pacific Math.J., 1955, S. 975ff.
\item Umkehrformeln gibt es für allgemeinere geordnete Mengen als ($R_\text{nor}, |$), siehe Johnson, Algebra I.
\item Für Analysis-Freunde: Viel Analysis über zahlentheoretische Funktionen. Viele Sätze über asymptotisches Verhalten (ähnlich $p_n \sim n\cdot\log n$), siehe Schwarz, Spieker, "`Arithmetical functions"', Cambridge University Press, 1994.
\end{enumerate}

\chapter{Kongruenzen und Restklassenringe}

In diesem Kapitel betrachten wir entweder $R=\MdZ$ oder $R=K[X]$,
wobei $K$ ein Körper ist.

\section*{Grundbegriffe}

In den betrachteten Ringen gibt es eine eindeutige Restwahl: In
$R=\MdZ$ ist die Division mit Rest $a=qm+r$ mit $0\le r < |m|$.
Andere Restwahl wäre etwa $a=qm+r'$ mit $-\frac {|m|}2 < r' \le
\frac{|m|}2$. Es besteht folgender Zusammenhang:
$$r' = \begin{cases} r, & 0\le r\le \frac{|m|}{2} \\ r-|m|,& \frac{|m|}2 <
r \le |m|\end{cases}$$ %

In $R=K[X]$ haben wir $a=qm+r$ mit $\grad r < \grad m$.

Diese Reste sind eindeutig: Haben wir $a=qm+r=\tilde q m + \tilde r$
mit $0\le r,\tilde r, |m|$. Dann ist $(q-\tilde q)m=\tilde r -r
\implies |m|\big|\tilde r - r$. Annahme: $q-\tilde q \ne 0 \implies
|\tilde r - r| \ge m$, Wid. Also ist $q=\tilde q$ und $r = \tilde
r$. Der Beweis für $R=K[X]$ funktioniert ähnlich.

\begin{definition}[Gauß für $R=\MdZ$]
$m,a,b, \in R$
\begin{enumerate}
\item \[ a \equiv b \mod m \text{ (lies $a$ kongruent $b$ modulo $m$} \]
\[ \equizu a \mod m = b\mod m \]
Gauß schreibt "`Zwei Zahlen heißen kongruent mod $m$, wenn sie bei
Division durch $m$ den selben Rest lassen."'
\item $\overline a := \{ b \in R| b \equiv a \mod m \}$ heißt Restklasse modulo $m$.
\item $\overline R := R/mR := \{\overline a | a \in R\}$ heißt Restklassenring modulo $m$.
\end{enumerate}
\end{definition}

Warum ist Letzeres ein "`Ring"'? Der Dozent führt einen schönen
Beweis durch Aufwickeln einer Schnur auf einer Tesa-Rolle durch.

\begin{beispiel}
$\MdZ/2\MdZ = \{ \overline 0, \overline 1\}$ mit $\overline 0 = \{0,
\pm 2, \pm 4, \ldots \}$ (die geraden Zahlen) und $\overline 1 =
\{\pm 1, \pm 3, \ldots \}$ (die ungeraden Zahlen). Aus der Schule
sind folgende Regeln bekannt:
\begin{enumerate}
\item $\overline 0 + \overline 0 = \overline 0$, "`gerade + gerade = gerade"'
\item $\overline 0 + \overline 1 = \overline 1$, "`gerade + ungerade = ungerade"'
\item $\overline 1 + \overline 1 = \overline 0$, "`ungerade + ungerade = gerade"'
\end{enumerate}
\end{beispiel}

\begin{bemerkung}
\[ \text{(i) } a\equiv b \mod m \iff \text{ (ii) } \overline a =  \overline b \iff \text{ (iii) }m|a-b\]
Merke: Kongruenz ist Gleichheit der Restklassen.

$\overline{qm} = \overline 0$. Die Idee: In $\overline R$ wird alles
durch $m$ teilbare als "`unwesentlich"' angesehen  und durch $0$
ersetzt.
\end{bemerkung}

\begin{beweis}
$(i) \iff (ii)$: Kongruenz mod $m$ ist Offensichtlich eine
Äquivalenzrelation auf $R$. $\overline a$ ist die Äquivalenzklasse
von $a$. Lineare Algebra: Zwei Elemente sind genau dann äquivalent,
wenn die zugehörigen Äquivalenzklassen überstimmen.

$(i) \implies (iii)$: $r = a \mod m = b\mod m \implies a = qm + r,
b= q'm + r$ (Division mit Rest) $\implies a - b = (q-q')m \implies m
| a-b$
\end{beweis}

Um mit Restklassen zu rechnen, brauchen wir folgende Definitionen:
\begin{definition}
Jedes $b\in \overline a$ heißt Vertreter der Klasse $\overline a \in
\overline R$. Die Idee ist, die Operationen $+$ und $-$
vertreterweise zu definieren. Wir haben also:
\[ (\overline R, +, \cdot)\text{ mit } \overline a + \overline b  := \overline {a+b},\ \overline a \cdot \overline b = \overline{a\cdot b} \]
\end{definition}

Zu zeigen: Die Definition ist vertreterunabhängig, also : $\overline
a = \overline {a'} \implies \overline {a+b} = \overline {a'+b}
\text{ und } \overline {a\cdot b} = \overline {a'\cdot b}$. Das ist
klar:
\begin{align*}
\overline a = \overline a' \iff &m|a-a' = a+b - (a'+b) \implies \overline {a+b} = \overline {a'+b} \\
&m|a-a' \implies m|(a-a')b = ab - a'b \implies \overline {ab} =
\overline {a'b}
\end{align*}

\begin{bemerkung}
$e\in R^\times, m \in R \implies R/mR = R/emR$ (da $m|x \iff em|x$).
Ohne Beschränkung der Allgemeinheit kann man $m$ also normiert
annehmen.

$m=0$, dann $a\mod m = b\mod m\iff a=b$, also $\overline a = \{a\}
\text{\glqq}=\text{\grqq} a$. Also: $R/oR = R$ und $R/eR = R/R =
\{\overline 0\}$ ("`Nullring"')
\end{bemerkung}

Diese uninteressanten Fälle werden meist beiseite gelassen.

\begin{satz}[Restklassenring-Satz]
Sei $R$ ein euklidischer Ring, $m\in R$.
\begin{enumerate}
\item ($\overline R = R/mR, +, \cdot)$ ist ein Ring
\item $\overline R^\times = \{\overline a \in \overline R| \ggt(a,m) = 1\}$

Zusatz: Zu $\overline a \in \overline R^\times$. Kann ${\overline
a}^{\,-1}$ effektiv mit Euklids Algorithmus berechnet werden.
\end{enumerate}
\end{satz}

\begin{definition}
$\overline a \in \overline R^\times$ heißt eine prime Restklasse
modulo $m$, $\overline R^\times$ heißt prime Restklassengruppe
modulo $m$. (Sprachlich besser wäre eigentlich: Gruppe der zu m
relativ primen Restklassen)
\end{definition}

\begin{beweis}
\begin{enumerate}
\item Alle Ringaxiome vererben sich von den Vertretern auf die Klassen. $\overline a + \overline b = \overline {a+b} = \overline {b+a} = \overline b + \overline a \implies$ $(\overline R,+)$ ist kommutativ. $0 := 0_{\overline R} = \overline 0$, da $\overline a + \overline 0 = \overline (a+0) = \overline a$. $1_{\overline R} = \overline 1$ ebenso.

Assoziativität der Addition: $(\overline a+ \overline b) + \overline
c = \overline {a+b} + \overline c = \overline {(a+b) +c } =
\overline {a + (b+c)} = \overline a + \overline {b+c} = \overline a
+ (\overline b + \overline c)$, Assoziativität der Multiplikation
und Distributivgesetzt analog.
\item $\overline a \in \overline R^\times \equizunach{Def.} \exists x \in R: \overline x \overline a = 1_{\overline R} = \overline 1 \iff 1 \equiv ax \mod m \iff \exists q \in R: 1 =ax + qm \implies \ggt(a,m)=1$, (da normal).

Der LinKom-Satz \ref{satz:LinKom} liefert: $d=\ggt(a,m) \implies
\exists x,y\in R: d=ax+by$. Diesen Satz dürfen wir anwenden, da $R$
euklidisch ist. Wir wenden ihn mit $d=1, q=y$ an und erhalten
$1=ax+qm$, wobei $x$ durch Euklids Algorithmus geliefert wird.
$\implies \overline 1 = \overline a \overline x + \overline q
\overline m = \overline a \overline x$. Resultat: $\overline a
^{\,-1} = \overline x$ mit dem so berechnetem $x$.
\end{enumerate}
\end{beweis}

\begin{folgerung}
Ist $m\in \MdNp$, dann gilt für Eulers Funktion $\varphi$:
\[ \varphi(m) = \#\{R/mR\}^\times \]
\end{folgerung}

Der Grund ist dass $R/mR = \{\overline 0, \ldots, \overline {m-1}\}$
und $(R/mR)^\times = \{\overline r | 0 \le r < m, \ggt(r,m)=1\}$,
derer es $\varphi(m)$ gibt.

Im Allgemeinen ist $\overline R$ nicht integer. Beispielsweise in
$\MdZ/4\MdZ=\overline R$ gilt: $\overline 2 \cdot \overline 2 =
\overline 4 = 0_{\overline R} = 0$, aber $\overline 2 \ne 0$

\begin{folgerung}
Falls $m$ unzerlegbar (also $m$ Primzahl oder -polynom). Dann gilt:
$R/mR$ ist ein Körper.
\end{folgerung}

Speziell:
\begin{enumerate}
\item  $\MdF_p := \MdZ/p\MdZ$, $p\in\MdP$ ist Körper mit $p$ Elementen.
\item Ist $f\in K[X]$, f irreduzibel, so ist $K[X]/f\cdot K[X] = \overline R$ ein Körper.
\end{enumerate}

Grund: $m$ sei unzerlegbar. Dann $\overline a \in \overline R,$
$\overline a\ne 0 = \overline 0 \iff m\not|a \implies \ggt(m,a) = 1$
($1,m$ sind die einzigen normierten Teiler von $m$!) $\implies a \in
\overline R^\times$. Es gilt also $\overline R^\times = \overline
R\setminus\{0\} \implies \overline R$ ist Körper.


$\overline R = R/mR \ni \overline a = a + Rm := \{a + qm \big| q \in R\}$ Restklasse von a.\\
Rechne in $\overline R$:
\textbf{Idee}: Kodiere die Restklasse $\overline a$ durch den Vertreter $a \mod m$.\\

Beliebige Vertretersysteme (ohne Einschränkung $m \in \MdN_+, m >
1)$
\begin{itemize}
    \item[] \underline{R = $\MdZ$}: \\$\text{Versys}_m = \{0,1,...,m-1\}$ "`\emph{System Betrag kleinster positiven Reste}"' oder $\text{Versys}_m =\{v \in \MdZ \big| - \frac{m}{2} < r \le \frac{m}{2}\}$ "`\emph{Symmetrisches Restsystem}"'
    \item[] \underline{R = $K[X]$}: \\$\text{Versys}_m = \{f \in K[X] \big| \text{Grad } f < \text{ Grad } m\}$ (Grad $m > 0$)
\end{itemize}

\textbf{Klar}:
\[\begin{array}{rcl}
        \text{Versys}_m & \longrightarrow & R/mR \text{ (Ist bijektiv)}\\
        r & \longmapsto & \overline r\\
        a & \longmapsto & \overline a \mod m \text{ (Umkehrung)}
    \end{array}
\]

Transportiere die Struktur $(\text{Versys}_m, \oplus, \odot)$, wobei
gilt:
\[
    r \oplus s := r+s \mod m \qquad
    r \odot s := rs \mod m
\]
Klar, $r \mapsto \overline r$ ist ein Ringisomorphismus.

Vorzug bei $R = \MdZ$:
\begin{itemize}
    \item[] $r+s \mod m$ mit $1$-Addition: Zahlen $< 2m$
    \item[] $r\cdot s$: Zahlen $<m^2$
    \item[] ($m \mod \frac{m^2}{4}$ bei symmetrischen Resten)
\end{itemize}

Vorzug bei $R = K[X]$:
\begin{itemize}
    \item[] Ist $n = \text{Grad }f$, so ist $\text{Versys}_m$ ein $K$-Vektorraum der Dimension $n$ (Basis z.B.: $1, X, X^2, ..., X^{n-1}$)
    \item[] $\text{Grad }f < m$, $\text{Grad }g < m \implies \text{ Grad } (f+g) < m \implies f \oplus g = f + g \implies \oplus = +$
    \item[] $\text{Versys}_m$ enthält $K$ als Teilkörper (konstante Polynome), da:\\
    $\alpha, \beta \in K \subset K[X] \implies \alpha \odot \beta = \alpha \beta \mod m = \alpha\beta$
\end{itemize}

\begin{folgerung}
    $\overline R = K[X]/mK[X]$ ist ein $K$-Vektorraum der $\text{dim }n = \text{ Grad }m$ mit Basis $1, \overline X, \overline X^2, ..., \overline X^{n-1}$. Identifiziert man $\alpha \in K$ mit der Restklasse $\overline \alpha$, so enthält $\overline R$ den Körper $R$.
\end{folgerung}

\begin{folgerung}
    Ist $m \in \MdF_p[X] = R$ irreduzibel, so ist $R/mR = \overline R$ ein Körper mit $q = p^n$ ($n = \text{ Grad }m$) Elementen!\\
    \textbf{Grund:} $\MdF_p$-Basis ist $1, \overline X, \overline X^2, ... , \overline X^{n-1}$.\\
    $\overline R = \{\alpha_0 \cdot 1 + \alpha_1 \overline X +... + \alpha_{n-1} \overline X^{n-1} \big| \alpha_0,...,\alpha_{n-1} \in \MdF_p\}$ mit $\# \overline R = p^n$
\end{folgerung}

Zum Rechnen in $\overline R$ wird empfohlen $\overline \alpha \in \MdF_p$ durch $r = a \mod p$ zu ersetzen, mit $r \in \text{Versys}_p$. $f \in \text{Versys}_p[X]$ hat die Form $f = \sum_{i=0}^n c_iX^i$, $c_i \in \text{Versys}_p$. \\
Bei der Bestimmung von $f+g, f\cdot g$ ist bei allen Rechnungen mit
Koeffizienten $c_1,...,c_n$, $+$ durch $\oplus$ und $\cdot$ durch
$\odot$ zu ersetzen. Man kann auch $f+g, f\cdot g$ in $\MdZ[X]$
berechnen und dann zu allen Koeffizienten die Reste $\mod p$ nehmen.

\begin{beispiel}
    $\MdF_3[X]$, $\MdF_3 =\{\overline 0, \overline 1, \overline 2\}, \text{ Versys}_3 = \{0,1,2\}$\\
    $\begin{array}{rcl}
    \underbrace{(X^2 + 2X + 1)}_{\mathclap{(\text{= } \overline 1 \cdot X^2 + \overline 2 \cdot X + \overline 1 \text{ in } \overline R[X])}} \cdot (2X + 1) & = & 2X^3 + \underbrace{2 \odot 2}_{\mathclap{=1 \text{ in } \MdZ[X]}}X^2 + 2X + X^2 + 2X + 1\\
    &=&2X^3 + 4X^2 + 2X + X^2 + 2X + 1\\
    &=&2X^3 + \underbrace{5}_{\mathclap{2 \mod 3}}X^2 + \underbrace{4}_{\mathclap{1 \mod 3}}X + 1\\
    &=&2X^3 + (1 \oplus 1)X^2 + (2 \oplus 2)X + 1\\
    &=&2X^3 + 2X^2 + X + 1
    \end{array}$
\end{beispiel}

\begin{beispiel}
    $\MdF_4 = \{\underbrace{0, 1}_{\MdF_2}, \underbrace{\overline x}_{=: \varrho}, \overline x + 1\}$, wenn $m$ irreduzibel in $\MdF_2[X]$, $\text{Grad }f = 2$\\ $X^2 + 1 = (X + 1)^2$ $(= X^2 + \underbrace{\overline 2}_{=0}X + 1 = X^2 + 1 \text{ in } \MdF_2[X])$\\
    $X^2 + X + 1$ ist irreduzibel. (Alle Polynome vom $\text{Grad }1$ sind $X, X+1, X^2, X(X+1),(X+1)^2 = X^2 + 1$ sind von $m$ verschieden $\implies$ irreduzibel)\\
    $\MdF_4 = \{0,1,\varrho, \varrho + 1\}$, $\varrho^2 = ?$\\
    $(\overline X)^2 = \underbrace{\overline{X^2 \mod m}}_{\in \text{ Versys}_m} = \overline{X + 1} = \overline X + 1 = \varrho + 1$\\
    $X^2 -1\cdot (X^2 + X + 1) = -X - 1 = X + 1$ in $\MdF_2[X]$\\
    Rechenregel: $\varrho^2 = \varrho +1 \implies $ Multiplikationstafel
\end{beispiel}

\begin{bemerkung}
    \text{ }
    \begin{itemize}
        \item $R \to \overline R = R/mR$, $\kappa: a \mapsto \overline a = \kappa(a)$, so ist $\kappa$ surjektiver Ringhomomorphismus. $\kappa(a+b) = \overline a + \overline b = \overline{a+b} = \kappa(a+b)$
        \item Ist $R$ ein Ring und $z \in \MdZ$, so definiert man:
            \[z \cdot \varrho := sgn(z)\underbrace{(\varrho + \varrho + ... + \varrho)}_{|z|-\text{Stück}}\]
    \end{itemize}
\end{bemerkung}

\begin{beispiel}
    $\overline R = \MdZ/m\MdZ, z \in \MdZ$\\
    $z\overline a = \overline{za}$ (leicht selbst nachzuweisen)\
    $m \cdot 1_{\overline R} = m \cdot \overline 1 = \overline m = 0_{\overline R}$
\end{beispiel}

\textbf{Rechenregeln:} $z, z_1, z_2 \in \MdZ, \varrho, \varrho_1,
\varrho_2 \in R$
\begin{itemize}
    \item[] $(z_1 + z_2)\varrho = z_1\varrho + z_2\varrho$
    \item[] $z(\varrho_1 + \varrho_2) = z\varrho_1 + z\varrho_2$
    \item[] $(z_1z_2)\varrho = z_1(z_2\varrho)$
    \item[] $z(\varrho_1\varrho_2) = (z\varrho_1)\varrho_2 = \varrho_1(z\varrho_2)$ (Beweis leicht)
\end{itemize}

Für $f \in \MdZ[X], \overline a \in \MdZ/\MdZ m$ ist definiert ($f =
\sum_{i=0}^n z_iX^i$):
\[f(\overline a) = \sum_{i=0}^nz_i\overline a^i \in \overline R \text{ } (= \sum_{i=0}^n\overline{z_ia^i} = \overline{f(a)}\]
Ergebnis: $f(\overline a) = \overline{f(a)}$

\section{Zyklische Gruppen}

\textbf{Aufgabe:} Berechne $3^{10^{500}} \mod
\underbrace{167}_{=:p}$ (Rechne in $\text{Versys}_{167}$ !)

\textbf{Mathematische Hilfsmittel:} Ordnung eines Gruppenelements.

\begin{definition}
    Sei $G$ eine (ohne Einschränkung multiplikative) endliche Gruppe, $x \in G$. (Das neutrale Element werde mit $1 = 1_G$ bezeichnet)
    \begin{itemize}
        \item[(i)] $\ord(x) = \min \{n \in \MdN_+ \big| x^n = 1\}$ heißt "`\emph{Ordnung von x}"'
        \item[(ii)] $\#G$ heißt "`\emph{Ordnung von G}"'
    \end{itemize}
\end{definition}

\begin{bemerkung}
    $\ord(x)$ existiert, da $n > m, n,m \in \MdN_+$ vorhanden sind mit $x^n = x^m$, da $G$ endlich. $\implies x^{n-m} = 1$. In allgemeinen Gruppen kann sein $\{n \in \MdN_+ \big| x^n = 1\} = \emptyset$, dann schreibt man $\ord(x) = \infty$
\end{bemerkung}

\begin{satz}[Elementordnungssatz]
    Sei $G$ eine endliche Gruppe, $x \in G$, $m, n \in \MdZ$. Dann gelten:
    \begin{itemize}
        \item[(i)] $x^m = x^n \iff m \equiv n \mod \ord(x)$\\
        Insbesondere $x^m = x^{m \mod \ord(x)}$ und $\mod x^m = 1 \iff \ord(x) \big| m$
        \item[(ii)] $x^{\# G} = 1$ (d.h. nach (i) $\ord(x) \big| \# G$)
        \item[(iii)] $\ord(x^m) = \frac{\ord(x)}{\ggt (m, \ord(x))}$
    \end{itemize}
\end{satz}

\textbf{Anwendung}:
\begin{itemize}
    \item[] \underline{Satz von Euler:} Sei $m, x \in \MdZ, m > 0, \ggt (x,m) = 1, \varphi$ sei die Eulersche Funktion. Dann gilt: $x^{\varphi(m)} \equiv 1 \mod m$
    \item[] \underline{(Kleine) Satz von Fermat:} Sei $p \in \MdP, x \in \MdZ$. Dann gilt: $x^p \equiv x \mod p$
\end{itemize}

\textbf{Zum Satz von Euler:}\\
$G = (R/Rm)^\times, \# G = \varphi(m)$. $\overline x \in G \iff
\ggt(x,m) = 1$. Elementordnungssatz (ii) $\implies \overline 1 = 1_g
= \overline x^{\# G} = \overline x^{\varphi(m)} = x^{\varphi(m)}
\iff 1 \equiv x^{\varphi(m)} \mod m$

\textbf{Zum Satz von Fermat:}\\
$\varphi(p) = p-1$. Aussage klar, wenn $p \big| x (x \equiv 0 \equiv
xp)$. $p \nmid x \implies \ggt(p,x) = 1 \implies \overline x^{p-1} =
\overline x^{\# G} = \overline 1 \implies \overline x^p = \overline
x \implies x^p \equiv x \mod p$

\begin{beweis}[Elementordnungssatz]
Sei $x\in G$, $\ord(X) =: l$.
\begin{enumerate}
\item $x^m = x^n \iff x^{m-n}= 1 = 1_G \iff 1= x^{ql+r} = (w^l)^q\cdot x^r = 1^q\cdot x^r = 1x^r = x^r$ (Falls $r\ne0$, so haben wir einen Widerspruch zur Minimalwahl von $l$) $\iff r=0 \iff l\mid m-n \iff m \equiv n \mod l$.

Insbesondere: $x^m=1 \iff l \mid m$, $x^n = x^{n\mod l}$
\item $x^{\#G} = 1$. Dies wird in dieser Vorlesung nur für kommutative $G$ benötigt und bewiesen. Betrachte die Abbildung $G\to G$, $x \mapsto y\cdot x$. Sie ist bijektiv (die Umkehrabbildung ist $y\mapsto y x^{-1}$), also $\{y \mid y\in G\} = G = \{yx \mid y\in G\}$.
\[ \prod_{y\in G}y  =\prod_{y,x\in G} (yx) = \prod_{y\in G} y \cdot x^{\#G} \implies x^{\#G} = 1 \]

Also laut (1): $\ord(x) \mid \#G$

\item $\ord(x^m)=k \implies 1= (x^m)^k = x^{mk} \folgtnach{(1)} l \mid mk$. Sei $d=\ggt(m,l) \implies \frac{l}{d} \mid \frac{m d} \cdot k \folgt \frac l d \mid k$. Warum sind $\frac l d$ und $\frac md$ relativ prim? $d = \ggt(m,l)= d\cdot \ggt(\frac m d, \frac l d) \folgt \ggt(\frac m d, \frac l d) = 1$. Aber $k \mid \frac l d$ wegen $(x^m)^{\frac l d} = x^{l\cdot \frac md} = 1^{\frac m d} = 1$, $k= \ord(x^m)$ nach (1).

Ergebnis: $k = \frac l d = \frac {\ord(x)} {\ggt(\ord(x),m)}$

\end{enumerate}
\end{beweis}

\subsubsection*{Hilfestellungen zur Berechnung von $\ord(x)$}

\begin{bemerkungen}
\begin{enumerate}
\item $\ord(a)\mid\#G$ (\emph{wirklich $a$?})
\item Sei $x^d=1$. Dann gilt: $d = \ord(x) \iff \forall p\in\MdP$ mit $p\mid d$: $x^{\frac dp} \ne 1$.
\end{enumerate}
\end{bemerkungen}
\begin{beweis}[Der Bemerkung (ii)]
"`$\Longrightarrow$"': Klar

"`$\Longleftarrow$"': Sei $x^d=1$, $x \ne \ord(x)$. Nach (1):
$\ord(x)\mid d \implies \exists p\in\MdP: \ord(x) \mid \frac dp
\folgt x^{\frac d p } = 1$
\end{beweis}

Zur Berechnung von $x^n$: Naive rekursive Berechnung: $x^{j+1} =
x^j\cdot x$. Hier hätten wir $n$ Produkte zu berechnen! Westentlich
bessere Methode: Stelle $n$ binär da: $n = \sum_{i=0}^t c_i \cdot
2^i$, $c_t\ne 0$, $c_i\in\{0,1\}$. Bezeichnung
$n=(c_t,c_{t-1},\ldots,c_0)_2$ mit den Binärziffern $c_j$.
\[ x^n = x^{ \sum_{i=0}^t c_i \cdot 2^i} = \prod_{i=0}^t \left(x^{2^i} \right)^{\mathrlap{c_i}} = \prod_{\mathclap{i=0,\ c_i\ne 0}}^t x^{(2^i)} \]
Rekursiv: $x^{2^0} = x^1 = x$ und $x^{2^{i+1}} = (x^{2^{i}})^2$. $t$
ist etwa $\log_2 n$, man hat ungefähr $2\cdot\log_2 n$ Produkte zu
berechnen.

\begin{beispiel}
$G=\MdF_9^\times$, $\#G = 9-1 = 8$. Mögliche $\ord(\alpha)$ für ein
$\alpha\in G$: 1,2,4, oder 8.
\begin{align*}
\ord(\alpha) = 1 &\iff \alpha = 1 \\
\ord(\alpha) = 2 &\iff \alpha \ne 1, \alpha^2 = 1 \iff \alpha = -1_G = -1 \\
\ord(\alpha) = 4 &\iff \alpha^4  =1 , \alpha^2 \ne 1 \text{ (d.h. $\alpha \ne \pm 1$)} \\
\ord(\alpha) = 8 &\iff \alpha^4 \ne 1
\end{align*}

$\MdF_9 = \MdF_3[X] / m\cdot\MdF_3[X]$, $\ord(m)=2$, $m$
irreduzibel. Beispielsweise ist $X^2+1$ in $R=\MdF_3[X]$
irreduzibel.

$\MdF_9$ hat $\MdF_3$-Basis $1;\overline x$. $\MdF_9 =
\{\underbrace{0,1,-1}_{\mathclap{\MdF_3=\text{Versys}_3}},\ldots
\}=\{a+b\overline x\mid a,b\in\MdF_3\}$

$m=X^2+1\equiv 0 \mod m \folgt X^2 \equiv -1 \mod m \folgt \overline
X^2 = -1 = -1_{\MdF_9} = -1_{\MdF_3} \folgt \overline X^4 = (-1)^2 =
1 \folgt \ord(\overline X)=4$.

$(\overline X+1)^2 = \overline X^2 + 2\overline X +1 = -1 + 1 + 2X =
-X \ne 1$, $(\overline X+1)^4 = (-\overline X)^2 = \overline X^2 =
-1 \folgt {\ord(\overline X+1)=8}$
\end{beispiel}

Zurück zum Problem $3^{(10^{500})} \mod 167$, $167\in\MdP$.
$G=\MdF_{167}$, $\#G=\varphi(167) = 166 = 2\cdot 83$, also gilt
$\ord(n) \in \{1,2,83,166\}$.

Laut Ordnungsatz: $3^{10^{500}} \equiv 3^{10^{500} \mod
\ord(\overline 3)}$.

Wir brauchen $\ord(3)$: $\overline 3 ^2 = \overline 9 \ne 1_G \folgt
\ord(\overline 3)\ne 1,2$, $\ord(\overline 3) = 83 \iff \overline 3
^{83} = 1_G = \overline 1$. $83 = (1010011)_2=64+16+2+1$. Tabelle:
$3^{2^0}$ in $\MdF_{167}$ ist 3, $3^{2^1}$ in $\MdF_{167}$ ist
$3^2=9$, $3^{2^2}$ in $\MdF_{167}$ ist $9^2=81$, $3^{2^3}$ in
$\MdF_{167}$ ist $81^2 = 6651 = 30\cdot 167+48\equiv 48$, $3^{2^4}$
in $\MdF_{167}$ ist $48^2 \equiv 133$, $3^{2^5}$ in $\MdF_{167}$ ist
$133^2 = 17629 \equiv 154$, $3^{2^6}$ in $\MdF_{167}$ ist $154^2 =
\equiv 2$. Also: $\overline 3 ^{83}  =\overline 3 \cdot \overline 9
\cdot \overline{133} \cdot \overline 2\cdot \overline{7182} \cdot
\overline{1}  =1_G$. Ergebnis: $\ord(\overline 3)=83$.

$3^{10^{500}} = 3^{10^{500} \mod 83}$. Noch zu berechnen: $10^{500}
\mod 83$. Man kann $\overline{10}$ in $\MdF_{83}$ berechnen. Reicht
auch $\overline{10}^{500} = 10^{500\mod \varphi(83)}$. $\varphi(83)
= 82$, $500\equiv 8 \mod 82 \folgt 10^{500} \equiv 10^8 \equiv 23
\mod 83$

Also: $\overline 3 ^{10 ^{500}} = \overline 3 ^{23} =
\overline{124}= \overline{-33}$ und somit $3^{100^{500}} = 124 \mod
167$

\begin{satz}[Mersenne-Teiler-Satz]
Es seien $p,q\in\MdP$ mit $q \mid M_p = 2^p-1$. Dann gilt: $q\equiv
1 \mod p$
\end{satz}

\begin{beweis}
$q\mid M_p \iff M_p = 2^p-1 \equiv 0 \mod q \iff \overline 2^p = 1$
in $\MdF_q^\times=G \folgt \ord(\overline 2) = p$, da 1 nicht geht
und $\ord(\overline 2)\mid p$ nach dem Ordnungsatz. $\ord(\overline
2) \mid \# G = \varphi(q) = q-1 \folgt q-1 \equiv 0 \mod p \folgt q
\equiv 1 \mod p$
\end{beweis}


\paragraph{Bezeichnungen:}
\begin{enumerate}
\item $\langle x \rangle  = \{1,x,x^2,\ldots,x^{l-1}\}$, ($l=\ord(x)$), heißt die von $x$ erzeugte zyklische Untergruppe von $G$.
\item $G$ heißt zyklisch $\iff \exists x \in G: G=\langle x\rangle  \iff \exists x \in G: \ord(x)=\#G$
\end{enumerate}

\begin{bemerkung}
Die Abbildung $(\MdZ/\MdZ l,+)\to (\langle x\rangle ,\cdot)$ mit
$\overline m \mapsto x^m$ ist ein Isomorphismus von Gruppen.
\end{bemerkung}
\section{Primitivwurzeln}

Vorbereitungen über $R=K[X]$, $K$ ein Körper.

\begin{bemerkung}
Sei $\alpha \in K$, $f\in R$, $\ord(f)>0$. Dann gilt:
\[ 0 = f(\alpha) \iff X-\alpha \mid f \iff v_{X-\alpha}(f) > 0 \quad ( X-\alpha \in \MdP_{R} )\]
$v_{X-\alpha}$ heißt Vielfachheit der Nullstelle $\alpha$ von $f$.
\end{bemerkung}

\begin{beweis}
Division mit Rest: $f= q\cdot (X-\alpha) + r$. $\grad r < \grad
(X-\alpha) = 1 \folgt r\in K$ (konstantes Polynom), insbesondere
$r(\alpha) = r$. $f(\alpha) = q(\alpha)(\alpha-\alpha) + r(\alpha) =
r $. Also: $r(\alpha) = 0 \iff {r = 0} \iff {X-\alpha \mid f}$
\end{beweis}

\begin{satz}[Nullstellenanzahls-Satz]
$f\in K[X]$, $f\ne 0$, $n=\grad f$, so gilt: $f$ hat höchstens $n$
verschiedene Nullstellen in $K$.
\end{satz}

\begin{beweis}
$\alpha_1,\ldots,\alpha_l$ seien $l$ Nullstellen. $v_{X-\alpha_j}(f)
> 0  \folgt \prod_{j=1}^l (X-\alpha_j) \mid f$, wegen
${v_{X-\alpha_i}(\prod_{j=1}^l(X-\alpha_j)) = 1}$ und
${v_{m}(\prod_{j=1}^l(X-\alpha_j)) = 0}$ für alle anderen $m\in
\MdP$ sowie $v_{X-\alpha_j}(f) \ge 1$. Daraus folgt: $l\le \grad f$
\end{beweis}

Der Spezialfall $K=\MdF_p$ ergibt den

\begin{satz}[Satz von Lagrange]
Sei $p\in\MdP$, $f=\sum_{i=0}^n c_iX^n \in \MdZ[X]$. Es gibt ein
$j\in\{0,\ldots,n\}$ mit $c_j\not\equiv 0 \mod p$. Dann fallen die
"`Lösungen"' $x\in\MdZ$ der Kongruenz
\[ f(x) \equiv 0\mathrlap{ \mod p} \]
in höchstens $n$ verschiedene Restklassen modulo $p$.
\end{satz}

\begin{beweis}
Der Satz ist eine Übersetzung des Nullstellenanzahls-Satzes auf
Kongruenzen. Betrachte die $\overline {c_j} = \alpha _j \in \MdF_p
\folgt \exists j:\, \overline{c_j}\ne 0\folgt f=\sum_{i=0}^n
\overline{c_j}X^j \ne 0$ in $\MdF_p[X]$, $\ord(f)\le n$. $f(x) = 0
\mod p \iff \overline{f(x)} = f(\overline x) = 0_{\MdF_p}$. Es gibt
höchstens $n$ Nullstellen $\overline x$, das heißt lösende
Kongruenzklassen.
\end{beweis}

$p\in\MdP$ wird gebraucht, Aussage modulo $m$, $m\notin \MdP$, im
Allgemeinen falsch. Beispiele: $m=6$, $f=X^2+X$ hat in $\MdZ/6\MdZ$
die Nullstellen $\overline 0$, $\overline 2$, $\overline 3$,
$\overline 5$. $m=9$, $f=X^2$ hat in $\MdZ/9\MdZ$ die Nullstellen
$\overline 0$, $\overline 3$, $\overline {-3}$.

\begin{satz}[Primitivwurzelsatz]
Sei $K$ Körper, $G$ eine \emph{endliche} Untergruppe von $K^\times$.
Dann ist $G$ zyklisch. Genauer gilt: $\#\{\alpha \in K| \ord(\alpha)
= \#G \} = \varphi(\#G)$ ($\varphi$ die Eulersche Funktion)
\end{satz}

\begin{bemerkung}
Ist $\ord(\alpha)=\#G$, so heißt $\alpha$ primitive $\#G$-te
Einheitswurzel, da $\alpha^{\#G}=1$, sozusagen $\alpha =
\sqrt[\#G]{1}$. primitiv, da $\alpha^m=1$, wobei $\#G\mid m$.
\end{bemerkung}

\paragraph{Spezialfälle}
\begin{enumerate}
\item $K=\MdF_q$, also ein Körper mit $q<\infty$ Elementen. $G=\MdF_q^\times=\MdF_q\setminus\{0\}$, $\#G=q-1$. Nach dem Satz ist $F_q^\times$ zyklisch $\alpha$ mit $\langle \alpha\rangle=\MdF_q^\times$ heißt primitives Element.
\item Noch spezieller: $\MdF_p = \MdZ/p\MdZ$ mit $p\in\MdP$ besitzt $\varphi(p-1)$ primitive Elemente $\alpha = \overline w$, $(0\le w < p-1)$. Solve $w$ heißen Primitivwurzel modulo $p$.
\end{enumerate}

\begin{beweis}
Sei $l = \#G$, $G$ wie im Satz.

Für die $d\mid l$, $d\in\MdNp$, sei $\lambda(d)=\#\{\alpha \in G\mid
\ord(\alpha)=d\}$. Laut Elementordnungssatz gilt: $l = \sum_{d\mid
l}\lambda(d) = \sum_{d\mid l} \varphi(d)$ (Lemma von Gauß). Man will
zeigen: $\lambda(d)\le \varphi(d)$ $(*)$, denn dann muss gelten:
$\forall d\mid l: \lambda(d)=\varphi(d)$, denn sonst würde gelten:
$\sum_{d\mid l}\lambda(d) < \sum_{d\mid l} \varphi(d)$.

$(*)$ ist klar, wenn $\lambda(d) = 0$. Sei also $\lambda(d)\ne 0
\folgt \exists \alpha \in G: \ord(\alpha) = d$. Sei $A=\langle
\alpha \rangle = \{1,\alpha,\alpha^2,\ldots,\alpha{d-1}\}$. Klar:
$(\alpha^d)^d = 1 \folgt \alpha^j$ ist eine Nullstelle von $X^d-1$.
Wegen $\#A = d$ sind das $d$ Nullstellen von $X^d-1$, also alle
solche. $B=\{\beta \in G\mid \ord(\beta) = d \}$, dann $\beta^d = 1
\folgt \beta$ Nullstelle von $X^d-1\folgt \beta\in A$. $B\subseteq
A$.

$\alpha^j\in B \iff \ord(\alpha^j) = d \folgt d= \ord(\alpha^j) =
\frac{\ord(\alpha)}{\ggt(d,j)}$ (Elementordnungssatz) $\folgt
\ggt(d,j) = 1 \folgt B\subseteq \{\alpha ^j \mid \ggt(d,j)=1, 0\le j
\le d\}$. $\#B=\lambda(d)\le \#\{\alpha ^j \mid \ggt(d,j)=1, 0\le j
\le d\} = \varphi(d)$

\end{beweis}

Der folgende Satz ist eine Anwendung des Primitivwurzelsatzes:

\begin{satz}[Eulers Quadratkriterium]
Sei $\alpha\in\MdF_q^\times$ ($\MdF_q$ ein Körper mit $q$ Elementen,
$2\mid q$). Dann gilt:
\[ \alpha \text{ ist ein Quadrat in }\MdF_q^\times \iff \alpha^{\frac{q-1}{2}} = 1 \]
Anderenfalls gilt: $\alpha^{\frac{q-1}{2}} = -1$

Euler formuliert den Satz so: Sei $p\in\MdP$, $p>2$, $n\in\MdZ$,
$p\mid m$. Dann existiert ein $x\in\MdZ$ mit $x^2 \equiv m \mod p
\iff m^{\frac{p-1}2} \equiv 1 \mod p$. Solche $m\mod p$ heißen
quadratische Reste.

Wenn Kongruenz als Gleichung in $\MdF_p = \MdZ/p\MdZ$ gelesen wird,
so gilt:
\[ \alpha = \overline x \text{ Quadrat in }\MdF_p^\times \iff x \text{ quadratischer Rest modulo }p \]
\end{satz}

\begin{beweis}
Sei $\zeta$ eine Primitivwurzel (Existenz folgt aus dem
Primitivwurzelsatz).

"`$\Longleftarrow$"': Sei $\alpha^{\frac{q-1}2} = 1$ und $\alpha  =
\zeta ^j$. $\zeta^{j\cdot\frac{q-1}2} = 1 \folgt q-1=\ord(\zeta)\mid
j^{\frac{q-1}2} \folgt \frac j 2 \in \MdZ \folgt 2 \mid j \folgt
\beta = \zeta^{\frac j 2}$ zeigt den Satz: $\beta^2 = \zeta ^j =
\alpha$

"`$\Longrightarrow$"': $\alpha$ Quadrat $\iff \exists \beta
\in\MdF_q: \alpha = \beta^2 \folgt \exists k\in\MdZ: \beta =
\zeta^k$. $\alpha= \zeta^{2k} \folgt \alpha ^{\frac{q-1}2} = \zeta
^{(q-1)k} = 1$, da $\ord(\zeta) = q-1$

$\alpha^{\frac {q-1}2}$ ist Nullstelle von $X^2-1$. Alle Nullstellen
sind $\{1,-1\}$. $1$ entfällt, also ist $\alpha^{\frac{q-1}2} = -1$
\end{beweis}

% Wer das verstanden hat darfs gerne umformulieren:
Eulers Formulierung "`$m$ nicht quadratischer Rest"', auch
"`quadratischer Nichtrest"'. $\ggt(m,p) = 1 \folgt m^{\frac{p-1}2}
\equiv -1 \mod p$

\section{Zifferndarstellung nach Cantor}

In diesem Abschnitt seien $R=\MdZ$ oder $R=K[X]$, $K$ ein Körper.

Ausgangspunkt ist die Folge $\gamma=(m_0,m_1,m_2,\ldots)$, $m_j\in
R$ mit $m>1$ bei $R=\MdZ$ oder $\grad(m_j)>0$ bei $R=K[X]$.

Definiere $M_0=1$, $M_k=m_0\cdot \ldots \cdot m_{k-1}$.

\begin{satz}[Ziffernsatz]
Jedes $n\in\MdNp$ bzw. $n\in K[X]$, $n\ne 0$ hat eine eindeutige
Darstellung
\[n = z_rM_r + z_{r-1}M_{r-1} + \cdots + z_1M_1 + z_0 \quad (*)\]
wobei $r\in\MdN$ und $0\le z_j < m_j$ bzw. $\grad(z_j) < \grad(m_j)$

\end{satz}

\paragraph{Bezeichnungen:}
Die $z_j$ heißen $\gamma$-adische Ziffern und $(*)$
Zifferndarstellung (vorlesungs-spezifisch). Kurzbezeichnung:
$n=(z_r,z_{r-1},\ldots,z_0)_\gamma$. Die Kommata dürfen bei
Eindeutigkeit weggelassen werden.

Spezialfall: $m_0=m_1=m_2=\cdots=:m$ gibt Zifferndarstellung
$n=z_rm^r + z_{r-1}m^{r-1}+\cdots+z_0 = (z_r,\ldots,z_0)_m$ heißt
$m$-adische Darstellung von $n$.

Speziallbenennungen:\\
\begin{tabular}[h]{r|l|l|l}
$m$ & Zifferndarstellung & Ziffern &  \\
\hline
10 & Dezimaldarstellung & 0,1,\ldots,9  & bei Menschen beliebt\\ &&& (10 Finger) \\
2 & Binär oder dyadisch & 0,1 & bei Comptern beliebt \\ &&& (0,1 gut realisierbar)  \\
8 & Oktaldarstellung & 0,\ldots,7 & \\
16 & Hexadezimal & 0,\ldots,9,A,B,C,D,E,F & Speicherverwaltung \\ &&& im Rechner \\
\end{tabular}

\begin{beispiel}
\begin{align*}
(A8C)_{16} &= 10\cdot 16^2 + 8\cdot 16 + 12\cdot 1 \\
&= 2700 := (2700)_{10} \\
&= (10101001100)_2 \\
&= (5214)_8
\end{align*}
\end{beispiel}

$\gamma = (m_0, m_1, ...), m_j \in \MdZ$ (bzw. $K[X]$), $m_j > 1$ bzw. $\text{Grad } m_j > 0$\\
$M_0 = 1, M_k = m_0 \cdot ... \cdot m_{k-1}$

$\gamma$-adische Entwicklung von $n \in \MdN_+$ bzw. $n \in K[X], n \not= 0:$
\begin{equation}\label{eq:3.3star}
     n = z_rM_r + z_{r-1}M_{r-1} + ... + z_1M_1 + z_0 \cdot 1
\end{equation}
$\gamma$-adische Darstellung, wenn $0 \le z_j < m_j$ (bzw.
$\text{Grad }z_j < \text{Grad }m_j$)

\begin{beweis}[Ziffernsatz]
Fall \eqref{eq:3.3star} vorliegt: Wegen $M_k \big| M_{k+1} \big| M_{k+2} \big| ...: \\
n \equiv z_{k-1}M_{k-1} + z_{k-2}M_{k-2} + ... + z_0 \mod M_k$

Speziell: $n \equiv z_0 \mod M_1 = m_0 \implies n-z_0 = n'm_0, n'
\in \MdZ$ bzw. $K[X]$

\underline{Beweisidee:} Induktion nach $n$ bzw. $\text{Grad }n$
(hier nur $\MdZ, K[X]$ fast genau so)

\underline{Behauptung:} Sei $n \in \MdZ_+$. Dann existiert für alle $\gamma$'s dieser Art die $\gamma$-dische Darstellung \eqref{eq:3.3star}.\\
Induktion nach $n$:\\
Falls $n < m_0$, dann $z_0 = n, n = z_0M_0$ ist ($\star$)\\
Falls $n \ge m_0, z_0 = (n \mod m_0), n'$ aus $n - z_0 = n'm_0 (n' = \frac{n-z_0}{m_0})$. Klar $0 \le z_0 < m_0 \le n \implies 0 < n' < n$.\\
Induktionshypothese anwendbar auf $n'$ mit $\gamma' = (m_1', m_2',
...), m_j' = m_{j+1} (j \ge 0)$.

$\exists \gamma'$-adische Darstellung von $n'$:\\
$n' = z_{r'}'M_{r'}' + z_{r'-1}M_{r'-1}' + ... + z_1'M_1' + z_0' (r' \in \MdN, z_{r'}' \not= 0)$\\
$n \le z_j' < m_j' = m_{j+1} \implies n = n'm_0 + z_0 = z_{r'}'M_{r'+1} + ... + z_1'M_1 + z_0$\\
Das ist die gesuchte $\gamma$'-adische Darstellung von $n$ mit $r := r' + 1, z_j' = z_j + 1 (j = 0, ..., r')$ also $0 \le z_{j+1} = z_j < m_j' = m_{j+1}$\\
Dies ist ein Algorithmus, wenn die Abbildung $j \mapsto m_j$
berechenbar ist.

\underline{Eindeutigkeit:} Ebenfalls Induktion. $z_0$ muss $n \mod
m_0$ sein. Induktionshypothese $n'$ eindeutig dargestellt $\implies$
Darstellung von $n$ eindeutig (Details: selbst!)
\end{beweis}

\underline{Bemerkung:} Zur Berechnung von $(n_1 +/\cdot n_2)_\gamma$
aus $(n_1)_\gamma$ und $(n_2)_\gamma$ ähnliche Algorithmen wie für
$()_{10}$.

\section{Simultane Kongruenzen}

\subsection{Prinzip des Parallelen Rechnens}

$R_j (j = 1,..., l)$ seien algebraische Strukturen gleicher Art mit
gleichbezeichneten Verknüpfungen $\ast$, zum Beispiel:
\begin{itemize}
    \item[] Gruppen $\ast \in \{\cdot\}$
    \item[] Abelsche Gruppen $\ast \in \{+\}$
    \item[] Ringe $\ast \in \{+, \cdot\}$
    \item[] Vektorräume $\ast \in \{+, \text{Skalarmultiplikation}\}$
\end{itemize}

Dann ist auch $S = \prod_{i=1}^lR_j = R_1 \times ... \times R_l$ eine algebraische Struktur mit Verknüpfungen (komponentenweise):\\
$S \ni (a_1,...,a_l), (b_1,...,b_l), a_j, b_j \in R_j$\\
$(a_1, ...,a_l) \ast (b_1,...,b_l) := (a_1 \ast b_1, ..., a_l \ast b_l)$\\
$\alpha(a_1, ..., a_l) := (\alpha a_1, ..., \alpha a_l)$ bei K-Vektorräumen.\\
Sind $_j$ Ringe/Gruppen/Abelsche Gruppen/Vektorräume, so
\underline{auch $S$}.

\underline{Grund:} Alles vererbt sich von den Komponenten!\\
Zum Beispiel Ringe: $0_S = (0_{R_1}, ..., 0_{R_l}), 1_S = (1_{R_1}, ..., 1_{R_l})$, kurz: $0 = (0,...,0), 1 = (1,...,1)$, $-(a_1,...,a_l) = (-a_1, ..., -a_l)$\\
Zum Beispiel Assoziativität:\\
$((a_1,...,a_l) \ast (b_1,...,b_l)) \ast (c_1,...,c_l) = ((a_1 \ast
b_1) \ast c_1, ..., (a_l \ast b_l) \ast c_l) = (a_1, ..., a_l) \ast
((b_1, ..., b_l) \ast (c_1, ..., c_l))$

\underline{\textbf{Warnung!}} Sind die $R_j$ Körper, so ist für $l > 1$, $S$ \underline{kein} Körper.\\
Zum Beispiel: $\underbrace{(1,0)}_{\not= 0} \cdot
\underbrace{(0,1)}_{\not= 0} = (1 \cdot 0, 0 \cdot 1) = (0, 0) = 0$

\begin{lemma}
Sind dir $R_j$ Ringe, so $S^\times = \prod_{j=1}^lR_j^\times$
\end{lemma}

\underline{Grund:} Muss sein $(a_1, ..., a_l)^{-1} = (a_1^{-1}, ...,
a_l^{-1})$

Falls ein Isomorphismus $\psi: R \to S = \prod_{j=1}^lR_j$ vorliegt, so wird das Rechnen in $R$ zurückgeführt auf das gleichzeitig ("`\emph{parallele}"') Rechnen in dem $R_j$ wie folgt:\\
$\psi(a) = (a_1,...,a_l), \psi(b) = (b_1,...,b_l)$\\
$a \ast b = \psi^{-1}(\psi(a \ast b)) = \psi^{-1}(\psi(a) \ast
\psi(b)) = \psi^{-1}((a_1 \ast b_1, ..., a_l \ast b_l))$

\underline{Praxis:} Berechne die $a_j \ast b_j$ gleichzeitig auf
verschiedenen Prozessoren. Wende $\psi$, $\psi^{-1}$ wie oben an.
Nützt nur, wenn $\psi$, $\psi^{-1}$ gut und schnell berechenbar
sind.

\subsection{Der Chinesische Restsatz}

\underline{Frage:} Morgen ist Freitag, der 2. Juni. Nach wievielen
($x$ = ?) Tagen fällt frühestens der Dienstag auf einen 17. des
Monats?

\underline{Vorraussetzung:} Chinesische Kalender vor ca. 2000
Jahren: Alle Monate haben 20 Tage.

\begin{tabular}{|r|c|c|c|c|c|c|c|c|c|c|}
    \hline
    Wochentag     & Fr & Sa & So & Mo & Di & Mi & Do & Fr & Sa & So \\
    \hline
    Wochentagsnr. & 0  & 1  & 2  & 3  & 4  & 5  & 6  & 0  & 1  & 2  \\
    \hline
    Monatstagnr.  & 2  & 3  & 4  &  5 & 6  & 7  & 8  & 9  & 10 & 11 \\
    \hline
\end{tabular}
\\(Wochentagsnummer modulo 7, Monatstagnummer modulo 30)

Gesucht ist also die kleinste positive Lösung $x$ der Kongruenzen:
\begin{alignat*}{2}
    x &\equiv 4 &&\mod 7\\
    x &\equiv 17-2 &&\mod 30
\end{alignat*}

$R$ sei euklidischer Ring, $a_1,\ldots,a_l, m_1,\ldots,m_l \in R$\\
\begin{equation}\label{eq:SystemSimultanerKongruenz}
x \equiv a_j \mod m_j,\ (j = 1, \ldots, l)
\end{equation}
heißt \emph{System simultaner Kongruenzen} (mit gesuchter Lösung $x
\in R$).

\begin{bemerkung}
    Im Allgemeinen gibt es \emph{keine} Lösung.\\
    $x \equiv a \mod m \implies x \equiv a \mod m$, falls $d \mid m$\\
    System: $x \equiv 1 \mod 4, x \equiv 0 \mod 6 \implies x \equiv 1 \mod 2, x \equiv 0 \mod 2 \implies 1 \equiv 0 \mod 2 \implies$ Widerspruch!
\end{bemerkung}

\begin{satz}[Chinesischer Restsatz, rechnerische Form]
    Sei $R$ ein euklidischer Ring, $m_1, ..., m_l \in R$, $a_1,...,a_l \in R$ derartig, dass $\forall i,j \in \MdZ$ mit $1 \le i < j \le l$ gilt:\\
    \[ \ggt(m_i, m_j) = 1\text{ ("`paarweise relativ prime $m_j$"')}\]
    Dann hat das System simultaner Kongruenzen \eqref{eq:SystemSimultanerKongruenz} eine Lösung. Sämtliche Lösungen bilden \underline{eine} Restklasse modulo $m$ mit $m = m_1 \cdot \ldots \cdot m_l$
\end{satz}

\begin{beweis}
\begin{description}
    \item{$l=1$:} $x = a_1$ oder $x = (a_1 \mod m_1) \iff (x \equiv a_1) \mod m_1$ und $0 \le x \le m_1$
    \item{$l=2$:} $x \equiv a_1 \mod m_1$. $x$ muss in der Form $x = a_1 + um_1, u \in R$ angesetzt
        werden.\\
        \emph{Idee}: Bestimme $u$ so, dass $x \equiv a_2 \mod m_2$. Also in $\overline R = R / m_2R$
        soll werden:\\
        $\overline a_1 + \overline u \overline m_1 = \overline{a_1 + um_1} = \overline a_2$, daher tut
         es: $\overline u = (\overline a_2 - \overline a_1)\overline m_1^{-1}$\\
    Geht, da $\overline m_1^{-1}$ existiert und da $\overline m_1 \in (R/m_2R)^\times$. Nach
    dem Restklassensatz: $\overline m_1 \in (R/m_2R)^\times \iff \ggt(m_1,m_2) = 1$\\
    Algorithmisch $\overline u = \overline m_1^{-1}$, $u$ kann mit LinKom-Satz, also
    euklidischem Algorithmus, bestimmt werden. \emph{Erinnerung}: $\ggt(m_1,m_2) = um_1 + vm_2,\ u,v$
    berechnet der Algorithmus.\\
    $1 = \overline u \overline m_1, \overline m_2 = 0, \overline u = \overline m_1^{-1}$\\
    Für dieses $u \in R$ ist $x = a_1 + um_1$ (eventuell $\mod m, m_2$) die gesuchte
    Lösung.
    \item{$l>2$:} Induktionshypothese löst $x' \equiv a_j \mod m_j (j = 1,..., l-1)$.\\
     Löse dann $x \equiv x' \mod m_1 \cdot ... \cdot m_{l-1}$ ($\implies x \equiv x' \equiv a_j \mod m_j, j=1,...,l-1) \implies x \equiv a_l \mod m_l \implies$ $x$ ist die gesuchte Lösung.
\end{description}
\end{beweis}

\begin{beispiel}
Gegeben sind die Kongruenzen:
\begin{alignat*}{2}
    x &\equiv 4 &&\mod 7\\
    x &\equiv 19 &&\mod 30
\end{alignat*}
Ansatz: $x=4 + u\cdot 7 \equiv 19 \mod 30$. Im $\MdZ/30\MdZ$:
$\overline 4 + \overline u\cdot \overline 7  = \overline {19} \folgt
\overline u = (\overline {19} - \overline 4)^{-1} \cdot \overline
7^{\,-1}$. Es ist $\overline 7^{\,-1}=\overline {13}$, also $u
\equiv 13\cdot 15$, etwa $x = 4 + 13 \cdot 15  \cdot 7 \equiv 109
\mod 210$.

Wir fügen eine Bedingung hinzu: $x\equiv 1 \mod 77$. So ist nun zu
lösen:
\begin{alignat*}{2}
    x &\equiv 109 &&\mod 30\\
    x &\equiv 1 &&\mod 11
\end{alignat*}
Es ist $\overline {210}^{\,-1} = \overline 1$ im $\MdF_{11}$, also
$x=109+2\cdot 210 \equiv 529 \mod 11\cdot3\cdot 7$
\end{beispiel}

\begin{bemerkung}[zur Praxis]
% Hilfe! wer das verstanden hat oder korrekt mitgeschrieben hat, bitte korrigieren und ggf. erklären.
Das Sytem $x\equiv x_i \mod m_i$, $(i=0,\ldots,l)$. Der Beweis
liefert eine $\gamma$-adische Darstellung von $x$ und $m=y$
$\gamma=(m_0,\ldots,m_l)$ wie folgt: $y = z_{l-1}M_{l-1}+\cdots+
z_0$. Die $z_i$ sind rekursiv aus $z_0 = x_0 \bmod m_0$, $y' \equiv
x_i' \bmod m_j$, $(i=1,\ldots,l)$. Also $y'=\frac{x-z_0}{m_0}$,
$x_i' = (x_i - z_0)u_{i0} \bmod m_j$. $\overline {u_{i0}} =
\overline {m_0^{-1}}$ in $\MdZ/m_i\MdZ$. $x_i'$ in
$\gamma'$-adischer Darstellung nach Induktions-Voraussetzung
$(\gamma'=(m_1,\ldots,m_l))$.

Empfehlung zur Praxis, vor allem wenn viele Kongruenzen zu den
selben $m_i$ zu lösen sind:
\begin{enumerate}
\item Berechne die $u_{ij}$ nur einmal.
\item Belasse die Ergebnisse $m$ in der Form $x=(z_{l-1},\ldots,z_0)_\gamma$
\end{enumerate}
\end{bemerkung}

Zum paralellen Rechnen: Seien $R,m_1,\ldots,m_l$ wie im chinesischen
Restsatz. Betrachte  die Abbildung
\begin{align*}
R/mR &\to \prod_{j=1}^l (R/m_jR) \\
\psi: x + mR &\mapsto (\ldots, x + m_jR, \ldots )
\end{align*}
$\psi$ ist wohldefiniert: $x+mR = x'+mR \iff x \equiv x' \mod m \iff
x \equiv x' \mod m_j$ und ein Ringhomomorphismus (leicht zu sehen).

Wir beobachten: Ist $\psi: A\to B$ eine Abbildung, so gilt, dass
$\psi$ injektiv genau dann ist wenn die Gleichung $\psi(x) = b$
höchstens eine Lösung $x$ hat. Surjektivität heißt analog, dass jede
Gleichung $\psi(x) = b$ mindestens eine Lösung $x$ hat. $\psi$
bijektiv ist dann gleichbedeutend damit, dass $\psi(x) =b $ genau
eine Lösung hat.

Für obiges $\psi$ gilt: $b = (\ldots,a_j+ m_jR,\ldots)$.
$\psi(x+m_jR)=b$: $(\ldots,x+m_jR,\ldots) =(\ldots,a_j+m_jR,\ldots)
= b$. $x+mR$ Urbild von $b$ $\iff \forall j: x+m_jR_j = a_j+m_jR
\iff \forall j: x \equiv a_j \mod m_j$. Also:
\begin{itemize}
\item $\psi$ surjektiv $\iff \forall b \exists \text{Lösung } x\equiv a_j \mod m_j$
\item $\psi$ injektiv $\iff$ Lösung $x$ ist eindeutig modulo $m$
\end{itemize}
Ergebnis: Der chinesische Restsatz wie oben ist gleichbedeutend mit:

\begin{satz}[Theorem B, Chinesischer Restsatz, theoretische Form]
$R$ ein euklidischer Ring, $m_1,\ldots,m_l\in R$, $\ggt(m_i,m_j)=1$
für $i\ne j$. Dann hat man den Ringisomorphismus:
\begin{align*}
R/mR &\to \prod_{j=1}^l (R/m_jR) \\
\psi: x + mR &\mapsto (\ldots, x + m_jR, \ldots )
\end{align*}
\end{satz}

\begin{bemerkung}[Zur Praxis]
$\psi^{-1}$ wird gegeben durch lösen simultaner Kongruenzen.
"`Komponentenweises Rechnen: Rechnen im $R/mR$ ersetzt durch
paralleles Rechnen in den $R/m_jR$"'
\end{bemerkung}

\begin{bemerkung}[Theoretische Anwendung]
Voraussetzungen wie im Satz. Die Einheitengruppe $(R/mR)^\times$ ist
isomorph durch $\psi$ zu $\prod_{j=1}^l (R/m_jR)^\times $. Ist
$R=\MdZ$, so gilt $\varphi(m) = \prod_{j=1}^l \varphi(m_j)$, also
ein neuer Beweis für die Multiplikativität von $\varphi$.
\end{bemerkung}

\section{Ausgewählte Anwendungen von Kongruenzen}

\subsection{Diophantische Gleichungen}
Sei $0\ne f \in \MdZ[X_1,\ldots,X_n]$ (Polynom mit $n$ Unbekannten
und Koeffizienten aus $\MdZ$), $x=(x_1,\ldots,x_n)\in\MdZ^n$.

Eine diophantische Gleichung ist eine Gleichung der Form $f(x)=0$,
$f$ wie oben, mit eine "`Lösung $x$"'.

Der Wunsch hier ist: Man finde möglichst viel Informationen über die
Menge $\mathcal{V}_f(\MdZ) := \{ x\in\MdN^n \mid f(x) = 0\}$ aller
ganzzahligen Lösungen.

Das Problem ist oft extrem schwierig. Zum Beispiel die
diophantischen Gleichungen $x^n+y^n+z^n=0$, $x=(x,y,z)$, auch
bekannt als das Fermatproblem.

Information für Logik-Freunde: Das 10. Hilbertsche Problem (Paris
1900):
\begin{quote}
Man finde einen Algorithmus, der zu gegebenem $f\in
\MdZ[X1,\ldots,X_n]$ entscheidet, ob $\mathcal{V}_f(\MdZ) =
\emptyset$ oder $\mathcal{V}_f(\MdZ) \ne \emptyset$ ist.
\end{quote}
Satz von Julia Robison (1910-85), J. Matjasevi\v{c}: Es gibt keinen
solchen Algorithmus!

Triviale, aber wichtige Methode: $f(x)=0$ hat Lösung $x\in\MdZ^n$
$\folgt$ $f(x)=0$ hat Lösung $x \in \MdR^n$ (Analysis) und $\forall
m \in \MdZ: f(x) \equiv 0 \mod m$ lösbar $\iff \forall t \in \MdNp
\forall p\in\MdP: f(x) \equiv 0 \mod p^t$ lösbar. Die Folgerung ist,
dass falls für ein $m\in\MdNp$ gilt, dass für alle
$(x_1,\ldots,x_n)\in\MdZ^n$, $0\le x_j < m_j$ gilt: $f(x)\not\equiv
0 \mod m$, so gilt $\mathcal{V}_j(\MdZ)= \emptyset$, es gibt also
keine Lösung.

\begin{beispiel}
$f=X_1^2 + X_2^2 - k$, $k\in\MdZ$, diophanischsche Gleichung $x_1^2
+ x_2^2 = k$. Unlösbar für $k<0$ (da keine Lösung in $\MdR^2$). Nur
interessant: $k>0$.

Betrachtung modulo 4:\\ $0^2 = 0, (\pm1)^2 = 1, (\pm2)^2= 0 \folgt
(x_1^2 + x_2^2) \bmod 4 =
\begin{cases}
0+0 \\ 0+1 \\ 1+1
\end{cases} \in \{0,1,2\}$.

Für $k\equiv 3 \mod 4$ hat $x_1^2 + x_2^2 = k$ also keine
ganzzahlige Lösung!

Es kann eine Primzahl $p\ne 2$ nur dann Summe zweier Quadrate sein,
wenn $p\equiv 1 \mod 4$ ist. Hier gilt auch die Umkehrung, Beweis
folgt eventuell später.
\end{beispiel}

\begin{beispiel}
$f=X_1^2 + X_2^2 + X_3^2 - k$, also $x^2 + y^2 + z^2 = k$. Modulo 4
führt hier zu keiner Aussage. Wie betrachten modulo 8: $0^2 = 0$,
$(\pm1)^2 = 1$, $(\pm2)^2 = 4$, $(\pm3)^2=1$, $(\pm4)^2 = 0$. Also
gilt: $(x_1^2 + x_2^2 + x_3^2) \bmod 8 =
\begin{cases}
0+0+0 \\ 0+1+0 \\ 1+1+1 \\ 1+1+1 \\ 0 + 4+ 0 \\ \ \ \ \ \ \,\vdots  % Spacing-Hack
\end{cases} \in \{0,1,2,3,4,5,6\}$.

Ergebnis: Für $k<0$ oder $k\equiv 7 \mod 8$ hat die Diophantische
Gleichung $x_1^2 + x_2^2 + x_3^2 = k$ keine Lösung.

Zur Information, nach Gauß: Die Umkehrung gilt auch für ungerade $k$. % Hä?

Satz von Lagrange: $x_1^2 + x_2^2 + x_3^2 + x_4^2 = k$ ($k\in\MdN$)
hat immer Lösungen.
\end{beispiel}

Gelegentlich erlangt man Ergebnisse auch über andere Gleichungen:

\begin{beispiel}
Gesucht sind  Lösungen von $9^x + x^3 = k$ mit $x\in\MdNp$.

Betrachtung modulo 9: $9^x \equiv 0 \mod 9$. $0^3 = 0$, $(\pm1)^3 =
\pm1$, $(\pm2)^3 = \mp 1$, $(\pm3)^3 = 0$, $(\pm4)^3 = \pm1 \folgt
x^x+ x^3 \equiv 0 , \pm 1 \mod 9$. Ergebnis: Für $k \equiv
2,3,4,5,6,7 \mod 9$ hat die Gleichung keine Lösung in $x\in\MdZ$.
\end{beispiel}

\subsection{Interpolation}

Hier sei $R=K[X] \ni f$, $\alpha,\beta \in K$:
\begin{align*}
f(\alpha) = \beta &\iff (f-\beta)(\alpha) = 0 \\
&\iff (X-\alpha)\mid f-\beta \\
&\iff f \equiv \beta \mod (X-\alpha)
\end{align*}
Das Sytem $f \equiv \beta_j \mod (X-\alpha_j)$ $(j=0,\ldots,n) \iff
\forall j=0,\ldots,n : f(\alpha_j)=\beta_j$ (Vorraussetzung
$\alpha_i\ne \alpha_j$ für $i\ne j$, d.h
$\ggt(X-\alpha_i,X-\alpha_j)\ne 0$).

Der Chinesische Restsatz ergib nun: Zu gegebenen $n+1$ Punkten
$\alpha_0, \ldots, \alpha_n \in K$ ($\alpha_i \ne \alpha_j$) und
Punkten $\beta_0, \ldots, \beta_n\in K$ gibt es genau ein $f \bmod
(X-\alpha_0)\cdots(X-\alpha_n)$, also $\ord(f) \le n$ mit
$f(\alpha_j) = \beta_j$. Damit ist das Interpolationsproblem gelöst.

%Vorlesung  08.06.2006

\emph{Frage:} Kann man bei Interpolation die Tangentensteigung
(allgemein $f^{(j)}(\alpha_k)$) auch vorschreiben (Hermitesche
Interpolationsaufgabe)? Ja für $K=\MdQ, \MdR, \MdC$ (Übung).

$f\in K[X]$, $(X-\alpha)$-adische Darstellung. Ziffern $z_j \in
K[X]$ haben Grad $z:j < \grad (X-\alpha)=1$, das heißt $z_j \in K$.
$f=\sum_{j=0}^n z_j (X-\alpha)^j$, das ist die Taylor-Entwicklung in
$\alpha$. $z_j$ gegeben durch $\frac{f^{(j)}(\alpha)}{j!}$.
\begin{equation}\label{eq:3.5stern}
    f \equiv g_{\alpha,d} \mod (X-\alpha)^{\alpha+1}, \qquad
    g_{\alpha,d}:=\sum_{j=0}^d z_j(X-\alpha^j)
\end{equation}
$g_{\alpha,d}$ ist gegeben durch
$f(\alpha),f'(\alpha),\dotsc,f^{(d)}(\alpha)$.\\System
\eqref{eq:3.5stern} entspricht der Vorgabe der $f^{(j)}(\alpha)$,
Interpolation mit $m_{j,k}=(X-\alpha_k)^{d_j}$ ist lösbar mit dem
Restsatz.

\subsection{Rechnen im Computer mit großen ganzen Zahlen}

Prinzip: Gleicheit in \MdZ\ entspricht Kongruenz und einer passender
Abschätzung.
\begin{bemerkung}
    $m\in\MdN,\ m>1$, etwa $2 \nmid m$. Ist $u \equiv v \mod m$ und
    $|u|,|v|\leq \frac{m}{2}$, so ist $u=v$, weil $u,v$ sind im
    symmetrischen $\text{Versys}_m$.
\end{bemerkung}
Wende dies an auf die Berechnung von $f(x), f\in
\MdZ[X_1,\dotsc,X_n],\ x=(x_1,x_2,\dotsc,x_n) \in \MdZ^n$. Kennt man
eine Schranke $|f(x)|<\frac{m}{2}$, so genügt es, $f(x) \mod m$
auszurechnen. $f(x) \mod m$ kann für $m=m_1 \cdot \dotsb \cdot m_l$
durch Berechnen von $y_j=f(x) \mod m_j\ (j=1,\dotsc,l)$ ersetzt
werden, das ergibt simultane Kongruenz $y=y_j \mod m_j$, die mit dem
chinesischen Restsatz gelöst werden kann.

\section{Struktur der Primrestklassengruppe mod $m$}
$R$ euklidisch, $m=\prod_{i=1}^l p_i^{t_i}$ Primzerlegung, $t_j \in
\MdN_+$. Aus dem Chinesischen Restsatz: $(R/mR)^{\times} \cong
\prod_{j=1}^l (R/p_j^{t_j} R)$ (beachte:
$\ggt(p_i^{t_i},p_j^{t_j})=1$ für $i \neq j$). Es genügt also
$G:=R/p^t R$ mit $p\in P,\ t\in \MdN_+$ zu betrachten. Hier nur der
Fall $R=\MdZ$ $(R=\MdF_p[X ]$ geht ähnlich).

\emph{Erinnerung:} $t=1,\ \MdZ/p\MdZ=\MdF_p,\ \MdF_p^{\times}$ ist
zyklich, es existiert eine Primitivwurzel $w \mod p$.

\emph{Frage:} Wie ist der Fall für $t>1$?

Für $p>2$ existiert eine Primitivwurzel!

Gesucht ist also eine Primitivwurzel $u$, das heißt $\ord \overline
u = \varphi\left(p^t\right)=(p-1)p^{t-1}$ in $G$. Es genügt $u_1,u_2
\in \MdZ$ mit $p-1 \mid \ord \overline u_1$ und $p^{t-1} \mid \ord
\overline u_2$ zu finden. Wegen $\ord \overline u_j \mid
\#G=(p-1)p^{t-1}$ gilt $s \mid p-1$. Daraus folgt, für
$v_1:=u_1^{p^{t-1}},v_1:=u_2^{p-1}$ ist
\[
\ord \overline v_1=\ord \overline u_1^{p^{t-1}}=\frac{\ord \overline
u_1}{\ggt\left(\ord \overline u_1,p^{t-1}
\right)}=\frac{(p-1)p^r}{p^r}=p-1.
\]
Ebenso: $\ord\overline v_2=p^{t-1}$ (Nachrechnen). Aus Übungsaufgabe
3 (a) Blatt 7 folgt mit $u:=v_1v_2 \mod p^t,\ \ord \overline
u=(p-1)p^{t-1}$. Bevor wir fortfahren, benötigen wir noch ein Lemma,
das wir zum Beweis eines Hilfssatzes benötigen.

\begin{lemma}[$(1+p)$--Lemma]\label{lemma:1+p}
    $p\in\MdP,\ p>2,\ r\in \MdN_+,\ u\in \MdZ.$ Dann gilt:
    $(1+up)^{p^{r-1}} \equiv 1+up^r \mod p^{r+1}$.
\end{lemma}
\begin{beweis}
    Beweis via Induktion nach $r$.
    \begin{description}

    \item{$r=1$:} $(1+up)^{p^{1-1}}=1+up\equiv1+up^1 \mod p^2\quad
    \checkmark$.\\
    \item{$r>1$:} Induktionshypothese (für $r-1$):
    \[
        (1+up)^{p^{r-2}}\equiv 1+ up^{r-1} \mod p^r.
    \]
    $\implies(1+up)^{p^{r-2}}=1+up^{r-1}+zp^r$ mit $z\in\MdZ$
    $\implies
    (1+up)^{p^{r-1}}=\left((1+up)^{p^{r-2}}\right)p=\left(1+\left(up^{r-1}+zp^r\right)\right)^p=1+\sum_{i=1}^p
    \underbrace{{p \choose i}}_{\in  \MdZ}
    \underbrace{\left(up^{r-1}+zp^r\right)}^i_{=\left(p^{r-1}(u+zp)\right)^i=:c_i}$.\\
    \item{$r\geq 2,\ i>2$:} $v_p(c_i)=\underbrace{v_p\left({p \choose
    i}\right)}_{\geq 0} +
    v_p\left(p^{(r-1)i}\right)+\underbrace{v_p(u+zp)^i}_{\geq 0}
    \geq (r-1)i \geq (r-1)r>r+1 \implies p^{r+1} \mid c_1 \implies
    c_i \equiv 0 \mod p^{r+1}$.
    \item{$i=2$:} $v_p(c_2)=\underbrace{v_p\left( \frac{p(p-1)}{2} \right)}_{\geq
    1} + \underbrace{v_p\left( p^{2(r-1)} \right)}_{=2(r-1)} + \underbrace{v_p(u+zp)^2}_{\geq
    0} \geq 2r-2+1=2r-1 \geq r+1 \implies c_2 \equiv 0 \mod p^{r+1}$.
    \item{$i=1$:} $c_1=p\cdot p^{r-1}(u+zp)=up^r+zp^{r+1}\equiv up^r \mod
    p^{r+1}$.
    \end{description}
    $\implies$ Behauptung.
\end{beweis}
\begin{hilfssatz}
    Sei $p\in\MdP,\ p>2,\ t\in \MdN_+$.
    \begin{enumerate}
        \item Ist $w$ eine Primitivwurzel $\mod p$, so gilt in $G=\left(\MdZ/p^t
        \MdZ\right)^{\times}:\ p-1 \mid \ord \overline w,\ \overline
        w=w+p^t \MdZ$. (\emph{$u_1=w$ wählbar}).
        \item $\ord (\overline{1+p})=p^{t-1}$ (\emph{$v_2=1+p$
        wählbar}).
    \end{enumerate}
\end{hilfssatz}
\begin{beweis}
\begin{enumerate}
    \item Sei $l=\ord \overline w$, also $\overline w^l=1$, das
    heißt $w^l\equiv 1 \mod p^t$. $t\geq 1 \implies w^l \equiv 1
    \mod p^1 \implies $ in $\MdF_p$ ist $\overline w^l=1$, $\ord
    \overline w=p-1 \implies p\cdot a \mid l$ (Elementar-Ordnungssatz).
    \item Folgt aus Lemma \ref{lemma:1+p}
\end{enumerate}

$(1+p)^{p^{t-1}}\equiv1+1\cdot p^t \mod p^{t-1} \implies
(1+p)^{p^{t-1}}\equiv 1 \mod p^t \implies \overline{1+p}^{p^{t-1}}
\implies \ord \overline{1+p} \mid p^{t-1}$. Für $t \geq 2$ ist noch
zu zeigen: $(1+p)^{p^{t-2}} \neq 1$. $(1+p)^{p^{t-2}} \equiv 1+
p^{t-1} \mod p^t$ (nach Lemma \ref{lemma:1+p}).
$\overline{1+p}^{p^{t-2}}=\overline 1 +
\underbrace{\overline{p^{t-1}}}_{\neq 0} \neq \overline 1 =1$.
\end{beweis}

Gezeigt (für $p>2$):
\begin{satz}[Struktursatz für $(\MdZ/p^t \MdZ)^{\times}$, eigentlich ein Theorem]
Sei $p\in\MdP,\ t\in \MdN_+$. Dann gilt:
\begin{enumerate}
    \item Falls $p>2$, so ist $(\MdZ/p^t \MdZ)^{\times}$ zyklisch
    (das heißt, es gibt eine Primitivwurzel $u \mod p^t$, also $(\MdZ/p^t
    \MdZ)^{\times}=\{ 1,\overline u,\dotsc, \overline
    u^{p^{t-1}(p-1)-1}\}$
    \item Falls $p=2$: $(\MdZ/2 \MdZ)^{\times},\ (\MdZ/4
    \MdZ)^{\times}$ zyklisch. Für $t>2$ ist $(\MdZ/2^t
    \MdZ)^{\times}$ \emph{nicht} zyklisch, doch es gilt: Jedes
    $\overline a \in (\MdZ/2^t \MdZ)^{\times}$ lässt sich eindeutig
    in der Form $\overline a=\overline{(-1)}^{\ep} \cdot \overline 5
    ^s$ schreiben, mit $\ep \in \{0,1\}, s \mod 2^{t-2}$
    (eindeutig). $(\MdZ/2^t \MdZ)^{\times}$ ist sozusagen bis auf
    das Vorzeichen $(-1)^{\ep}$ zyklisch.
\end{enumerate}
\end{satz}

% === Vorlesung vom 14.06.2006 / Robert Geisberger === %

\underline{Info:}\\
Man kann sagen: Ist $u \in \MdZ$  Primitivwurzel mod $p^2$, so auch mod $p^t$ $\forall t \in \MdN_+$
\\

Es gibt viele Arbeiten über Primitivwurzeln, z. B. analytische Zahlentheorie (sehr schwierig) gibt Schranken $s(p)$ so, dass in $\left\{2,\dotsc,s(p)\right\}$ PW mod $p$ zu finden. \\

\underline{Artins Vermutung:} 2 (oder jedes $n \in \MdN_+, n \neq 1$) ist Primitivwurzel für $\infty$-viele $p \in \MdP$. \\
\\
\underline{Rechnen} in $(\MdZ/m\MdZ)^x$ auf dem Computer, falls viele Produkte zu berechnen sind. \\
Primzerlegung $m = p_1^{t_1}\cdot \dotsc \cdot p_l^{t_l}$ $t_j \in \MdN_+$ \\
Kodiere $a + m\MdZ = \overline a$ wie folgt: \\
Berechne vorab PW $u_j$ mod $p_j^{t_j}$ \\
\begin{tabular}{rcl}
$(\MdZ/m\MdZ)^x$ & $\to$ & $\prod_{j=1}^l(\MdZ/p^j\MdZ)^x$ \\
$\alpha=a+m\MdZ$ & $\mapsto$ & $(\dotsc,a+p_j,\dotsc)$
\end{tabular}

\begin{tabular}{lrcl}
Bijektiv: & $\alpha$ & $\leftrightarrow$ & $(\dotsc,r(\alpha,j),\dotsc)$ \\
 & $\alpha \cdot \beta$ & $\leftrightarrow$ & $(\dotsc,r(\alpha,j)+r(\beta,j) \mbox{ mod }p_j^{t_{j-1}}(p_j-1),\dotsc)$ 
\end{tabular} \\
$\alpha^{-1}$ ähnlich\\
\\
\underline{Zum Rechnen} mit großen ganzen Zahlen (Skizze) \\
\begin{tabular}{ll}
\underline{Prinzip:} & Gleichheit in $\MdZ$ = Kongruenz + passende Abschätzung \\
\underline{Bemerkung:} & $m \in \MdN, m > 1,$ etwa $2 \nmid m$. Ist $u \equiv v$ mod $m$ und $|u| \leq \frac{m}{2}$, $|v| \leq \frac{m}{2}$, so ist $u=v$. \\
\underline{Grund:} & $u, v$ sind in Versys$_m$ (symm. Vertretersystem der Reste mod $m$), also $u = v$.
\end{tabular}
Wende dies an auf die Berechnung von $f(x), f \in \MdZ[X_1,\dotsc,X_n], x = (x_1,\dotsc,x_n) \in \MdZ$. Kennt man Schranke $|f(x)| < \frac{m}{2}$ so genügt es $f(x)$ mod $m$ auszurechnen. \\
$f(x)$ mod $m$ kann für $m = m_1 \cdot \dotsc \cdot m_l$ durch Berechnen von $f(x)$ mod $m_j =: y_j$ $(j=1,\dotsc,l)$ ersetzt werden + 1x chinesischer Restsatz: $y \equiv y_j$ mod $m_j$. \\
\\

\underline{Aufgabe:} \\
Berechne mit dem Computer $\det A$ (exakt), $A \in \MdZ^{n \times n}$ \\
Soll sein $n$ mäßig groß, $A=(a_{ij})$, die $a_{ij}$ mäßig groß.\\
Naives Verfahren: Gauß-Algorithmus in $\MdQ$: \\
Ärger: Sehr große Integer-Zahlen als Zähler und Nenner entstehen während
der Rechnung unkontrolliert. Mögliche bessere Vorgehensweise, etwa $|a_{ij}| \leq s$
(Schranke) \\
Leibnitzformel: $\det A=\sum_{\pi \in
S_n}\text{sgn}(\pi)\prod_{i=1}^na_{i,\pi(i)}$
liefert Abschätzung $|\mbox{det }A|\leq s^n\cdot n!$ (${n! = \#S_n}$)\\
Schranke $S = 2\cdot|\mbox{det }A| = 2 \cdot s^n\cdot n!$ kann sehr
groß sein. Wähle Primzahlen ($\neq 2$) $p_1,\dotsc,p_t$
($t$~verschieden) mit $S \leq p_1\cdot \dotsc \cdot p_t$. Dann
$|\mbox{det }A| \leq \frac{p_1\cdot \dotsc \cdot p_t}{2} = \frac{m}{2}\mbox{, }m=p_1\cdot \dotsc \cdot p_t$\\
Kann oft sein: $t$ mäßig groß, alle $p_j$ mäßig groß. (z. B.:
$s=100$, $n=100 \Rightarrow S = 100^{100}\cdot 2 \cdot 100! \leq 2 \cdot 100^{120}$
Es reichen also 130 $p_j$'s mit $p_j > 100$, diese können $< 1000$ gewählt werden
$\Rightarrow$ in $\MdF_{p_j}$ kann sehr gut und schnell gerechnet werden!\\
$\Rightarrow$ Berechnung von $\det \overline{A}$, $\overline{A} =
(\overline{a_{ij}})$ in $\MdF_{p_j}^{n \times n}$ kann durch
Herstellen von Dreiecksform von $\overline{A}$ für mäßig große $n$
schnell berechnet werden. (Durch Arbeiten in Versys$_p$ entstehen
niemals große Zahlen!) Das ergibt $y_j \in \mbox{Versys}_p$ mit
$\det A \mod p_j = y_j$. Es ist dann $y \equiv y_j$ mod $p_j$ zu
lösen (simultane Kongruenz $m=p_1\cdot \dotsc \cdot p_t$). Daher für
$y \in$ Versys$_m$ (symm.) ist $\det A=m$. $y$ kann sehr groß sein,
aber die Kongruenz ergibt sehr große Zahlen nur kontrolliert! (Mäßig
große Zahlen, falls man mit $\gamma$-adischer Darstellung von
$y=\det A$, $\gamma=(p_1,\dotsc,p_t,\dotsc)$ zufrieden ist.


\chapter{Endliche Körper und der Satz von Chevalley}

Schon bekannt:
\begin{enumerate}
\item $\forall p \in \MdP$ gibt es den Körper $\MdF_p = \MdZ/p\MdZ$ mit $\#\MdF_p=p$
\item Hat man ein irred. Polynom (Primpolynom) $g$ in $R = \MdF_p[X]$ mit Grad $g = n$, so ist $\overline R = R/gR$ ein Körper mit $q=p^n$ Elementen, der $\MdF_p$ als Teilkörper enthält.
\item Jeder endl. Körper $L$ enthält primitives $\zeta$, $L^x = L\setminus 0 = \left\{1,\zeta,...\right\}$.
\end{enumerate}

\section{Untersuchung eines endl. Körpers $L$ mit $\#L=q$}
ord$(1)=p=$min$\left\{n \in \MdN_+ | n \cdot 1_L = 0\right\}$ (Ordnung in $(L,+)$, neutr. Element ist $0$, statt $x^n$ steht $nx$)\\
Beh.: $p \in \MdP$\\
Ann.: $p=uv$ zerlegbar, $1\leq u < p$, $1\leq u < p$, $uv \cdot 1 = (u\cdot 1)(v\cdot 1) = 0$ \\ $\Rightarrow u\cdot 1=0$ oder $v \cdot 1 = 0$, Widerspruch.
$\Rightarrow L$ enthält $\MdF_p$, wenn man $\MdF_p \cong $ Versys$_p = \left\{0,\dotsc,p-1\right\} \ni z$ nimmt und $z \cdot 1$ mit $\overline{z}$ identifiziert (inj. Ringhomomorphismus $\MdF_p \to L$, $\overline{z} \mapsto z \cdot 1$) \\
Außerdem ist $L$ ein $\MdF$-Vektorrraum, wenn die Skalarmultiplikation so erklärt wird: \\
$\alpha \in L,\overline{z} \in \MdF_p:\overline{z}\alpha = (z\cdot 1) \cdot \alpha$ (VR-Axiome leicht nachprüfbar!)\\
$\#L=q < \infty \Rightarrow n :=$ dim $L < \infty$.\\
LA I: Basiswechsel liefert einen VR-Isomorphismus $L \to \MdF_p^n$ \\
$\Rightarrow q = \#L = \#\MdF_p^n = p^n$

\begin{enumerate}
\item Gesucht zu $n \in \MdN_+,p \in \MdP$ ein Körper mit $q=p^n$ Elementen.
\item Wie eindeutig ist $L$. (Wunsch: Je zwei solche L's sind isomorph)
\end{enumerate}
\begin{tabbing}
\underline{Idee:} \= "Kleiner Fermat" gilt in L, d.h. $\forall \alpha \in L: \alpha^q = \alpha$ \\
\> $\Rightarrow L$ besteht aus allen Nullstellen $\alpha$ von $X^q-X$ \\
\> $\Rightarrow X^q-X=\prod_{\alpha \in L}(X-\alpha)$ \\
\> Suche "große" Körper $K \supset \MdF_p$, so dass $X^q-X$ so zerfällt! \\
\> \underline{Hoffnung}: Die Nullstellen $\alpha$ von $X^q-X$ bilden dann den gesuchten Körper.
\end{tabbing}

Durchführung der Idee: Kette von Hilfssätzen
\begin{hilfssatz}[1]
Ist $R$ ein Ring der $\MdF_p$ als Teilring enthält, so gilt $\forall \alpha,\beta \in R,n\in \MdN_+,a=p^n$
$$(\alpha \pm \beta)^a = \alpha^a \pm \beta^a$$
\end{hilfssatz}
\begin{beweis}
In $\MdZ$ gilt für $1 \leq i \leq p: (1 \cdot 2 \cdot \dotsc \cdot i){p \choose i} = p \cdot (p-1) \cdot \dotsc \cdot (p-i+1)$
In $\MdF_p$ gilt für $1 \leq i \leq p: \underbrace{(\overline 1 \cdot \overline 2 \cdot \dotsc \cdot \overline i)}_{\in \MdF_p^x}\overline{{p \choose i}} = \overline 0 \dotsc = \overline 0$ \\
$\Rightarrow \overline{{p \choose i}} = \overline 0$\\
$\Rightarrow (\alpha + \beta)^p = \alpha^p + \beta^p + \sum_{i=1}^{p-1}{p \choose i} \alpha^i\beta^{p-i} = \alpha^p + \beta^p$, ok für n = 1 ($-$ ähnlich)\\
Rest Induktion, sei $j > 1$\\
$(\alpha + \beta)^{p^j} = (\alpha + \beta)^{p^{j-1} \cdot p} = (\alpha^{p^{j-1}} + \beta^{p^{j-1}})^p = \alpha^{p^{j-1} \cdot p} + \beta^{p^{j-1} \cdot p} = \alpha^{p^j} + \beta^{p^j}$
\end{beweis}

\begin{hilfssatz}[2]
Sei $K$ ein Körper, der $\MdF_p$ als Teilkörper enthält, so dass $(q=p^n, n\in\MdN_+)$
$$X^q-X = \prod_{j=0}^{q-1}(X-\alpha_j) \mbox{ mit } \alpha_0,\dotsc,\alpha_{q-1} \in K$$
Dann ist $L:= \left\{\alpha_0,\dotsc,\alpha_{q-1}\right\}$ ein Körper mit $q$ Elementen.
\end{hilfssatz}
\begin{beweis}
$K \ni \alpha$ Nullstelle von $X^q-X \Leftrightarrow \alpha^q-\alpha=0 \Leftrightarrow \alpha^q=\alpha$ \\
$\alpha \in L \Leftrightarrow \alpha^q = \alpha$ \\
Prüfe nach: $(L,+)$ ist Untergruppe von $(K,+)$, $(L^x=L\setminus 0,\cdot)$ ist Untergruppe von $(K^x,\cdot)$ $\Leftrightarrow$ Teilkörper, $\MdF_p \subseteq L$ wegen $\alpha^p = \alpha = \alpha^q$ für $\alpha \in \MdF_p$\\
$0 \in L \neq \emptyset$ \\
$\alpha,\beta \in L \Rightarrow \alpha^q = \alpha, \beta^q = \beta \Rightarrow (\alpha-\beta)^q = \alpha^q - \beta^q \mbox{ (HS1) } = \alpha - \beta \Rightarrow \alpha - \beta \in L$ also $L$ Untergruppe von $K$.\\
Analog $L^x$ $\alpha,\beta \in L^x \Rightarrow \alpha^q = \alpha, \beta^q=\beta \Rightarrow \alpha^q(\beta^q)^{-1} = \alpha\beta^{-1} \Rightarrow \alpha\beta^{-1} \in L^x$, also $L^x$ Untergruppe von $K^x$. \\
Wieso $\#L=q$? Wieso hat $X^q-X$ in $K$ nur einfache Nullstellen? \\
$\alpha \in L$, Wende HS1 an auf $K[X]$\\
$X^q-X=(X-\alpha)^q=X^q-\alpha^q-(X-\alpha) \Rightarrow 0 = (X-\alpha)^q - (X-\alpha) = (X-\alpha)\left((X-\alpha)^{q-1}-1\right)$, $\alpha$ ist nicht Nullstelle von $(X-\alpha)^{q-1}-1$ \\
Die NST ist einfach, Hinweis: $L = \left\{\zeta-\alpha|\zeta \in L\right\}$
\end{beweis}

Existenz von $L$: Suche $K \supseteq \MdF_p$ (Körper), so dass $K$ $q$ NST von $X^q-X$ enthält.

\begin{hilfssatz}[3]
Ist $K$ ein Körper, $f \in K[X]$, Grad $f > 0$, $K \supseteq \MdF_p$ (als Teilkörper), so gibt es einen endl. Körper $\tilde K$, der $K$ (und damit $\MdF_p$) als Teilkörper enthält und ein $\alpha \in \tilde K$ mit $f(\alpha)=0$
\end{hilfssatz}
\begin{beweis}
Primzerlegung von $f$, sei $f=g_1^{m_1} \cdot \dotsc \cdot g_t^{m_t}$, $g_j$ irred. in $K[X]$ (EuFa-Satz) \\
$f(\alpha) = 0 \Rightarrow 0 = g_1(\alpha)^{m_1} \cdot \dotsc g_t(\alpha)^{m_t} \Rightarrow \exists j: g_j(\alpha) = 0$ \\
So ein $\alpha$ ist gesucht! (und $\tilde K$)\\
$\tilde K := K[X]/g_jK[X]$ ist ein Körper, der $K$ als Teilkörper enthält. \\
$\alpha = \overline X$ ist NST von $g_j$, also $f$! $g_j(\overline X) = \overline{g_j(X)}=\overline 0 = 0$
\end{beweis}
\begin{hilfssatz}[4]
Es gibt einen endl. Körper $K$, in dem $f \in \MdF_p[X]$ (Grad $f > 0$, $f$ normiert) in Linearfaktoren zerfällt, d.h.
$$f=\prod_{j=1}^m(X-\alpha_j)\quad(\alpha_1,\dotsc,\alpha_m \in K)$$
\end{hilfssatz}
\begin{beweis}
Induktion nach $m =$ Grad $f$, $m=1$, $f=X-\alpha$, $\alpha \in \MdF_p$ 
\begin{tabbing}
$m>1$ \= $\tilde \MdF_p$ nach HS3 mit $\alpha \in \tilde \MdF_p$, $f(\alpha)=0$ \\
\> $\Rightarrow X-\alpha | f$ in $\tilde \MdF_p[X]$ \\
\> $\Rightarrow f = (X-\alpha)\tilde f$, Grad $\tilde f=$ Grad $f$ $-1$ \\
\> IH für $\tilde f$ $\Rightarrow$ Beh.
\end{tabbing}
\end{beweis}

\begin{hilfssatz}[5]
Sei $M$ ein Körper mit $p^n$ Elementen, $R=\MdF_p[X]$, $\xi\in M$, $g\in R$ mit $g(\xi)=0$ und $g$ irreduzibel.

Ist dann entweder $\grad g= n$ oder $\xi$ ein primitives Element von $M$, so seind die Körper $M$ und $R/gR = \overline R$ isomorph. Ein irreduzibles Polynom, das $\xi$ als Nullstelle hat, hat den Grad $n$.
\end{hilfssatz}
\begin{beweis}
$\psi:\overline R \to M, \overline h \mapsto h(\xi)=\psi(\overline h)$ ist der gesuchte Isomorphismus.
\begin{enumerate}
\item $\psi$ ist wohldefiniert:
\begin{align*}
\overline{h_1} = \overline{h_2} &\iff h_1 \equiv h_2 \mod g \\
&\iff \exists u\in R: h_2 = h_1 + ug\\
&\folgt h_2(\xi) = h_1(\xi) + u(\xi)\cdot g(\xi) = h_1(\xi)
\end{align*}
\item $\psi$ ist ein Ringisomorphismus, also $\psi(\overline {h_1} + \overline{h_2}) = \psi(\overline{h_1}) + \psi(\overline{h_2})$:

Klar wegen $(h_1 \pm h2)(\xi) = h_1(\xi) \pm h_2(\xi)$
\item $\psi$ ist injektiv:

Es genüg zu zeigen: $\kernn \psi = \{0\}$.

Ann: $\alpha \in \kernn \psi$, $\alpha \ne 0$. $1 = \psi(1) = \psi(\alpha^{-1} \alpha) = \psi(\alpha^{-1})\psi(\alpha) = 0$, Wid!
\item $\psi$ ist surjektiv:
\begin{enumerate}
\item[a)] $\grad g = n \folgt \#\overline R = p^n$, $\psi:M\to\overline R$ injektiv. Da $\#M = p^n \folgt \psi$ surjektiv.
\item[b)] $\xi$ primitiv $\iff M=\{0,\xi,\xi^2,\ldots,\xi^{q-2}\}$. $\psi(\overline R) \ni h(\xi)$ für z.B. $h=X^n$ $(n\in\MdN)$ $\folgt \psi(\overline R) \ni X^n(\xi) = \xi ^n \folgt \psi(\overline R) \supseteq M \folgt \psi$ surjektiv.
\end{enumerate}
\end{enumerate}
\end{beweis}

\begin{satz}[Endliche-Körper-Raum]
\begin{enumerate}
\item Ist $L$ ein endlicher Körper, $\#L=q$, dann $\exists p \in \MdP$, $n\in\MdNp$ mit $q=p^n$. (Genauer: Dann ist $\MdF_p$ ein Teilkörper von $L$ und $K$ ein $\MdF_p$-Vektorraum der Dimension $n$).
\item Zu jedem $n\in\MdNp$, $p\in\MdP$, existiert ein Körper mit $q=p^n$ Elementen.Zusätzlich gilt: Es gibt ein irreduzibles Polynom $g=\MdF_p[X]$ mit $\grad g=n$. Es ist $g \mid X^q -X$.
\item Je zwei Körper mit $q$ Elementen sind isomorph.
\end{enumerate}
\end{satz}

Also ist es gerechtfertigt, von \emph{dem} Körper $\MdF_q$ oder $GF(q)$ zu sprechen.

\begin{beweis}
\begin{enumerate}
\item Wurde bereits geleistet. (\emph{Aber wo?})
\item Erinnerung: Es gibt einen Körper $K$, der $\MdF_p$ enthält, so dass $X^q-X=\prod_{j=0}^{q-1}(X-\alpha_j)$, $(\alpha_j\in K)$, $L=\{\alpha_j\mid j=0,\ldots,q-1\}$ ist Körper mit $q$ Elementen.
\item $M$, $L$ seien Körper mti $q=p^n$ Elementen. $\xi$ sei ein primitives Element von $M$ (Existenz: Satz vom primitiven Element). $X^q-X = \prod_{\alpha \in L}(X-\alpha)$. Betrachte die Primzerlegung $X^q-X = \prod_{j=1}^tp_j^{n_j}$ in $\MdF_p[X]$, $p_j$ irreduzibel in $R$, die es nach dem EuFa-Satz gibt.

Wegen $(X^q-X)(\xi) = 0 = \prod_{j=1}^t p_j(\xi)^{n_j}$ existiert ein $j\in\{1,\ldots,t\}$, $p_j(\xi)=0$, $p_j=g$ irreduzibel in $\MdF_p[X]$. Hilfssatz 5 liefert: $M\cong R/gR$ und $\grad g = n$ (wo $\#M=p^n$). Wir folgern also: Jedes $p_j$ (also auch $g$) ist Produkt gewisser $(X-\alpha)$ (EuFa-Satz für $L[X]$) $\folgt \exists \alpha \in L: X-\alpha \mid g \folgt g(\alpha) = 0$. Wir benutzen nun den Hilfssatz für $L$ statt $M$ und erhalten: $\overline R = R/gR \cong L$. Damit erhalten wir: $L\cong M$.
\end{enumerate}
\end{beweis}

\begin{satz}[Teilkörpersatz]
\begin{enumerate}
\item Sei $K$ ein Teilkörper von $\MdF_q$ mit $q=p^n$ wie oben. Dann existiert ein $d\in\MdN$ mit $d\mid n$ und $K\cong \MdF_{p^d}$.
\item Ist $d\mid n$, so gibt es genau einen Teilkörper von $\MdF_q$ mit $\#K=p^d$
\end{enumerate}

Fazit: Teilkörper enpsrechen bijektiv den Teilern $d$ von $n$.
\end{satz}

% \begin{center}
% \begin{tikzpicture}
% \node (50) at (1,3) {50} ;
% \node (5) at (1,1) {5};
% \node (1) at (0,0) {1};
% \node (2) at (-1,1) {2};
% \node (10) at (0,2) {10};
% \node (25)at (2,2) {25};
% \draw[->] (1) -- (2);
% \draw[->] (2) -- (10);
% \draw[->] (1) -- (5);
% \draw[->] (5) -- (10);
% \draw[->] (10) -- (50);
% \draw[->] (25) -- (50);
% \draw[->] (5) -- (25);
% \draw[->] (-1,2) -- node[left]{"`teilt"'} (-.5,2.5) ;
% \end{tikzpicture}
% \begin{tikzpicture}
% \node (50) at (1,3) {$\MdF_{p^{50}}$} ;
% \node (5) at (1,1) {$\MdF_{p^5}$};
% \node (1) at (0,0) {$\MdF_p$};
% \node (2) at (-1,1) {$\MdF_{p^2}$};
% \node (10) at (0,2) {$\MdF_{p^{10}}$};
% \node (25)at (2,2) {$\MdF_{p^{25}}$};
% \draw[->] (1) -- (2);
% \draw[->] (2) -- (10);
% \draw[->] (1) -- (5);
% \draw[->] (5) -- (10);
% \draw[->] (10) -- (50);
% \draw[->] (25) -- (50);
% \draw[->] (5) -- (25);
% \draw[->] (-1,2) -- node[left]{"`Teilkörper von"'} (-.5,2.5) ;
% \end{tikzpicture}
% \end{center}

\begin{beweis}
\begin{bemerkung}
Ist $K$ ein Teilkörper von $L$, so ist $L$ ein $K$-Vektorraum (Skalare Multiplikation ist die von $L$). 
\end{bemerkung}
Also ist $\MdF_q$ ein $K$-Vektorraum $\folgt$ (Basiswahl) $\MdF_q \cong K^{d'}$; $d'$ ist die Dimension des $K$-Vektorraums $\MdF_q = q^n= q=\#K_q=(p^d)^{d'}$, (da $\#K=p^d$) $\folgt n= dd' \folgt d\mid n$. 

Ist $\#K=p^d$, $d\mid n$, $K$ Teilkörper von $\MdF_q$, so muss $K$ aus den Nullstellen von $X^{p^d}-X$ in $\MdF_p$ bestehen, also ist $K$ eindeutig bestimmt. ($K=\{\alpha^{p^{\frac n d}}\mid \alpha \MdF_p\}$).
\end{beweis}

\section{Die Sätze von Chevalley und Warming}

Es sei generell hier $K=\MdF_q$, $q=p^n$ wie oben, mit dem wichtigsten Fall $n=1$, $K=\MdF_p$.

Das Problem ist: $f\in K[X_1,\ldots,X_n]$ liege vor mit $f(\underline 0) = 0$, $\underline0 = (0,\ldots,0)\in K^n$. Gesucht: Möglichst gute Bedingungen, so dass $f$ eine nicht-triviale Nullstelle $\underline x = (\alpha_1,\ldots,\alpha_n) \in K^n$ besitzt. (nicht-trivial: $\underline x \ne \underline 0)$.

Bezeichnungen:
\begin{enumerate}
\item $f = \sum_{\underline m\in\MdN^n} \alpha_{\underline m}X ^{\underline m}$, wobei $\underline m = (m_1, \ldots, m_n)$, $\underline 0 = (0,\ldots,0)$, $\alpha_{\underline m} \in K$, davon nur endlich viele $\ne 0$.
\item $X^{\underline m} := X_1^{m_1}\cdots X_n^{m_n}$
\item Setze $|\underline m| = m_1 + \cdots + m_n$. Damit ist der Gesamtgrad $\grad f$ wie folgt definiert: $\grad 0 = -\infty$, $f\ne 0$: $\grad f = \max\{|m| \mid \alpha_{m} \ne 0\}$.
\end{enumerate}

\begin{satz}[von Warming]
Sei $f\in\MdF_q[X_1,\ldots,X_n]$, $\grad f < n$. Dann ist die Anzahl der Nullstellen von $f$ in $\MdF_q^n$ durch $p$ teilbar.

Dabei heißt $\mathcal{V}_f(k) := \{\underline x\in K^n \mid f(\underline x) = 0\}$ die Nullstellenmannigfaltigkeit von $f$ in $K$.

Allgemeiner: $f_1,\ldots,f_l \in K[X_1,\ldots,K_n]$: $\mathcal{V}_{f_1,\ldots,f_l}(K)=\{\underline x \in K^n \mid f_1(\underline x)= \cdots =  f_l(\underline x) = 0\}  = \bigcap_{i=1}^l \mathcal{V}_{f_j}(K)$

Die Aussage des Satzen ist nun: Ist $\grad f < n$, so gilt $p \mid \#\mathcal{V}_f(K)$
\end{satz}

\begin{satz}[Satz von Chevalley]
Sei $f\in K[X_1,\ldots,X_n]$, $f(\underline 0) = 0$ und $\grad f < n$. Dann hat $f$ ein nichttriviale Nullstelle.
\end{satz}

Es ist klar: Satz von Warming impliziert den Satz von Chevalley, da: $f(\underline 0) = 0 \folgt \underline 0 \in \mathcal{V}_f(K) \folgt \#\mathcal{V}_f(K)>0$. $p\mid \mathcal{V}_f(K) \folgt \#\mathcal{V}_f(K) \ge p \ge 2$

Spezielles Beispiel:
\begin{satz}
Seien $\alpha_1,\ldots,\alpha_{n+1} \in \MdZ$, $d\le n$, $d\in\MdN$. Dann hat die Kongruenz $\alpha_1 x_1^d + \cdots + \alpha_{n+1}x_{n+1}^d \equiv 0 \mod p$ stets eine nicht-triviale Lösung $x=(x_1,\ldots,x_n)\in\MdZ^{n+1}$
\end{satz}

Noch spezieller: $\alpha_1x_1^2 + \alpha_2x_2^2 + \alpha_3x_3^2\equiv 0$ hat stets nicht-triviale Lösung $(x_1,x_2,x_3 \in \MdZ)$.

\begin{beweis}
$\grad \alpha_1xX^d + \cdots \alpha_{n+1}X_{n+1}^d \le d \le n+1$ (Variablenzahl). Satz von Chevalley liefert die Behauptung.
\end{beweis}

Gegenbeispiel: $x_1^2 + x_2^2 \equiv 0 \mod 3$: $x_j^2 \in \{0,1\} \folgt$ Jede Lösung hat $3\mid x_1$ und $3\mid x_2$

Weitere Sätze (siehe z.B. Lidl/Niederreiter, Finite Fields):
\begin{satz}[Satz I]
Sei $d = \grad f_1+ \cdots + \grad f_l < n$ und $f_j\in\MdF_q[X_1,\ldots,X_n]$. Falls $\mathcal{V}_{f_1,\ldots,f_l}(\MdF_q) \ne \emptyset$, so gilt: $\#\mathcal{V}_{f_1,\ldots,f_l}(\MdF_q) \ge w^{n-d}$
\end{satz}

\begin{satz}[Satz II]
Falls $f\in\MdF_1[X_1,\ldots,X_n]$, $0< \grad f=d$, so gilt: $\#\mathcal{V}_{f}(\MdF_q) \le d\cdot 1^{n-1}$
\end{satz}

\begin{satz}
Sei $0 \ne f\in \MdZ[X_1,\ldots,X_n]$. Dann gibt e es eine konstante $c_f$ unabhängig von $p$, so dass 
\[ \forall p\in\MdP: |\#\mathcal{V}_{f}(\MdF_q) - p^{n-1} | \le c_f \frac{p^{n-1}}{\sqrt p} \]
\end{satz}
Der Beweis ist äußerst schwierig, bereits für $n=2$.

%%%
%%%  Vorlesung 22.06. LaTeXer [Stephan]
%%%

\begin{beweis}
Der Beweis des Satzes von Warming \ref{satz:Warming} gliedert sich
in mehrere Ideen, wie bringen sie hier schön isoliert. In vielen
Büchern ist der Beweis ziemlich unübersichtlich.
\begin{description}
    \item{Idee 1:} Das Kronecker-$\delta$ ist als Polynom darstellbar.

        \begin{lemma}\label{lemma:BeweisWarmingLemma1}
            $\delta:K\to K$ sei definiert wie folgt:
            \[\delta(\alpha)=\delta_0(\alpha)=\begin{cases}1,&\alpha=0\\0,&\text{sonst}\end{cases}\]
            Dann
            $\delta(\alpha)=1-\alpha^{q-1}=(1-X^{q-1})(\alpha)$,
            weil $\alpha^{q-1}=1$, wenn $\alpha \in
            K^{\times}=\MdF_q^{\times}$ und $\alpha^{q-1}=0$, wenn
            $\alpha=0$.
        \end{lemma}
        \begin{satz}
            Jede Funktion $\MdF_1 \to \MdF_1$ ist als Polynom
            darstellbar.
        \end{satz}
        \begin{beweis} Übung. \end{beweis}
    \item{Idee 2:} Aus $f$ kann man eine Funktion $F$ konstruieren, so
    dass $F$ die Nullstellen von $f$ zählen hilft.\\
    $F=A-f^{q-1}$. Dann
    \[F(x)=1-f(x)^{q-1}=\delta_{0,\,f(x)}=\begin{cases}1,&x\in
    V_f(K)\\0,&\text{sonst}\end{cases}\]
    Es folgt die Formel $\sum_{x\in K^n}=\# V_f(K) \cdot 1_K$.
    \item{Idee 3:} Versuche die linke Seite der Formel zu
    berechnen, nämlich $\sum_{x\in K^n}g(x)$, $
    {g\in K[X_1,X_2,\dotsc,X_n]}$. Beginne mit $n=1,\ g=X^k$.
    $\sum_{\alpha \in K}\alpha^k=?$.
    \begin{lemma}\label{lemma:BeweisWarmingLemma2}
        Ist $k\in \MdN$ und $k=0$ oder $q-a \nmid k$, so ist $\sum_{\alpha \in K}
        \alpha^k=0$ (Dabei muss $0^0=1$ definiert werden).
    \end{lemma}
    \begin{beweis}
        $k=0$: $\sum_{\alpha\in K}\alpha^0=\sum_{\alpha\in K} 1= q
        \cdots 1_K=0$ und $1_K \mit q = p^n$.\\
        $k>0$: Dann existiert ein primitives Element $\xi \in K$,
        das heißt, $K^{\times}=K \setminus \{0\}=\left\{1,\xi,\xi^2,\dotsc,x^{q-2}
        \right\}$ und $\ord \xi=q-1$, daraus folgt $\xi^k \neq 1$
        (laut Elementarordnungssatz).\\
        \[
            \sum_{\alpha \in K} \alpha^k=\sum_{\alpha \in K
            \setminus \{0\}} \alpha^k=\sum_{j=0}^{q-2}
            \xi^{j-k}=\sum_{j=0}^{q-2} \left(
            \xi^k\right)^j=\frac{\xi^{k(q-1)}-1}{\xi^k-1}
            \text{ (geometrische Reihe!})
        \] (wegen $\xi^{q-1}=1)$.
    \end{beweis}
    \begin{lemma}\label{lemma:BeweisWarmingLemma3}
        Sei $g \in K[X_1,X_2,\dotsc,X_n]$, $\grad g < n(q-1)$, dann
        ist $\sum_{x\in K^n}g(x)=0$.
    \end{lemma}
    \begin{beweis}
        Ohne Beschränkung der Allgemeinheit ist $g=x^m$ mit
        $|m|<n(q-1),\ m\in K^n$, denn wenn $g=\sum\beta_m X^m$, dann
        $\forall m$ mit $\beta_m \neq 0:\ |m|<n(q-1)$, denn die
        Summe von Nullen ergibt null. Weiterhin gilt
        \[
            \sum_{x\in K^n}
            X^m(x)=\sum_{(\alpha_1,\alpha_2,\dotsc,\alpha_n)\in K^n}
            \alpha_1^{m_1} \cdot \alpha_2^{m_2} \cdot \dotsb  \cdot \alpha_n^{m_n}
        \](Durch Ausmultiplizieren erhält man \[ \prod_{j=1}^n
        \left(\sum_{\alpha_j \in K} \alpha_j^{m_j}\right)=\sum_{(\alpha_1,\alpha_2,\dotsc,\alpha_n)\in K^n}
            \alpha_1^{m_1} \cdot \alpha_2^{m_2} \cdot \dotsb  \cdot
            \alpha_n^{m_n}.\] (Kann man, wenn man Lust hat, mit Induktion
            beweisen))\\
            Voraussetzung: $m_1+m_2+\dotsb+m_n<n(q-1) \implies
            \exists j\in \{1,2,\dotsc,n\}$ mit $m_j<q-1 \implies
            m_j=0$ oder $q-1 \mid m_j$. Anwendung von Lemma
            \ref{lemma:BeweisWarmingLemma2} mit $k=m_j$
            \[
                \implies \sum_{\alpha_j \in K} \alpha_j^{m_j}=0
                \implies \prod \sum \alpha_j^{m_j}=0=\sum X^m(x).
            \]
    \end{beweis}
    Wende das Lemma \ref{lemma:BeweisWarmingLemma3} an auf
    $g=F=1-f^{q-1}$. $\grad g=(q-1) \underbrace{\grad f}_{<n}
    \implies {\grad g < (q-1) n}$, also kann letztes Lemma angewandt
    werden \[\implies \sum_{x\in K^n}F(x)=0=\#V_f(K) \cdots 1_k
    \implies p=\ord 1_K \mid \#V_f(K).\]
\end{description}
\end{beweis}

\chapter{Quadratische Kongruenzen}
\section{Einführende Diskussion}
\paragraph{Problem:} Gegeben $a,b,c \in \MdZ$. Wann ist die
quadratische Kongruenz $ax^2+bx+c \equiv 0 \mod m$ lösbar und wann
nicht? In diesem Rahmen wird nur der Fall $a=1$ behandelt (andere
Wahl von $a$ ergibt keine schönen Ergebnisse).

1. Gedanke: Mittels des Chinesischen Restsatzes reicht die
Betrachtung des Falls $m=p^t,\ p\in \MdP,\ t\in \MdN_+$ aus.\\
$p=2$: Explizite Aussage möglich (Übung). Hier betrachten wir nur
$p>2$. Dann gilt aber ohne Beschränkung der Allgemeinheit $2 \mid
b$, denn $\overline b=\overline 2 \underbrace{(\overline
2^{-1}b)}_{=: b_0} =2\overline b_0$.\\
\[
    x^2+2b_0x+c=\underbrace{(x+b_0)^2}_{=:x'}+\underbrace{c-b_0^2}_{=:-k}=x'-k
\]
Dann genügt zu zeigen: Wann ist $x^2\equiv k \mod p^t$ lösbar.
$k=p^{v_p(k)} k_0,\ p \nmid k_0$, falls $v_p(k)\geq t \implies$
lösbar mit $x=0$. Falls $v_p(k)=u < t$: Ansatz $x=p^{v_p(x)}x_0,\ p
\nmid x_0$, falls $x$ Lösung ist, dann gilt für ein $c\in \MdZ$:
\[
    p^{2v_p(x)} x_0^n=p^k k_0 + c
    p^t=p^k(\underbrace{k_0+cp^{t-u}}_{\not \equiv 0 \mod p}),\
    t-u>u \implies u=w v_p(x),
\]
also $2\mid u$ und $x_0 \equiv k_0 \mod p^{t-u}$ mit $p \nmid k_0$.
Die Umkehrung gilt auch. Ergebnis: Die Kongruenz $x^2 \equiv k \mod
p^t$ ist lösbar, wenn $v_p(k)\geq t$, wenn $v_p(k) < t$, so genau
dann lösbar, wenn $2 \mid v_p(k)$ und die Kongruenz $x_0^2 \equiv
k_0 \mod p^{t-u}$ lösbar ist. Hiernach genügt es, den Fall $x^2
\equiv k \mod p^t$ mit $p\nmid k$ zu behandeln, also $\overline k
\in G=(\MdZ/p^t\MdZ)^{\times}$.
\begin{hilfssatz}
    Sei $t \in \MdN_+,\ p \in \MdP,\ p>2,\ p \nmid k$. Dann gilt:
    \[
        x^2\equiv k \mod p^t \text{ lösbar} \iff x^2 \equiv k \mod p
        \text{ lösbar}.
    \]
\end{hilfssatz}
\begin{beweis}
    "`$\Longrightarrow$"' trivial\\
    "`$\Longleftarrow$"' Induktion nach $t$. $t=1$ ist klar. Sei also
    $t>1$ und $x_0 \in \MdZ$ mit $x_0^2 \equiv k \mod p^{t-1}$.
    Gesucht $x$, nötig $x\equiv x_0 \mod p^{t-1}$.\\
    Ansatz: $x=x_0+cp^{t-1},\ x_0^2=k+vp^{t-1}\ (c,v \in \MdZ)$.\\
    Idee: Bestimme $c$, so dass $x^2\equiv k \mod p^t$.
    \begin{align*}
        &x^2=\left(x_0+cp^{t-1}\right)^2&=k+vp^{t-1}+2x_0cp{t-1}+c^2+\underbrace{p^{2t-2}}_{\equiv 0 \mod p^t}\\
                                       &\stackrel{!}{\equiv} k \mod
                                       p^t\\
        \iff\,  v p^{t-1} &\equiv -2 x_0 c p^{t-1} \mod p^t\\
        \iff \qquad v &\equiv -2x_0c\mod p
    \end{align*}
    Klappt mit $\overline c=\overline v(\overline{-2x_0})^{-1}$ in
    $\MdF_p$, da $p \nmid x_0$ (wegen $x_0^2\equiv k \neq 0 \mod
    p$), $p \nmid 2 \implies \overline{-2 x_0} \in \MdF_p^{\times}$.
\end{beweis}
Resultat der Diskussion: Frage der Lösbarkeit von quadratischen
Kongruenzen lässt sich zurückführen auf die Frage, welche $k$ mit
$p\nmid k$ für prime $p$ größer zwei quadratische Reste sind oder
nicht. Erinnerung an Eulers Quadratkriterium!

\section{Grundaussagen über Potenzreste}
\paragraph{Bezeichnung}
\begin{enumerate}
    \item $(G,\cdot)$ abelsche Gruppe, $l\in\MdN_+:\ G^{(l)}:=\{x^l:\ x\in
    G\}$, $G^{(l)}$ ist Untergruppe von G (Ist mit
    Untergruppenkriterium schnell gezeigt).
    \item $k\in \MdZ$ heißt $l$-ter \emph{Potenzrest} $\mod m,\ m\in
    \MdN_+ \iff k\in((\MdZ / m \MdZ)^{\times})^{(l)}$ $\iff
    \ggt(m,k)=1$ und es existiert $x\in \MdZ$ mit $x^l \equiv k \mod
    m$.
\end{enumerate}
\begin{lemma}
    $(G,\cdot)$ abelsche Gruppe, $n=\#G<\infty$.\\
    $d:=\ggt(n,l)$. Dann ist $G^{(l)}=G^{(d)}$.
\end{lemma}
\begin{beweis}
    $x\in G,\ \underbrace{x^l}_{\in
    G^{(l)}}=\underbrace{x^{\frac{l}{d}d}}_{\in G^{(d)}}$, also ist
    $G^{(l)} \subset G^{(d)}$. Der LinKom-Satz
    \ref{satz:LinKom} liefert $d=un+vl$ mit $u,v\in\MdZ$.
    $\underbrace{x^d}_{\in G^{(d)}}=\underbrace{x^{nu}}_{=1
    \text{(EOS)}} x^{lv}=(x^v)^l \in G^{(l)}$, also ist
    $G^{(d)}\subset G^{(l)}$. Folglich sind beide Mengen gleich.
\end{beweis}
Nächste Frage: Was ist $\#\left( ((\MdZ/p^t\MdZ)^{\times})^{(d)}
\right)$?\\

% Mitschrieb 28.6.2006 Robert Geisberger
Klar: Falls $G=\langle\zeta\rangle=\left\{1,\zeta,\dotsc,\zeta^{m-1}\right\}$ dann $d=\ggt(k,m)$ \\
 $G^{(k)}=G^{(d)} = \left\{1,\zeta^d,\zeta^{2d},\dotsc,\zeta^{\left(\frac{m}{d}-1\right)d}\right\}$\\
$\implies \#G^{(k)}=\#G^{(d)}=\frac{m}{d}$\\
Ergebnis also\\
\begin{satz}[Potenzrestklassenanzahlsatz]
\begin{enumerate}
\item[(i)] Sei $p \in \MdP$, $p>2$, $k,t\in \MdN_+$. Dann gilt
\[
    \#\left(\left(\MdZ/p^t\MdZ\right)^{\times}\right)^{(k)} =
    \frac{\varphi(p^t)}{\mbox{ggT}(\varphi(p^t),k)}
\]
 (In Worten: Es
gibt genau $\frac{\varphi(p^t)}{\mbox{ggT}(\varphi(p^t),k)}$ $k$-te
Potenzrestklassen.
\item[(ii)] Für $2\nmid k$ ist $\left(\left(\MdZ/2^t\MdZ\right)^{\times}\right)^{(k)}=\left(\MdZ/2^t\MdZ\right)^{\times}$. \\
Für $t>2$ und $2\mid k$ ist
$\left(\left(\MdZ/2^t\MdZ\right)^{\times}\right)^{(k)}$ zyklisch und
hat $\frac{2^{t-2}}{\ggt(2^{t-1},k)}$ Elemente.
\item[(iii)] (Potenzrestkriterium a la Euler) \\
Sei $p \in \MdP$, $p>2$, $t,k\in\MdN_+$, $d=$ggT$(\varphi(p^t),k)$\\
$r$ ist $k$-ter Potenzrest mod $p^t$ $\iff$
$r^{\frac{\varphi(p^t)}{d}} \equiv 1$ mod $p^t$.
\end{enumerate}
\end{satz}
\begin{beweis}
Beweise (iii) wie Eulerkriterium, benutze primitives Element!\\
\end{beweis}
\begin{tabular}{ll}
\underline{Folge:} & $p\in \MdP$, $p>2$ $\implies$ Es gibt genau $\frac{p-1}{2}$ quadratische Reste und $\frac{p-1}{2}$ quadratische Nichtreste. \\
\underline{Grund:} & (i) mit $k=d=2$, $t=1$, $\varphi(p)=p-1$ \\
\underline{Bsp:} & $p=11$ \\
 &
\begin{tabular}{l|rrrrr|l}
$x$ & $\pm 1$ & $\pm 2$ & $\pm 3$ & $\pm 4$ & $\pm 5$ & \\
$x^2$ mod $11$ & 1 & 4 & 9 & 5 & 3 & $\leftarrow$ quadratische Reste
\end{tabular}\\
& $\left\{2,6,7,8,10\right\}$ $\leftarrow$ quadratische Nichtreste
\end{tabular}

\section{Quadratische Reste und das quadratische Reziprozitätsgesetz}
$p\in\MdP$, $p>2$
\begin{definition}
\begin{enumerate}
\item \begin{align*}
    k \text{ quadratischer Rest } \mod p &      \iff \overline k \in \left(\left(\MdF_p\right)^{\times}\right)^{(2)} \\
    k \text{ quadratischer Nichtrest } \mod p & \iff \overline k \in \MdF_p^x \setminus \left(\left(\MdF_p\right)^{\times}\right)^{(2)} \\
    \end{align*}
\item Die Frage der Lösbarkeit quadratischer Kongruenzen lässt sich zurückführen auf die Frage, ob $k$ quadratischer Rest ist oder nicht ($\mod p$).
\end{enumerate}
\end{definition}

\begin{definition}
Sei $p \in \MdP$, $p>2$, $u \in \MdZ$, so sei
\[
\left(\frac{u}{p}\right) = \begin{cases}
    1  & u \text{ quadratischer Rest} \mod p\\
    -1 & u \text{ quadratischer Nichtrest} \mod p\\
    0  & \text{sonst, d.\,h. }p \mid u
    \end{cases}
\]

$\left(\frac{u}{p}\right)$ heißt \emph{Legendre-Symbol}.
\end{definition}

\begin{satz}[Legendre-Symbol-Satz]
Sei $a,b\in\MdZ$, $p \in \MdP$, $p>2$, dann gelten
\begin{enumerate}
\item[(i)] $a \equiv b$ mod $p$ $\implies$ $\left(\frac{a}{p}\right) = \left(\frac{b}{p}\right)$, und $\left(\frac{a}{p}\right) \in \{0,\pm 1\}$
\item[(ii)] $\left(\frac{ab}{p}\right) = \left(\frac{a}{p}\right)\left(\frac{b}{p}\right)$, insbesondere hat man den Gruppenhomomorphismus
\[
 \chi_p : \MdF_p^\times \to \MdC^\times,\ \chi_p(\overline a) =
\left(\frac{a}{p}\right) =: \left(\frac{\overline a}{p}\right)
\]
(Homomorphismen $G \to \MdC^\times$, $G$ abelsche Gruppe, heißen
traditionell \underline{Charaktere} der Gruppe $G$, $\chi_p$ heißt
Dirichlet-Charakter)
\item[(iii)] $\left(\frac{ab^2}{p}\right)=\left(\frac{a}{p}\right)$ falls $p \nmid b$.
\item[(iv)] $\left(\frac{a}{p}\right) \equiv a^{\frac{p-1}{2}}$ mod $p$.
\end{enumerate}
\end{satz}
\begin{beweis}
\begin{enumerate}
\item[(i)] Definition.
\item[(iv)]
Eulerkriterium:
    \begin{align*} a \text{ quadratischer Rest} & \iff \overline a ^{\frac{p-1}{2}} = 1 \text{ in } \MdF_p \\
                a \text{ quadratischer Nichtrest} & \iff  \overline a ^{\frac{p-1}{2}} = -1 \text{ in } \MdF_p
    \end{align*}
    $p \mid a$ $\iff$ $p \mid a^\frac{p-1}{2}$
\item[(ii)] $\left(\frac{ab}{p}\right) \stackrel{\mbox{(iv)}}{\equiv}(ab)^\frac{p-1}{2}\equiv a^\frac{p-1}{2} b^\frac{p-1}{2} \equiv \left(\frac{a}{p}\right)\left(\frac{b}{p}\right)$. Wegen $-\frac{p}{2}<\left(\frac{a}{p}\right)<\frac{p}{2}$ $\implies$ $\left(\frac{ab}{p}\right)=\left(\frac{a}{p}\right)\left(\frac{b}{p}\right)$
\item[(iii)] $\left(\frac{ab^2}{p}\right)\stackrel{\mbox{(ii)}}{=}\left(\frac{a}{p}\right)\left(\frac{b^2}{p}\right)=\left(\frac{a}{p}\right)\underbrace{\left(\frac{b}{p}\right)^2}_{=1}=\left(\frac{a}{p}\right)$
\end{enumerate}
\end{beweis}
Satz gibt Algorithmus zur Berechnung von $\left(\frac{a}{p}\right)$.

Skizze:\\
\begin{enumerate}
\item $\left(\frac{a}{p}\right)=\left(\frac{a\mbox{ mods }p}{p}\right) = \left(\frac{r}{p}\right) = \left(\frac{\mbox{sgn}(r)}{p}\right)\left(\frac{|r|}{p}\right)$
\item Primzerlegung von $|r|=p_1^{n_1}\cdot \dotsc \cdot p_t^{n_t}$\\
$\left(\frac{2}{p}\right)$ elementar "`Ergänzungssatz"'\\
$\left(\frac{q}{p}\right)$ $q \in \MdP$, $q>2$, $q \neq p$ geht
zurück auf $\left(\frac{p}{q}\right)$ mittels des quadratischen
Reziprozitätsgesetzes.
\end{enumerate}
Nämlich:
\paragraph{Legendre:} Experimente zeigen unerwartete und "`unerklärliche"'
Zusammenhänge zwischen $\left(\frac{p}{q}\right)$ und
$\left(\frac{q}{p}\right)$. Zum Beispiel
$\left(\frac{p}{5}\right)=\left(\frac{5}{p}\right)$ ($\star$)
oder $\left(\frac{p}{7}\right)=-\left(\frac{7}{p}\right)$ und Ähnliche. \\
($\star$) Beweisversuch: Wenn $x \in \MdZ$ mit $x^2 \equiv 5$ mod $p$ ($p \mid x^2-5$) so \\
konstruiere $y \in \MdZ$, $y=y(x,5,p)$ mit $y^2 \equiv p$ mod $5$ ($5 \mid y^2-p$).\\
Bis heute eine Formel für so ein $y$ unbekannt!

Der folgende Satz ist der berühmteste Satz der Elementaren
Zahlentheorie.

\begin{satz}[Quadratisches Reziprozitätsgesetz von Gauß]
\begin{enumerate}
\item[(i)] Es seinen $p,q\in\MdP$, $p>2$, $q>2$, $p \neq q$. Dann gilt
$$\left(\frac{p}{q}\right)\left(\frac{q}{p}\right)=(-1)^{\frac{p-1}{2}\frac{q-1}{2}}$$
\item[(ii)] \begin{tabbing}
    "`Ergänzungssätze"' \= $\left(\frac{-1}{p}\right) = (-1)^\frac{p-1}{2} = \left\{\mbox{\begin{tabular}{ll}
    $1$ & $p \equiv 1$ mod $4$\\
    $-1$ & $p \equiv -1$ mod $4$
  \end{tabular}}\right.$ \\
  \> $\left(\frac{2}{p}\right) = (-1)^\frac{p^2-1}{8} = \left\{\mbox{\begin{tabular}{ll}
    $1$ & $p \equiv \pm 1$ mod $8$\\
    $-1$ & $p \equiv \pm 3$ mod $8$
  \end{tabular}}\right.$
\end{tabbing}
\end{enumerate}
\end{satz}
Gauß gab 7 wesentlich verschiedene Beweise, heute 200 bekannt. Kein "`Eselsbeweis"' dabei. Heute befriedigender Beweis via "`Artins"' Reziprozitätsgesetz.\\
\\
Artins Hauptsatz der sog. "`Klassenkörpertheorie"' stellt eine Isomorphie her zwischen den Automorphismusgruppen ("`Galoisgruppen"'), sog. abelschen Zahlkörper und sog. Strahlklassengruppen (verallg. Restklassengruppen).\\
\\
\begin{beweis}
Hier: Raffinierter Beweis mit endlichen Körpern\\
In $L = \MdF_{p^{q-1}}$ existiert $\omega \in L^\times$ mit $\ord(\omega)=q$\\
Dann ist für $\alpha \in \overline a$ in $\MdF_q$ \underline{wohldefiniert} $\omega^\alpha:=\omega^a$ (Elementordnungssatz) \\
Fasse $\left(\frac{a}{q}\right) =: \left(\frac{\alpha}{q}\right)$ als Element von $L$ auf $\left(=\begin{array}{l} 0_L \\ \pm 1_L \end{array}\right)$ \\
Bezeichung $\tau := \sum_{\alpha \in \MdF_q}\left(\frac{\alpha}{q}\right) \cdot \omega^\alpha$ $(\in L)$ heißt Gaußsche Summe.\\
$\left[ \mbox{Gauß benutzte statt $\omega$ $\zeta=e^\frac{2\pi i}{q} \in \MdC$ (ord $\zeta=q$ in $\MdC^\times$)} \right]$ \\
\begin{tabbing}
Formeln a la Gauß \= $\tau^2=q \cdot \left(\frac{-1}{q}\right) \cdot 1_L$ \= (a) \\
                  \> $\tau^{p-1} = \left(\frac{p}{q}\right) \cdot 1_L$    \> (b)
\end{tabbing}
Aus diesen Formeln ergibt sich das Gesetzt mit dem Eulerkriterium \\
$\left(\frac{q}{p}\right) \equiv q^\frac{p-1}{2}$ mod $p$ (also $\left(\frac{q}{p}\right) \cdot 1_L = q^\frac{p-1}{2} \cdot 1_L$)\\
\begin{tabbing}
$\left(\frac{q}{p}\right) \cdot 1_L$ \= $= (q \cdot 1_L)^\frac{p-1}{2}$ \\
                                     \> $\stackrel{\mbox{(a)}}{=} \left(\left(\frac{-1}{q}\right)\tau^2\right)^\frac{p-1}{2} = \left(\frac{-1}{q}\right)^\frac{p-1}{2}\tau^{p-1} \stackrel{\mbox{(ii)}}{=} (-1)^{\frac{q-1}{2} \cdot \frac{p-1}{2}} \cdot \tau^{p-1}$ \\
                                     \> $\stackrel{\mbox{(b)}}{=} (-1)^{\frac{q-1}{2} \cdot \frac{p-1}{2}}\cdot \left(\frac{p}{q}\right) \cdot 1_L$ \\
                                     \> $\implies$ $\left(\frac{p}{q}\right)\left(\frac{q}{p}\right)=(-1)^{\frac{q-1}{2} \cdot \frac{p-1}{2}}$
\end{tabbing}
$\left[\mbox{Hinweis: } \left(\frac{p}{q}\right) \in \{\pm 1 \} \implies \left(\frac{p}{q}\right)^{-1} = \left(\frac{p}{q}\right) \right]$ \\
Details: 1. Man verschaffe sich $\omega$: $L = \MdF_{p^{q-1}}$ enthält primes Element $\zeta$, ord $\zeta=p^{q-1}-1$. Bekanntlich $p^{q-1} \equiv 1$ mod $q$ wegen $\overline p \in \MdF_q^x$ (Euler) \\
$\implies$ $q \mid p^{q-1}-1$ $=$ ord $\zeta$. Setze $\omega=\zeta^\frac{\mbox{ord }\zeta}{q}$ \\
$\implies$ ord $\omega=q$. \\
Nachrechnen (b): Verwende: In Körper $L$ mit $\MdF_p$ Teilkörper ist $(\alpha + \beta)^p=\alpha^p + \beta^p$ \\
$\tau^p=\sum_{\alpha \in \MdF_q}\underbrace{\left(\frac{\alpha}{p}\right)^p}_{=\left(\frac{\alpha}{q}\right)}\omega^{\alpha p}$ \quad$\left\{\alpha p \mid \alpha \in \MdF_q\right\} = \MdF_q$ da $p\in\MdF_q^x$.\\
$\left[\mbox{Summationstransfer: }\beta=\alpha p\implies\left(\frac{\alpha}{q}\right)=\left(\frac{\beta \overline p ^{-1}}{q}\right) = \left(\frac{\beta}{q}\right)\left(\frac{\overline p}{q}\right)^{-1}\mbox{( da $\chi_q$ Homomorphismus)}\right]$\\
$\implies$ $\tau^p=\sum_{\beta \in \MdF_q}\underbrace{\left(\frac{\overline p}{q}\right)^{-1}}_{\left(\frac{p}{q}\right)}\left(\frac{\beta}{q}\right)\omega^\beta = \left(\frac{p}{q}\right) \sum_{\beta \in \MdF_q} \left(\frac{\beta}{q}\right)\omega^\beta = \left(\frac{p}{q}\right)\tau$ \\
Wegen $\tau \neq 0$ (folgt aus a) (b) OK.\\
(a) später \\
Zu den Ergänzungssätzen\\
$\left(\frac{-1}{q}\right) \equiv (-1)^\frac{p-1}{2}$ mod $p$, $-\frac{p}{2} < \left(\frac{-1}{q}\right),(-1)^\frac{p-1}{2}<\frac{p}{2}$ \\
$\implies$ $\left(\frac{-1}{q}\right)=(-1)^\frac{p-1}{2}$\\
\begin{tabular}{lll}
Demnach & $-1$ quadratischer Rest mod $p$ & $\iff$ $p\equiv 1$ mod $4$, also für $p=5,13,17,23,\dotsc$ \\
        & $-1$ quadratischer Nichtrest mod $p$ &  $\iff$ $p\equiv -1$ mod $4$, also für $p=3,7,11,\dotsc$ \\
\end{tabular}\\
\underline{Bsp:} $-1\in\MdF_{13}$ $5^2 \equiv -1$ mod $13$
\end{beweis}

$\left(\frac{p}{q}\right)\left(\frac{q}{p}\right) =
(-1)^{\frac{p-1}{2} \frac{q-1}{2}}$

$\tau = \sum_{\alpha \in
\MdF_q}\left(\frac{\alpha}{q}\right)\omega^\alpha$, $\ord(\omega) =
q$, Gaußsche Summe

\underline{Berechnung $\tau^2$:}\\
Sei $\left(\frac{0}{q}\right) = 0,\ \alpha \in \MdF_q^\times$:
\begin{align*}
\tau^2 = &\sum_{\alpha \in \MdF_q}\left(\frac{\alpha}{q}\right)\omega^\alpha \cdot \sum_{\beta \in \MdF_q}\left(\frac{\beta}{q}\right)\omega^\beta \\
=& \sum_{\alpha \in \MdF_p^\times} \sum_{\beta \in \MdF_q}\left(\frac{\alpha}{q}\right)\left(\frac{\beta}{q}\right)\omega^{\alpha + \beta},\qquad   (\MdF_q = \{\underbrace{\alpha + \beta}_{:= \gamma} \big| \beta \in \MdF_q) \\
=& \sum_{\alpha \in \MdF_q^\times} \sum_{\gamma \in \MdF_q} \left(\frac{\alpha}{q}\right)\left(\frac{\gamma - \alpha}{q}\right)\omega^\gamma  \\
=& \sum_{\gamma \in \MdF_q} \underbrace{\sum_{\alpha \in \MdF_q^\times} \left(\frac{\alpha}{q}\right)\left(\frac{\gamma - \alpha}{q}\right)}_{=: C_\gamma}  \\
\end{align*}

\begin{itemize}
    \item[\underline{$\gamma = 0$:}] $C_0 = \sum_{\alpha \in \MdF_q^\times}\underbrace{\left(\frac{-\alpha^2}{q}\right)}_{\left(\frac{-1}{q}\right)} = (q-1)\left(\frac{-1}{q}\right)\cdot
 1_L$
    \item[\underline{$\gamma \not= 0$:}] $\left(\frac{\alpha}{q}\right)\left(\frac{\gamma - \alpha}{q}\right) = \underbrace{\left(\frac{\alpha}{q}\right)\left(\frac{\alpha}{q}\right)}_{= 1}\left(\frac{\gamma \alpha^{-1} -1}{q}\right)$
    \begin{align*}
        C_\gamma =& \sum_{\alpha \in \MdF_q^\times}\left(\frac{\gamma \alpha^{-1} - 1}{q}\right) \\
        & \left[X := \{\gamma \alpha^{-1} \big| \underbrace{\alpha \in \MdF_q^\times,\ \alpha \neq \gamma}_{q-2 \ \alpha\text{'s}}\} \subseteq \MdF_q^\times \implies \#X = q-2,\ -1 \not\in X \implies X = \MdF_q^\times \setminus \{-1\}\right] \\
        =& \sum_{\sigma \in \MdF_q^\times \backslash \{-1\}} (\frac{\sigma}{q}) \\
        =& \underbrace{\sum_{\sigma \in \MdF_q^\times} (\frac{\sigma}{q})}_{\tiny\begin{matrix}
             =\left(\frac{q-1}{2}\right)\cdot 1 - (\frac{q-1}{2})\\
             \text{(da gleich viele quadratische}\\ \text{Reste wie Nichtreste)}
             \end{matrix}}-\left(\frac{-1}{q}\right) \\
        =& - \left(\frac{-1}{q}\right)
    \end{align*}
    \begin{align*}
        \tau^2 =& \sum_{\gamma \in \MdF_q}C_\gamma \omega^\gamma \\
        =& (q-1)\left(\frac{-1}{q}\right)\cdot 1_L + \sum_{\gamma \in \MdF_q^\times}-\left(\frac{-1}{q}\right)\omega^\gamma \\
        =& (q-1)\left(\frac{-1}{q}\right)\cdot 1_L - (\frac{-1}{q})\sum_{j=0}^{q-1}\omega^j + \underbrace{\left(\frac{-1}{q}\right)}_{\text{Kompensiert } j=0} \\
        =& q\left(\frac{-1}{q}\right)\cdot 1_L - \underbrace{\left(\frac{-1}{q}\right)\frac{\omega^q - 1}{\omega - 1}}_{=0, q = \ord(\omega) \text{, da } \omega^q = 1}
    \end{align*}
    \item []    \underline{Ergebnis:} $\tau^2 = q\left(\frac{-1}{q}\right) \dot 1_L$ (a)\\
        \underline{Ergänzungssatz $\left(\frac{2}{q}\right):$} Übung
\end{itemize}

\paragraph{Anwendung der Eulerformel und des quadratischen
Reziprozitätsgesetzes}
Hiervon gibt es viele. Hier über $\MdF_n$.\\
Euler: $a^{\frac{p-1}{2}} \equiv \left(\frac{a}{p}\right) \mod p, p > 2, p \nmid a \implies \overline a^{\frac{p-1}{2}} = \left(\frac{a}{p}\right)$ in $\MdF_p$\\
$\left(\frac{a}{p}\right) = -1 \implies \ord(\overline a) \nmid \frac{p-1}{2}$, immer $\ord(\overline a) \mid p-1$\\
\underline{Also:} $v_2(\ord(\overline a)) = v_2(p-1)$\\
Sagt am Meisten, wenn $p-1 = 2^k, k > 0$. Dann $\ord(\overline a)
\mid 2^k, \ord(\overline a) \nmid 2^{k-1} \implies \ord(\overline a)
= p-1 = 2^k \implies \overline a$ ist primitiv.

Falls $2^k + 1 = p \in \MdP$, so ist $a$ Primitivwurzel $\iff
(\frac{a}{p}) = -1 (p \in \MdP \implies k = 2^n, n \in \MdN_+, p =
F_n = 2^{2^n}+1$ n-te Fermatzahl (1. Übungsblatt).

Falls das so ist, so ist $3$ eine Primitivwurzel $\mod p$.\\
Berechne $(\frac{3}{p})$. $p = 2^k + 1 \equiv 1 \mod 4 (k \ge 2) \implies (-1)^{\frac{p-1}{2}} = 1 \implies (\frac{3}{p})(\frac{p}{3}) = (-1)^{\frac{2}{2} \cdot \frac{p-1}{2}} = 1 \implies (\frac{3}{p}) = (\frac{p}{3})$ (quadratisches Reziprozitätsgesetz!)\\
Berechne $p \mod 3$. $p = F_n = 2^{2^n}+1, n \ge 1$. (Folgende
Äquivalenz stimmt wohl nicht ganz, bitte überprüft das jemand) $2
\equiv -1 \mod 3, p \equiv (-1)^{2^n} + 1 \equiv 1 + 1 \equiv -1
\mod 3 \implies \left(\frac{3}{p}\right) = \left(\frac{p}{3}\right)
= \left(\frac{-1}{3}\right)$

\begin{satz}[Fermat-Zahl-Satz]
    \begin{enumerate}
        \item Sei $k \in \MdN_+, p = 2^k + 1$. Dann gilt $p \in \MdP \implies k = 2^n (n \in \MdN) \implies p = F_n = 2^{2^n} + 1$
        \item Ist $p = F_n \in \MdP, a \in \MdZ, p \nmid a, n \ge 1$, so gilt: $a$ Primitivwurzel $\mod a \iff (\frac{a}{p}) = -1$. Trifft zu auf $a = 3$
        \item Pepins-Test: Sei $n \in \MdN_+$. Dann gilt: $F_n = 2^{2^n} + 1 \in \MdP \iff 3^{2^{(2^n - 1)}} \equiv -1 \mod F_n$
    \end{enumerate}
\end{satz}

\begin{beweis}
    \begin{enumerate}
        \item $\checkmark$
        \item $\checkmark$
        \item \underline{"`$\Longrightarrow$"':} $F_n = p \in \MdP \implies 3$ Primitivwurzel
        $\mod p$, $\ord(\overline 3) \mid p-1 = 2^{2^n} \implies \overline 3^{2^{2^n-1}}
        = \overline 3^{\frac{2^{2^n}}{2}} = \pm 1$. Bei $+1$ keine Primitivwurzel.\\
        \underline{"`$\Longleftarrow$"':} Sei $p \in \MdP, p \mid F_n = 2^{2^n} + 1$.
        $3^{2^{2^n-1}} \equiv -1 \mod F_n \implies 3^{2^{2^n-1}} \equiv -1 \mod p,
         3^{2^{2^n}} \equiv 1 \mod F_n \implies 3^{2^{2^n}} \equiv 1 \mod p$.
         $F_n -1 = \ord(\overline 3) = 2^{2^n} \le p-1\ (\ord(\overline 3) $ in $\MdF_p$
         teilt ${\#\MdF_p^\times = p-1)}$
    \end{enumerate}
\end{beweis}

\subsection{Jacobi-Symbol}
\begin{definition}
    Sei $a \in \MdZ,\ m \in \MdN_+,\ 2 \nmid n,\ \ggt(a,m) = 1\ (\star)$. Definiere in diesem
    Fall das Jacobi-Symbol $\left(\frac{a}{m}\right)$ durch:
    \[
        \left(\frac{a}{m}\right) = \prod_{\scriptsize\begin{matrix}p\in \MdP\\p
        \mid m\end{matrix}} \left(\frac{a}{p}\right)_L^{v_p(m)},
    \]
    %Stephan, 4.7.
    %hier stand zuerst, die Summe laufe über alle Primzahlen, die m _nicht_ teilen.
    %das sind ziemlich viele Primzahlen... Außerdem deckt sich die Definition jetzt mit
    %meinem Buch.
    andernfalls ist $\left(\frac{a}{m}\right)$ nicht definiert.
    Hierbei ist $\left(\frac a p\right)_L$ das Legendre-Symbol.

    Klar:
    \begin{itemize}
        \item[] $\left(\frac{a}{1}\right) = \left(\frac{1}{m}\right) = 1$
        \item[] $m \in \MdP, m > 2$, so ist Jacobi $\left(\frac{a}{m}\right) = $ Legendre $\left(\frac{a}{m}\right)$
    \end{itemize}
\end{definition}

\begin{satz}[Jacobi-Symbolsatz]\label{satz:Jacobisymbol}
    Falls $a, a' \in \MdZ, m, m' \in \MdZ$, so gelten, falls die vorhandenen Jacobi-Symbole definiert sind:
    \begin{itemize}
        \item[(i)] $a \equiv b \mod m \implies \left(\frac{a}{m}\right) = \left(\frac{b}{m}\right)$
        \item[(ii)] $\left(\frac{aa'}{m}\right) =
        \left(\frac{a}{m}\right)\left(\frac{a'}{m}\right),\
                    \left(\frac{a}{mm'}\right) = \left(\frac{a}{m}\right)\left(\frac{a}{m'}\right)$
        \item[(iii)] $\left(\frac{a}{m}\right)\left(\frac{m}{a}\right) = (-1)^{\frac{a-1}{2} \cdot \frac{m-1}{2}}\qquad$ (Reziprozitätsgesetz)
        \item[(iv)] $\left(\frac{-1}{m}\right) = (-1)^{\frac{m-1}{2}}, \left(\frac{2}{m}\right) = (-1)^{\frac{m-1}{8}}\qquad$ (Ergänzungssätze)
    \end{itemize}
\end{satz}

\underline{Algorithmus-Skizze zur Berechnung von
$\left(\frac{a}{m}\right)$}
\begin{itemize}
    \item[0.] $m = 1: \left(\frac{a}{m}\right) = \left(\frac{a}{1}\right) = 1$
    \item[1.] $m > 1, 2 \nmid m, \left(\frac{a}{m}\right) = \left(\frac{r}{m}\right)$ mit
        $r = a$ mods $m$ (also $|r| < \frac{m}{2}$)
    \item[2.] Stelle $r$ dar als $r = \text{sign}(r)2^{v_2(r)}r_0$ (also $r_0 > 0, 2 \nmid r_0, |r| < \frac{m}{2}$)\\
        Rechenaufwand minimal!\\
        $\left(\frac{r}{m}\right) = \underbrace{\left(\frac{\text{sign}(r)}{m}\right)\left(\frac{2}{m}\right)^{v_2(r)}}_{=: \Upsilon}(\frac{r_0}{m})$\\
        Rechenaufwand für $\Upsilon$ ist ebenfalls minimal.
    \item[3.] $\left(\frac{r_0}{m}\right) = \left(\frac{m}{r_0}\right)(-1)^{\frac{r_0 - 1}{2} \cdot \frac{m-1}{2}}$,
     wende Verfahren auf $(\frac{m}{r_0}$ an. Problem reduziert von $m$ auf $r_0$ mit $0 < r_0 < \frac{m}{2}$.
     Schleife wird ca. $\log_2 m$ mal durchlaufen.\\
     \textbf{!} Primzerlegung kommt nirgends vor \textbf{!}
\end{itemize}

\paragraph{Bemerkung}
Aus $\left(\frac{a}{m}\right) = 1$ folgt \underline{nicht}, dass $a$ quadratischer Rest $\mod m$ ist.\\
\begin{beispiel}
    $\left(\frac{2}{15}\right) = \left(\frac{2}{3}\right)\left(\frac{2}{5}\right) = (-1)(-1) =
    1$.
    $2$ ist quadratischer Nichtrest $\mod 3$ und erst recht quadratischer Nichtrest von $\mod 15$
\end{beispiel}

\begin{beweis}[Jacobi-Symbolsatz \ref{satz:Jacobisymbol}]
    \begin{itemize}
        \item[(i)] $p \mid m, p \in \MdP, a \equiv b \mod m \implies a \equiv b \mod p \implies
        \left(\frac{a}{p}\right) = \left(\frac{b}{p}\right) \implies \left(\frac{a}{m}\right) = \left(\frac{b}{m}\right)$
        \item[(ii)] $\left(\frac{a}{p}\right)\left(\frac{b}{p}\right) = \left(\frac{ab}{p}\right)$
        (Legendre Symbol) $\implies \left(\frac{a}{m}\right)\left(\frac{b}{m}\right) = \left(\frac{ab}{m}\right)$\\
            $\left(\frac{a}{mm'}\right) = \prod_{p \in \MdP}\left(\frac{a}{p}\right)^{v_p(mm')} =
            \prod_{p \in \MdP}\left(\frac{a}{p}\right)^{v_p(m) + v_p(m')} = \prod_{p \in \MdP}\left(\left(\frac{a}{p}\right)^{v_p(m)}\left(\frac{a}{p}\right)^{v_p(m')}\right)
            = \prod_{p \in \MdP}\left(\frac{a}{p}\right)^{v_p(m)} \cdot \prod_{p \in \MdP}\left(\frac{a}{b}\right)^{v_p(m')}
            = \left(\frac{a}{m}\right)\left(\frac{a}{m'}\right)$
        \item[(iii)] $\left(\frac{a}{m}\right)\left(\frac{m}{a}\right) = (-1)^{\frac{a-1}{2} \cdot \frac{m-1}{2}}$ klar für $m = 1$ oder $a = 1$. Also $m > 1, a > 1$ voraussetzbar. $2 \nmid m, 2 \nmid a$.\\
        Falls $m \in \MdP$ und $a \in \MdP (\ggt (m, n) = 1)$, so steht das quadratische Reziprozitätsgesetz für das Legendre Symbol da.\\
        Also nur noch zu beweisen, wenn $a$ oder $m \not\in \MdP$ etwa $m = uv,\ 1 < v < m$.\\
        Induktion nach $a,m$:\\
        Induktionhypothese: $\left(\frac{a}{u}\right)\left(\frac{u}{a}\right) = (-1)^{\frac{a - 1}{2} \cdot \frac{u - 1}{2}}$, $\left(\frac{a}{v}\right)\left(\frac{v}{a}\right) = (-1)^{\frac{a - 1}{2} \cdot \frac{v - 1}{2}}$\\
        $\left(\frac{a}{uv}\right)\left(\frac{uv}{a}\right) \stackrel{(ii)}{=} \left(\frac{a}{u}\right)\left(\frac{a}{v}\right)\left(\frac{u}{a}\right)\left(\frac{v}{a}\right) \stackrel{\text{I.H.}}{=} (-1)^{\frac{a-1}{2} \cdot \frac{u - 1}{2}} (-1)^{\frac{a-1}{2} \cdot \frac{v-1}{2}} \stackrel{\text{?}}{=} (-1)^{\frac{a-1}{2} \cdot \frac{uv-1}{2}}$

    Genügt: $n - 1 + v - 1 = uv-1 \mod 4$. Das stimmt, weil $2 \nmid u, 2 \nmid v$ und $u,v \equiv \pm 1 \mod 4$
    \item[(iv)] Ähnliche Induktion
     \end{itemize}
\end{beweis}

\chapter{Primzahltests }


Ein Primzahltest ist ein Algorithmus $Prim(m)$, der zu $m \in \MdN_+$ entscheidet, ob $m \in \MdP \vee m \not\in \MdP$.

Einteilung der Tests ($\neg$disjunkt):
\begin{itemize}
 \item[a)] 
  \begin{itemize}
   \item[+] Allgemeiner Test ($\forall m \in \MdN$)
   \item[-] Spezieller Test (nur gewisse $m \in \MdN$)
  \end{itemize}
 \item[b)] 
  \begin{itemize}
   \item[+] Voll bewiesener Test
   \item[-] Test abhängig von einer Vermutung (zB Riemann-Vermutung)
  \end{itemize}
 \item[c)] 
  \begin{itemize}
   \item[+] Sicherer Test
   \item[-] Propabilistischer Test (Monte-Carlo-Methode)
  \end{itemize}
 \item[d)] 
  \begin{itemize}
   \item[+] Praktikabler Test (geht für "`große"' $m$)
   \item[-] Unpraktischer Test
  \end{itemize}
\end{itemize}

\begin{beispiel}
 \begin{itemize}
  \item[a)] Pepins Test: nur für $F_n = 2^{2^n} + 1$
  \item[d)] Naiver Test: Probiere $a \mid m, \forall a \in \MdN, 1 < a \le \sqrt{m}$
  \item[d)] Wilsons Test: $m \in \MdP \Leftrightarrow (m-1)! \equiv -1 \mod m$, es sind mindestens $m$ "`Aktionen"' nötig
 \end{itemize}
\end{beispiel}

\begin{beweis}[Wilsons Test]
 \begin{itemize}
  \item[\underline{"`$\Rightarrow$"':}] $m = p \in \MdP$. In $\MdF_p:$\\
   $(m-1)! = \prod_{\alpha \in \MdF_p^\times}\alpha = \overline 1 \cdot (\overline{-1})$. Paare $\alpha\alpha^{-1}$ heben sich weg. Wenn $\alpha \not= \alpha^{-1}$ verbleibt $\alpha^2 = 1$, da $\alpha = \pm 1 \Rightarrow (m-1)! \equiv -1 \mod m$
  \item[\underline{"`$\Leftarrow$"':}] $m \not\in \MdP \Rightarrow \ggt ((m-1)!, m) = d > 1 \Rightarrow (m-1)! \not\equiv 1 \mod m$ (sonst $d \mid -1$)
 \end{itemize}
\end{beweis}

Prinzip moderner PZTests:\\
Meist ohne Einschränkung $m > 2, 2 \nmid m$. (Rechnung für große $m$ aufwändig, daher gewöhnlich erst $p \mid m$ probiert für die $p \in \MdP$, etwa $p \le 100000 \vee p \le 1000000$.). Man konstruiert Gruppe $G_m$ derart, dass die Struktur von $G_m$ für $m \in \MdP \wedge m \not\in \MdP$ verschieden ausfällt. Die Strukturverschiedenheit soll mit möglichst wenig und schnellen Rechnungen festgestellt werden.\\
EZT: Meist $G_m  = (\MdZ / m\MdZ)^\times$\\
Höhere ZT: Etwa $G_m = (\sigma_k / \sigma_k \cdot m)^\times$, webei $\sigma_k$ ein Ring "`ganzer algebraischer Zahlen "` im algebraischen Zahlenkörper $K$ ist.
\begin{beispiel}
 $K = \MdQ + \MdQ i, \sigma_k = \MdZ + \MdZ i$ (Ring der ganzen Gaußschen Zahlen)\\
 Algebraische Geometrie: $G_r$ konstruiert aus "`elliptischer Kruve"', die über $\MdZ$ definiert ist. Vorzug: Es gibt $\infty$ viele elliptische Kurven und Zahlenkörper. Man kann versuchen, möglichst "`geeignete"' zu finden. Hier $G_m = (\MdZ / m \MdZ)^\times$.
\end{beispiel}

\begin{itemize}
 \item[(A)] Ein $\neg$ganz geklückter Versuch\\
  Strukturaussage für $G_p (p \in \MdP)$:\\
  Satz von Euler-Fermat: $\overline a^{p-1} = 1$.
  \begin{definition}
   Sei ohne Einschränkung $m > 2, 2 \nmid m$. $a \in \MdZ$ heiße \underline{Carmichael-Zeuge} (für die Zerlegbarkeit von $m$), wenn gilt:
    \begin{itemize}
     \item[(i)] $\ggt (a,m) = 1$
     \item[(ii)] $a^{m-1} \not\equiv 1 \mod m$
    \end{itemize}
  \end{definition}
  Klar: Wenn Zeuge gefunden: $m \not\in \MdP$.\\
  Leider: $\exists m \in \MdN$ mit $m \not\in \MdP$, aber kein Zeuge vorhanden!
  \begin{definition}
   Solche $m \not\in \MdP$ (also die mit $\forall a \in \MdZ, 1 < a < m, \ggt (a,m) = 1$ ist $a^{m-1} \equiv 1 \mod m$) heißen \underline{Carmichael Zahlen}.
  \end{definition}
  \begin{satz}[Carcmichael, $\sim 1920$]
   Sei $m \in \MdN_+, m > 2, \MdP_m := \{p \in \MdP \big| p \mid m\}$. Dann: $m$ ist Carmichael Zahl $\Leftrightarrow$ Es gelten:
    \begin{itemize}
     \item[(i)] $2 \nmid m$
     \item[(ii)] $m$ ist qf (???) ($\forall p \in \MdP: v_p(m) \le 1$)
     \item[(iii)] $\forall p \in \MdP_m: p-1 \big| m-1$
     \item[(iv)] $m$ hat mindestens 3 verschiedene Primteiler ($\#\MdP_m \ge 3$)
    \end{itemize}
  \end{satz}
  \begin{beispiel}
   Kleinste Carmichael-Zahl: $m = 561 = 3 \cdot 11 \cdot 17$ - $2,10,16 \mid 560$
  \end{beispiel}
  \begin{beweis}
   \begin{itemize}
    \item[\underline{"`$\Leftarrow$"':}] 
     $\left.
      \begin{matrix}
       \text{Zeige } (i)-(iv)\\
       \ggt (a,m) = 1
       \end{matrix}\right
       \} \Rightarrow a^{m-1} \equiv 1 \mod m$.\\
     $\forall p \in \MdP_m:$ in $\MdF_p^\times: \ord \overline a \mid p-1 \stackrel{(iii)}{\mid} m-1 \Rightarrow \overline a^{m-1} = 1$ in $\MdF_p \Leftrightarrow a^{m-1} \equiv 1 \mod p \Leftrightarrow p \mid a^{m-1} -1 \stackrel{(ii)qf}{\Rightarrow} m = \prod_{p \in \MdP_m}p \mid a^{m-1}-1 \Rightarrow a^{m-1} \equiv 1 \mod m$
    \item[\underline{"`$\Rightarrow$"':}] ($-1$) kein Zeuge $\Rightarrow (-1)^{m-1} \equiv 1 \mod m$. Falls $2 \mid m \Rightarrow -1 \equiv 1 \mod m \Rightarrow m = 1,2$ (Widerspruch!). Also $2 \nmid m \leadsto (i)$.\\
    Zu (ii), (iii):\\
    Für $p \in \MdP_m$ ist $t:= v_p(m) \ge 1. \exists PW a \mod p$ mit $\ggt(a, m) = 1$ (Sei $w$ PW $\mod p$, lose das System $a \equiv w \mod p (ChRS), a \equiv 1 \mod q (q \in \MdP, q \not= p). \Rightarrow q \nmid a, p \nmid a \Rightarrow \ggt(a, m) = 1$)\\
    In $(\MdZ / p^t\MdZ)^\times$ ist $\overline a^{m-1} = 1$ (wegen $a^{m-1} \equiv 1 \mod m \Rightarrow a^{m-1} \equiv 1 \mod p^t) \Rightarrow \ord \overline a = \phi(p^t) = p^{t-1}(p-1) \mid m-1 \Rightarrow p-1 \mid m-1 \leadsto (iii)$\\
    Wäre $t > 1 \Rightarrow p \mid m-1$ (Wiederspruch zu $p \mid m$).\\
    Also $v_p(m) = 1 \leadsto$ (ii)\\
    Noch zu widerlegen: $\MdP_m = \{p, q\}, p \not= q$, etwa $2 < p < q (\star)$\\
    $m = pq$ laut (ii), $q-1 \stackrel{(iii)}{\mid} m-1 = pq - 1 = p(q-1) + p-1 \Rightarrow q-1 \mid q-1 \Rightarrow q \le p$ (Widerspruch $(\star)$)
   \end{itemize}
  \end{beweis}
 \item[(B)] Ein geglückter Versucht\\
  $m \in \MdN, m > 2, 2 \nmid m$. Schreibe $m-1 = 2^t \cdot u$ mit $t = v_2(m-1)$ also $2 \nmid u, t > 0$.
  \begin{definition}
   $a \in \MdN$ heiße \underline{Miller-Zeuge} (für die Zerlegbarkeit von $m$), wenn gilt:
   \begin{itemize}
    \item[(i)] $\ggt (a,m) = 1$
    \item[(ii)] $a^u \not\equiv 1 \mod m$
    \item[(iii)] $\forall s \in \{0,...,t-1\}: a^{u\dot 2^s} \not\equiv -1 \mod m$
   \end{itemize}
  \end{definition}
  \begin{satz}{Miller-Rabin-PZTest}
   Sei $m \in \MdN, m > 2, 2 \nmid m$. Dann: $m \not\in \MdP \Leftrightarrow \exists$ Miller-Zeuge $a$. $(0 < a < m)$
  \end{satz}
  Zusatz (Rabin): Es gibt dann höchstens $\frac{3}{4}\phi(m) \le \frac{3}{4}(m-1)$ $\neg$Zeugen\\
  $\leadsto$ Liefert voll bewiesenen Test:\\
  Test, ob $\frac{1}{4}(m-1)+1$ $a$s Zeugen sind.\\
  Sobald Zeugen gefunden $\Rightarrow m \not\in \MdP$.\\
  Kein Zeuge gefunden $\Rightarrow m \in \MdP$.\\
  Aber immer noch unpraktisch (ca $\frac{1}{4}m$ Aktionen). Es gibt einen sehr praktischen propabilistischen Test:\\
  Teste, ob $k$ zufällig ausgewählte Restklassen $\overline a$ $(1 < a < m)$ Zeuge sind (falls $\ggt (a,m) = d > 1$, so $m \not\in \MdP$, sonst $\ggt (a,m) = 1$). Falls Zeuge gefunden $\Rightarrow m \not\in \MdP$. Falls kein Zeuge gefunden: Die WK (???), dass man sich mit der Annahme "`$m$ ist prim"' irrt, ist $< \frac{1}{4^k}$.\\
  Für große $m$ scheint die WK sogar \underline{viel} kleiner als $\frac{1}{4^k}$. [experiment. Faktoren]
  
  $\begin{array}{cl}
   m < & \text{Zeuge, falls } m \not\in \MdP \\
   2047 & 2 \\
   1373653 & 2 \vee 3 \\
   3215031753 & 2,3 \vee 5
  \end{array}$
  \begin{beweis}
   \item[\underline{"`$\Leftarrow$"':}] $m = p \in \MdP, \overline a \in \MdF_p^\times$\\
    $ord \overline a \mid \phi(p) = p-1 = 2^t \dot u$\\
    $ord \overline a = 2^s \cdot v, 2 \nmid v, s \le t, v \mid u$
    \begin{itemize}
     \item[1. Fall:] $s = 0 \Rightarrow \overline a^v = 1 \Rightarrow \overline a^u = 1 \Rightarrow a^u \equiv 1 \mod p$, kein Zeuge
     \item[2. Fall:] $s > 0 \Rightarrow \overline a^{2^s \dot v} = 1, \overline a^{2^{s-1}\dot v} \equiv -1 \mod m$, $s \in \{0,...,t-1\} \Rightarrow$ kein Zeuge
    \end{itemize}
  \end{beweis}
\end{itemize}

Weiter bei der letzten Vorlesung:

$m-1 = 2^tu, 2 \nmid u$\\
\underline{Millerzeuge $a$}: $\ggt (a,m) = 1, a^u \not\equiv 1 \mod m$\\
$\forall s = 0, ..., t-1: a^{u2^s} \not\equiv 1 \mod m$

\underline{Rest:}\\
$m \not\in \MdP \Rightarrow \exists$ Millerzeuge
\begin{itemize}
    \item[Fall I:] $\#\MdP_m \ge 2, \MdP_m = \{p_1,...,p_l\}$\\
        $a \equiv -1 \mod p_1$\\
        $a \equiv 1 \mod p_j (j > 1)$\\
        (mit Chinesischem Restsatz lösen)\\
        $a^u \equiv (-1)^u \equiv -1 \mod p_1$, also ist $a^u \equiv 1 \mod m$ falsch (sonst $-1 \equiv 1 \mod p_2 \Rightarrow p_1 = 2$ [Widerspruch!]), also $a^u \not\equiv 1 \mod m$\\
        $a^{u2^s} \equiv 1^{u2^s} \equiv 1 \mod p_j (j > 1) \Rightarrow a^{u2^s} \equiv -1 \mod m$ ist falsch, also $a^{u2^s} \not\equiv 1 \mod m$\\
        Gesehen: $a$ ist Millerzeuge
    \item[Fall II:] $m = p^t, p \in \MdP, t > 1:$ ist $a$ Primitivwurzel $\mod m = p^t$, so ist $a$ Millerzeuge.\\
        $ord(\overline a) = \phi(p^t) = (p-1)p^{t-1}$
        \begin{itemize}
            \item{ $\Rightarrow$} $\overline a^u \not= 1$, weil sonst $ord(\overline a) \mid u \Rightarrow p \mid u \mid m-1$ (Widerspruch zu $p \mid m$)
            \item{ $\Rightarrow$} $\overline a^{u2^s} = -1 \Rightarrow \overline a^{us^{s+1}} = 1 \Rightarrow ord(\overline a) = (p-1)p^{t-1} \mid u2^{s+1} \Rightarrow p \mid u \mid m-1$ (Widerspruch!) $\Rightarrow a^{u2^s} \equiv -1 \mod m$
        \end{itemize}
\end{itemize}

\underline{Stand der Technik:}
\begin{itemize}
    \item[1.)] Primzahlen $< 10^{130}$ mit guter Sicherheit "`leicht"' auffindbar, z.B. mit Miller Rabin
    \item[2.)] Zahlen der Größe $> 10^{130}$, erstrecht $m = pq, p,q \ge 10^{130}$ können nicht faktorisiert werden.
\end{itemize}

Praktischer Test von Rumely, fast in Polynomial-Zeit, vorhanden
(Zeit $\approx \log(m)^{c\log \log \log m}$). Falls die
verallgemeinterte Riemann-Vermutung gilt, so ist dieser Test sogar
in Polynomial-Zeit.

Kayal, Saxena, Aal 2002: Voll bewiesener Primzahltest in
Polynomial-Zeit. Fraglich ob dies ein praktischer Test ist.

Faktorisierung großer Nichtprimzahlen schein ein viel härteres
Problem zu sein.

\underline{Idee von Fermat:}\\
$\MdN_+ \ni m = x^2 - y^2, x,y \in \MdN, m = (x-y)(x+y), x \ge y$
ist Faktorisierung, wenn $x-y \not= 1.m, x-y = 1$ und $x + y \not=
m.1, x+y = m \Rightarrow x = \frac{m+1}{2}, y = \frac{m-1}{2}$ also
echte Teiler, wenn $x,y \not= \frac{m \pm 1}{2}$

Viele moderne Tests arbeiten so: Suche $x,y \in \MdN$ mit $x^2
\equiv y^2 \mod m, x \not\equiv \pm y \mod m$

Gute Chance, dass $\ggt (m, x-y)$ oder $\ggt (m, x+y)$ echter Teiler
von $m$ ist. Sehr viel Test, um die Suche nach solchen $x,y$ zu
beschleunigen: Siehe z.B. Förster, Algorithmic number theory

\section{Anwendung der EZT in der Kryptographie}
Rivests öffentliches Chiffrier System. $m$ große Zahl.\\
Nachricht ist \underline{hier} $N \in \text{Versys}_m^\times = \{a
\in \MdN \big| 0 < a < m, \ge (a,m) = 1\}$ (Falls $m =
p_1^{n_1}\cdot ... \cdot p_l^{n_l}, p1 < ... < p_l \in \MdP, n_j \in
\MdN_+$, so sind alle $N \in \MdN$ mit $1 \le N < p_1$ im
$\text{Versys}_m$. $N$ kodiert Textabschnitt mit $k$ Zeichen, z.B.
Leerstelle = $000$, Jedes Zeichen erhält Ziffern $< 1000$.
\begin{beispiel}
    $\text{ }$\\
    $N =$ \begin{tabular}{ccccccccccr}
        K & O & M & M & & N & I & C & H & T & \\
        011 & 015 & 013 & 013 & 000 & 014 & 009 & 003 & 008 & 020 & $< 10^{3k}$
    \end{tabular}
\end{beispiel}

\begin{definition}
    \begin{itemize}
        \item[(i)] Eine Chiffre ist (für uns) eine bijektive Abbildung $P: \text{Versys}_m^\times \to \text{Versys}_m^\times, N' = P(N)$ ist die "`chriffrierte"' Nachricht.
        \item[(ii)] ein "`öffentliches Chiffresystem"' ist eine Liste ("`öffentliches Adressbuch"'):\\
            $(T, P_T), T \in \tau =$ Menge von Teilnehmern. $P_T$ Chiffre, derart, dass $T \not= T' \Rightarrow P_T \not= P_{T'}$
            \begin{itemize}
                \item[(a)] Jeder Teilnehmer $T \in \tau$ erhaält das Adressbuch $(T, P_T)_{T \in \tau}$
                \item[(b)] $T$ und nur $T$ erhält $P_T^{-1}$ (Umkehrabbildung von $P_T$)\\
                    Praktisch: $T$ muss $P_T^{-1}$ besonders gut sichern, gegen Diebstahl, Ausspähen, Hacker, usw.
            \end{itemize}   \end{itemize}
\end{definition}

\underline{Technische Anforderungen:}\\
\begin{itemize}
    \item[1.)] $P_T(N), P_T^{-1}(N)$ müssen in vernüftiger Realzeit berechenbar sein
    \item[2.)] Nicht einmal ein Supercomputer kann $P_T^{-1}$ aus $P_T$ ermitteln ($P_T$ \emph{Trapdoor}-Funktion)
    \item[3.)] Nur $T$ hat $P_T^{-1}$. Der Systemadministrator hat am Anfang die $P_T$'s und die $P_T^{-1}$'s. Nach Absenden von $P_T^{-1}$ an $T$ vernichtet er $P_T^{-1}$
\end{itemize}

\underline{Anwendungen:}\\
\begin{itemize}
    \item[I)] Geheime Nachricht über öffentlich  zugängliche Kanäle (etwa Internet) übermitteln $T$ von $A$ zu $B$, $A,B \in \tau$ ohne das Unbefugte $N$ gewinnen können.\\
        \underline{Methode:} $A$ berechnet $P(N) = N'$ und sendet $N'$ an $B$. \underline{Nur} $B$ kann aus $N'$ wieder $N = P_B^{-1}(N')$ ermitteln.\\
        \underline{Beispiel:}\\
            \begin{itemize}
                \item $A$ Spion des Geheimdienstes, $B =$ = Geheimdienstzentrale, $C,D$ die gegnerischen Geheimdienste
                \item $A$ ist Bank, $B$ ist Kunde, $N =$ Kontostand
            \end{itemize}
    \item[II)] Geheimnachricht mit elektronischer Unterschrift\\
        \underline{Methode:} $A$ sendet an $B$: "`$N = P_BP_A^{-1}(N), \text { Gruß }A$"'. Nur $A$ kann $N'$ herstellen, nur $B$ kann daraus $N = P_AP_B^{-1}(N')$ gewinnen.\\
        \underline{Beispiel:}\\
            $A =$ Kunde, $B =$ Bank, $N =$ "`Überweisen Sie 200'000.- von meinem Konto an $C$"'
    \item[III)] Sichere Speicherung von Nachrichten\\
        \underline{Methode:} Speichere $N' = P_{A_t}^{-1}(N)...P_{A_1}^{-1}(N)$. Benötigt werden $A_1,..., A_t \in \tau (t = 1)$. Nur mit Willen von allen Mitwirkenden $A_1,..., A_t$ kann $N$ aus $N'$ wieder rekonstruiert werden.
\end{itemize}

EZT kann z.B. zum Erfüllen der technischen Vorraussetzungen verwendet werden.\\
Rivests Vorschlag $\subseteq$ RSA-Code (Rinest, Shamn, Adleman 1978)\\
Adressbuch: Liste$(T, m_T, s_T), m_T, s_T \in \MdN, m_T = p_1^{n_1}...p_l^{n_l}, p_i$ zu Anfang dem Administrator bekannt, öffentlich nur $m_T$'s, $s_T$'s ziemlich groß.\\
Chiffre $P_T(N) := (N^{s_T} \mod m_i)$. Dann theoretisch
$P_T^{-1}(N') = N'^{t_T}$, wobei $t_Ts_T \equiv 1 \mod \phi(N)$
(Euler Funktion). Hiermit erhält $T$ auch noch $t_T$. $t_T$ ist nur
berechenbar, wenn $\phi(m) = m\prod_{p \mid m}(1-\frac{1}{p})$
bekannt, dass geht nur (nach heutigem Wissen), wenn Primzerlegung,
also die $p_i$ bekannt sind.

\chapter{Ganzzahlige lineare Gleichungen und Moduln über euklidischen Ringen}
\section{Der Elementarteileralgorithmus}
\subsection{Matrizen über euklidischen Ringen} Sei $(R, gr)$
ein Euklidischer Ring.
\begin{definition}
    \begin{itemize}
        \item[(i)] $GL_n(R) = (R^{n \times n})$ heißt \emph{allgemeine lineare Gruppe} über $R$ (GL = general linear)
        \item[(ii)] $1_n := 1_{GL_n(R)}$ ($n \times n$-Einheitsmatrix)
    \end{itemize}
\end{definition}
\begin{lemma}
    $GL_n(R) = \{U \in R^{n \times n} \big| \text{ det } U \in R^\times\}$\\
    (falls $R = \MdZ, U \in GL_n(\MdZ) \Leftrightarrow U \in \MdZ^{n \times n}, \det U = \pm 1$)
\end{lemma}
\begin{beweis}
    \begin{itemize}
        \item[(i)] $U \in (R^{n \times n})^\times \Leftrightarrow \exists V \in R^{n \times n}, VU = UV = 1_n \Rightarrow 1 = \det 1_n = \det (UV) = \underbrace{\det U}_{\in R} \cdot \underbrace{\det V}_{\in R} \Rightarrow \det U \in R^\times$
        \item[(ii)] Sei $U \in R^{n \times n}, \det U \in R^\times$. In LA I zeigt man für die Adjungierte $U^\#$ von $U$: $UU^\# = U^\#U = \det U \cdot 1_n$\\
            $U^\#$ wird aus $\det W$ gewonnen, wo $W$ Untermatrizen von $U$ sind, also $\det W \in R \Rightarrow U^\# \in R^{n \times n}, \det U \in R^\times \Rightarrow U^{-1} = \frac{1}{\det U}U^\# \in R^{n \times n} \Rightarrow U \in (R^{n \times n})^\times$
    \end{itemize}
\end{beweis}
\begin{definition}
    $B = (b_{ij}) \in R^{m \times n}$, so sei $\ggt(B) := \ggt(b_{ij})$ ($i = 1,...,m$ und $j = 1,...,n$)
\end{definition}
\begin{lemma}
    $A \in R^{l \times m}, B \in R^{m \times n}$. Dann gilt:
    \begin{itemize}
        \item[(i)] $\ggt(A) \mid \ggt(AB), \ggt(B) \mid \ggt(AB)$
        \item[(ii)] $U \in GL_m(R), V \in GL_n(R)$, so ist $\ggt(UBV) = \ggt(B)$
    \end{itemize}
\end{lemma}
\begin{beweis}
    \begin{itemize}
        \item[(i)] $A = (a_{ij}), B = (b_{kl}), d = \ggt(A) \Rightarrow a_{ij} = d \cdot a_{ij}', a_{ij}' \in R$. $AB = C = (c_{rs}), c_{rs} = \sum_{j = 1}^md_{rj}b_{js} = d \cdot \sum_ja_{ij}'\cdot b_{js} \Rightarrow \forall r,s: d \mid c_{rs} \Rightarrow d \mid \ggt(C) = \ggt(c_{rs} \big| r,s)$.\\
            $\ggt(B) = \ggt(AB)$ genau so.
        \item[(ii)] $\ggt(B) \mid \ggt(UB) \mid \ggt(U^{-1}(UB)) = \ggt(B) \Rightarrow \ggt(B) = \ggt(UB)$.\\
            $\ggt(UB) = \ggt((UB)V)$ genau so
    \end{itemize}
\end{beweis}
\underline{Spezielle Matrizen:}\\
$E_{ij}$ "`Matrizeneinheiten"', $E_{ij,kl} = \delta_{ik}\delta{jl}$. Es steht in der $i$-ten Zeile und der $j$-ten Spalte eine $1$.\\
Beispiel: $\begin{pmatrix}0 & & & 0\\
                            & \ddots & 1 \\
                            & & \ddots & \\
                          0 & & & 0\end{pmatrix}$\\
\emph{Elementarmatrizen} sollen folgende Matrizen genannt werden (in
$R^{n\times n}$):
\begin{itemize}
    \item[1.)] \emph{Additionsmatriizen}: $A_{ij}(b) = \underbrace{1_n}_{= E_n} + b\cdot E_{ij} (i \not= j)$.\\
        Beispiel: $\begin{pmatrix}1 & & & 0\\
                                  & \ddots & b \\
                                  & & \ddots & \\
                                0 & & & 1\end{pmatrix}$\\
  \item[2.)] \emph{Vertauschungsmatrizen}: $V_{ij} = 1_n - E_{ii} - E_{jj} + E_{ij} + E_{ji}$.\\
    Beispiel: $\begin{pmatrix}1 & & & & & 0 \\
                                  & \ddots & & & & \\
                                  & & 0 & 1 & & \\
                                  & & 1 & 0 & & \\
                                  & & & & \ddots & \\
                                0 & & & & & 1\end{pmatrix}$\\
  \item[3.)] \emph{"`Einheitsdiagonalmatrizen"'}: \\
    $diag_j(\epsilon) = \begin{pmatrix}1 & & & & & & \\
                                                                             & \ddots & & & & & & \\
                                                                             & & & 1 & & & & \\
                                                                             & & & & \epsilon & & \\
                                                                             & & & & & 1 & & \\
                                                                             & & & & & & \ddots & \\
                                                                             & & & & & & & 1\\\end{pmatrix}, \epsilon \in R^\times$
\end{itemize}
Laut LA: $\det A_{ij}(b) =1, \det(V_{ij}) = -1 (i \not= j), \det diag_j(\epsilon) = \epsilon \Rightarrow$ \\
\underline{Alle Elementarmatriizen sind in $GL_n(R)$}

Weiter Matrizen besonderer Form:\\
\emph{Diagonalmatrizen}: $D = \diag(d_1,...,d_r,0,...,0)$ (in $R^{m \times n}$). Für $r = 0: D = 0$.\\
Beispiel: $\begin{pmatrix}d_1 & & & & 0 \\
                              & \ddots & & & \\
                              & & d_r & & \\
                              & & & 0 & \\
                            0 & & & & \ddots\end{pmatrix}$

\begin{bemerkung}
%Ich denke hier ist Elementarteilerform gemeint, nicht Elementarform? (Julia)
    Eine Matrix $B \in R^{n \times n}$ heiße in "`\emph{Elementarteilerform}"' $\Leftrightarrow B = \diag(d_1,...,d_r,0,...0), d_1,...,d_r$ normiert und $d_r \not=0$ und $d_1 \mid d_2 \mid ... \mid d_r$ (dann $d_1 = \ggt(B)$)
\end{bemerkung}

Eine \underline{Elementaroperation} (ausgeübt auf $B \in R^{m \times n}$) ist eine der folgenden Operationen:\\
Zu $\Gamma$ Elementarmatrix bilde $B' = \Gamma B$ oder $B' =
B\Gamma$ und setzte wieder $B := B'$.

Liste:

\begin{tabular}{l|l}
    Zeilenoperationen & bewirkt \\
    \hline \\
    $B \to B =: B' = A_{ij}(b) \cdot B$        & Addition des $b$-fachen der $j$-ten Zeile von $B$ zur $i$-ten \\
    \hline \\
    $B \to B =: B' = V_{ij} \cdot B$           & Vertauschen der $i$-ten mit der $j$-ten Zeile \\
    \hline \\
    $B \to B =: B' = \diag_j(\epsilon) \cdot B$ & Multiplikation der $j$-ten Zeile mit $\epsilon$
\end{tabular}

\begin{tabular}{l|l}
    Spaltenoperationen & bewirkt \\
    \hline \\
    $B \to B =: B' = B \cdot A_{ij}(b)$        & Addition der $i$-ten Sapte $*b$ zur $j$-ten \\
    \hline \\
    $B \to B =: B' = B \cdot V_{ij}$           & Vertauschen der $i$-ten mit der $j$-ten Spalte \\
    \hline \\
    $B \to B =: B' = B \cdot \diag_j(\epsilon)$ & Multiplikation der $j$-ten Spalte mit $\epsilon$
\end{tabular}

Jeder Algorithmus der eine Matrix $A$ durch eine endliche Folge von
Elementaroperationen in Elementarteilerform überführt, heißt
\emph{Elementarteileralgorithmus}.

\underline{Vorschlag}:\\
Bearbeite Tripel $(U,B, V) \in GL_m(R) \times R^{m \times n} \times
GL_n(R)$ beginnend mit $(1_m, A, 1_n)$, so dass \underline{immer} $B
= UAV$ ist.

Elementaroperationen hier $(U,B,V) \to (U,B,V) := (\underbrace{\Gamma U}_{= U'},\underbrace{\Gamma B}_{= B'}, \underbrace{V}_{= V'})$ (Zeilenoperation) oder $(U,B,V) \to (U,B,V) := (\underbrace{U}_{= U'},\underbrace{B \Gamma}_{= B'}, \underbrace{V \Gamma}_{= V'})$ (Spaltenoperation).\\
Bedingung okay: $\underbrace{\Gamma UAV}_{U'A'V'} = \Gamma B = B'$,
ebenso $UAV\Gamma = B \Gamma = B'$

\underline{Ziel:} Steure die Operationen so, dass nach endlich vielen Elementaroperationen ein $(U,B,V)$ entsteht, mit $B =: D$ eine Elementarteilerform, also $A = UDV$.\\
Falls man so einen Algorithmus hat, so beweist das:
\begin{satz}[Elementarteilersatz]
    Sei $R$ ein euklidischer Ring, $m, n \in \MdN_+, A \in R^{m \times n}$
    \begin{itemize}
        \item[(i)] Dann gibt es ein $U \in GL_m(R), V \in GL_n(R)$ und $D \in R^{m \times n}, D$ in Elementarform, derart, dass \underline{$A = UDV$}
        \item[(ii)] $D$ ist durch $A$ eindeutig bestimmt
    \end{itemize}
\end{satz}

Zur Eindeutigkeit (Beweis-Skizze):\\
$d_1 = \ggt(D) = \ggt(UDV) = \ggt(A)$. Man kann zeigen: $d_1 \cdot
... \cdot d_j$ ist der ggT der Determinanten aller $j \times
j$-Untermatrizen von $A$.

\begin{bemerkung}
    \begin{itemize}
        \item[1.)] $A \in R^{m \times n}$, so $\det A = \det U \det D \det V$. Dann zur Berechnung von $\det A$ benutzt werden.
        \item[2.)] Idee für LGS: Für $A = D$ in Elementarteilerform kann Lösung unmittelbar abgelesen werden $\Rightarrow$ Lösung für $A$ wird mittels Rücktransformation ermittelt.
    \end{itemize}
\end{bemerkung}

\underline{LGS:}\\
$xA = b, A \in R^{m \times n}, b \in R^{1 \times n}$ (Zeile) ist
gegeben. Gesucht "`Lösung"' $x \in R^{1 \times m}$ (Zeile).
(LA oft $Ax = b$ mit Spalten, $Ax = b \Leftrightarrow x^TA^T = b^T$)\\
Besser: Information über die \emph{Lösungsmenge}: $\mathcal{L}(A,B) = \{x \in R^m = R^{1 \times m} \big| xA = b\}$\\
Antwort sehr leicht, falls $A = D = \begin{pmatrix}d_1 & & \\ &
\ddots & \\ & & d_r\end{pmatrix}$ in Elementarteilerform. $y =
(y_1,...,y_m) \in \mathcal{L}(D,c), c = (c_1,...,c_n)
\Leftrightarrow yD =
\underbrace{(y_1d_1,...,y_rd_r,0,...,0)}_{\text{n-Stück}}
\stackrel{!}{=} (c_1,...,c_n)$

Lösbarkeitsbedingung (notwendig und hinreichend): $\mathcal{L}(D,C)
\not= \emptyset \Leftrightarrow c_{r+1} = c_{r+1} = ... c_n = 0$
\underline{und} $d_1 \mid c_1, d_2 \mid c_2, ..., d_r \mid c_r$

Falls Bedingung erfüllt, so hat man die "`spezielle Lösung"' (wo $c_j = d_jy_j$, Bezeichnung $y_j = d_j^{-1}c_j)$.\\
$y \stackrel{(0)}{=}(d_1^{-1}c_1, ..., d_r^{-1}c_r, 0, ..., 0)$.

Die "`allgemeine"' Lösung hat die Form:\\
$y = y_0 + \sum_{j = r+1}^na_je_j, e_j = (0,...,0,1,0,...,0)$
Einheitsvektor, $a_j \in R$

$\begin{array}{ll}
y \in \mathcal{L}(D,c) & \Leftrightarrow yD = c (\text{auch } y_0D = c) \\
                       & \Leftrightarrow (y - y_0)D = 0 \\
                       & \Leftrightarrow z = (y - y_0)$ ist Lösung des zugehörigen homogenen Systems$ \\
                       & zD = 0$, d.h. von der Form $\sum_{j = r+1}^na_je_j
\end{array}$\\
Es muss $z_jd_j = 0$, also $z_0 = 0$ für $j = 1,...,r$ gelten.

Man transformiert $xA = b$ wie folgt auf Diagonalform: $xA = b \Leftrightarrow \underbrace{xU^{-1}}_{y} \underbrace{UAV}_{D} = \underbrace{bV}_{c} = 0$. $yD = c$, wo $c = bV$ und $y = xU^{-1}$, also $x = yU$ ist.\\
\underline{$\mathcal{L}(A,b) = \mathcal{L}(D,bV)\cdot U$}

%Vorlesung 13.07.2006 - Stephan

$(U,B,V) \in GL_m(R) \times R^{m\times n} \times GL_n(R)$, $B=UAV$.
\paragraph{Elementarteileralgorithmus} Idee: Falls $B \neq 0$, so
setzte
\[
    gr(B)=\min\{gr(b_{ij},\ i=1,2,\dotsc,m,\ j=1,2,\dotsc,n,\
    b_{ij}\neq 0 \}.
\]
Wenn es gelingt durch Elementaroperationen von $B$ nach $B'$
überzugehen, so dass $gr(B')<gr(B)$, so ist man induktiv fertig.

Zuerst benötigen wir einen Unteralgorithmus: ggTnachVorn(A):\\
Er soll zu einem $0\neq A \in R^{m\times n}$ $(U_1,B_1,V_1)$ mit
$U_1\in GL_m(R),\ v_1\in GL_n(R),\ b_1=U_1 A V_1$ gilt, wobei
\[
    B=\left(\begin{tabular}{c|c} $d_1$& 0\\\hline 0 & $A'$
    \end{tabular}\right),\quad d_1=\ggt(A).
\]
Skizze:\\
\begin{itemize}
    \item[0.] Initialisierung: $(U,B,V):=(1_m,A,1_n)$.
    \item[1.] Bestimme$(k,l)$ mit $gr(b_{kl}=gr(B)$.
    \item[2.] Fall I: Es gibt eine Zeile $i$ mit  $B_{kl} \nmid b_{il}$.
    Division  mit Rest: $b_{ij}=q b_{kl}+r$. Addiere $(-q)$--faches
    der $k$--ten Zeile. Das ergibt $B'$ mit
    $b_{il}'=b_{il}-qb_{kl}=r$. Induktiv sind wir fertig, denn:
    $gr(r)<gr(b_{kl})=gr(B)$. Weiter bei Schritt~1.
    \item[3.] Fall II: Es gibt eine Spalte $j$ mit $b_{kl}\nmid
    b_{kj}$. Genau wie bei Schritt~2, nur mit Spaltenoperationen
    erhalten wir $b_{kj}=q' b_{kl}+r'$. Addieren wir nun das
    $(-q')$--fache der $l$-ten Spalte auf die $j$--te Spalte,
    erhalten wir $B'$ mit $gr(B')<gr(B)$.
    \item[4.] Fall III: $b_{kl}\mid b_{il}$ und $b_{kl}\mid_{kj},\ \forall
    i,j$ aber $\exists (i,j)$ mit $b_{kl} \nmid b_{ij}$.
    $b_{il}=q''b_{kl},\ i\neq k, l\neq j$. Addiere $(1-q'')$-faches
    der $k$-ten Zeile zur $i$-ten hinzu:\\
    $b_{il}'=\underbrace{b_{ij}}_{q' b_{kl}}+(1-q'')b_{kl}=b_{kl}$\\
    $b_{ij}'=b_{ij}+(1-q'')b_kj \implies b_{kl}=b'{il}' \nmid
    b_{ij}$ (wegen $b_kl \nmid b_{ij}$, $b_{kl} \mid b_{kl}$)\\
    Fall~II liegt vor mit $i$-ter statt $k$-ter Zeile. $B:=B'$,
    $(k,l):=(i,l)$, weiter bei Schritt~3.
    \item[5.] $\forall i,j:\ b_{kl} \mid b_{ij}$ (letzter möglicher
    Fall). Vertausche $k$-te und 1.~Zeile und $l$-te und $j$-te
    Spalte. Entsteht $b$ mit $0\neq b_{11}\mid b_{ij}\ \forall i,j$
    $\implies  b_{11}$ ist ein $\ggt$, $\implies \exists \epsilon \in
    R^\times:d_1=\epsilon b_{11}=\ggt(B)\stackrel{\text{Lemma
    2}}{=}\ggt(A) \implies$ Multipliziere 1.~Zeile mit $\epsilon$:
    Es entsteht Matrix mit $b_{11}=d_1=\ggt(A)$. Wie bei
    Gaußalgorithmus erzeugt man jetzt in der ersten Spalte und
    ersten Zeile Nullen außer bei $b_{11}$. Jetzt hat man $(U,B,V)$
    mit $A=UBV)$ und $B=\left(\begin{tabular}{c|c} $d_1$& 0\\\hline 0 &
    $A'$
    \end{tabular}\right)$. Ausgabe: $(U_1,B_1,V_1):=(U,B,V)$
\end{itemize}
Klar: Man kann genauso mit $A'$ weitermachen: Braucht:
$d_n=\ggt(A)=ggt(B_1) \mid \ggt(A')$. Im Detail:
\paragraph{ELT(A)}:
\begin{enumerate}
    \item Falls $A\neq0$, Ausgabe: $(1_m,A,A_n)$.
    \item Anderfalls liefert ggTnachVorn(A) $(U_1,B_1,V_1)$ wie
    oben: Falls $n=1$ oder $M=1$, so fertig. Ausgabe
    $(U,D,V):=(U_1,B_1,V_1)$. Falls m,n>1 und $A'=0$, so wieder
    fertig. Ausgabe wie oben.\\
    Falls $A'\neq 0$, so liefert ELT($A'$) $(U',D',V')$ mit
    $U'D'V'=A'$ und
    \begin{align*}
        & U_1 \left(\begin{tabular}{c|c} $1$& 0\\\hline 0 & $U'$
    \end{tabular}\right) \left(\begin{tabular}{c|c} $d_1$& 0\\\hline 0 & $D'$
    \end{tabular}\right) \left(\begin{tabular}{c|c} $1$& 0\\\hline 0 & $V'$
    \end{tabular}\right) V_1\\
    =&U_1 B=\left(\begin{tabular}{c|c} $d_1$& 0\\\hline 0 & $\underbrace{U'D'V'}_{=A'}$
    \end{tabular}\right) V_1\\
    =&U_1B_1V_1\\
    =&A
    \end{align*}
    Ausgabe $(U,D,V)$ mit $U,D,V$ passend wie in obiger Formel.
\end{enumerate}
\paragraph{Einschub Beispielrechnung} (folgt vielleicht später, hab'
grade keine Lust, die zwei DinA4-Blätter abzutippen)

\section{Ganzzahlige Lösungen eines ganzzahligen linearen
Gleichungssystems} Betrache LGS $xA=B$, gegeben $a\in R^{m\times
n},\ b\in R^{1\times n}$.\\
Gesucht:$\mathcal{L}(A,B)=\{  x\in R^{1\times m}=R^m:\ xA=b  \}$

Elementarteilersatz: $A=UDV,\ D=\diag(d_1,d_2,\dotsc,d_r,0,\dotsc)$
in Elementarteilerform. $U\in GL_m(R),\ V\in GL_n(R)$. Gesehen:
$\mathcal{L}(A,b)=\mathcal{L}(D,bV) U$.
$c:=bV=(c_1,c_2,\dotsc,c_n)$.
\begin{satz}[LGS-Satz]\label{satz:LGS}
    Mit diesen Voraussetzungen und Bezeichnungen gilt:
    \begin{enumerate}
        \item $\mathcal{L}(A,b)\neq \emptyset \iff d_i\mid c_i,\
        i=1,2,\dotsc,r,\ c_{r+1}=c_{r+2}=c_n=0$.
        \item Lösung des homogenen Systems $xA=0$:\\
        $\mathcal{L}(A,0)=\mathcal{L}(D,0)U=\bigoplus_{j=r+1}^m R
        (e_jU)$. $e_j$ ist der $j$-te Einheitsvektor in $R^m$. Das
        heißt, eine $R$-Basis von $\mathcal{L}(A,0)$ ist gegeben
        durch Basis $b_{r+1},b_{r+2},\dotsc,b_m$, mit $b_j=e_jU$,
        also die $j$-te Zeile von $U$ ist. Falls $m\leq r$, so
        $\mathcal{L}(A,0)=0$, d-h- jede Lösung
        $y\in\mathcal{L}(A,0)$
        hat eindeutige Darstellung $y=\sum_{j=r+1}^m a_j b_j,\ a_j
        \in R$.
        \item Falls das LGS lösbar ist, so erhalt man die allgemeine
        Lösung $x$ aus einer spezielen Lösung $x_0$ in der Form
        $x=x_0+y$, $y\in \mathcal{L}(A,0)$. Mann kann wählen:
        $x_0=(d_1^{-1} c_1,d_2^{-1} c_2,\dotsc,d_r^{-1}
        c_r,0,\dotsc,0)$.
    \end{enumerate}
\end{satz}
\begin{beweis}
    Alles schon bewiesen\dots
\end{beweis}

\begin{bemerkungen}
\item Ist $A\in R^{n\times n}$, so gilt
\[ A\in GL_n(R) \iff D=1_n \]
\item Jedes $U\in GL_n(R)$ ist Produkt von Elementarmatrizen.
\end{bemerkungen}

\begin{beweis}
\begin{enumerate}
\item $A=UDV$, $U,V \in GL_n(R)$. $D\in GL_n(R) \iff n=r$, $d_1,\ldots,d_n=1\folgt D=1_n$
\item $A\in GL_n(R) \iff D=1_n \implies A=UV \implies$ Behauptung
\end{enumerate}
\end{beweis}

Freunde der Algebra mögen beachten, dass für ein $R$-Modul $M$ die selben Axiome wie für einen Vektorraum gelten, nur dass $R$ ein Ring statt einem Körper ist. Das $\MdZ$-Modul ist (fast) das selbe wie eine (additive) abelsche Gruppe. Die Hauptneuheit ist, dass man im Allgemeinen in $M$ eine $R$-Basis hat.

Ein Beispiel dazu ist mit $R=\MdZ$ das Modul $M=(\MdZ/2\MdZ),+)$. Wäre die Basis die leere Menge, so wäre $M=0$, Widerspruch. Ist nun $b$ ein Element der Basis, so wären alle $z\cdot b$, $z\in\MdZ$ verschieden, also $\#M = \infty$, was auch ein Widerspruch ist.

In der Algebra zeigt man leicht: Ist $M = \langle u_1, \ldots, u_m \rangle = \{ \sum_{i=1}^m \alpha_i u_i \mid \alpha _i \in R \}$, so existiert ein $A\in R^{m\times n}$ mit $M \cong R^n/R^m\cdot A$. Klar: $A=UDV$ wie im Elementarsatz, also $R^m = R^m\cdot U$, $R^n=V\cdot R^n$
\begin{align*}
\folgt M&\cong R^n/R^mUDV \\
&= R^nV/R^mDV \\
&\cong R^n/R^mD \\
&= (R\oplus \cdots \oplus R) / (Rd_1 \oplus \cdots \oplus Rd_r \oplus 0 \oplus \cdots \oplus 0) \\
&\cong R/Rd_1 \oplus \cdots \oplus R/R_dr \oplus R \oplus \cdots \oplus R
\end{align*}
Damit ist die Struktur bestimmt. So kann die Eindeutigkeit von $D$ auch bewiesen werden.

Ist $R=\MdZ$, so ist $(\MdZ/d\MdZ,+)$ zyklisch, erzeugt von $1+d\MdZ=\overline 1$, $\MdZ$ sowieso zyklisch.

Als Ergebnis haben wir: Jede endlich erzeugbare abelsche Gruppe ist direktes Produkt zyklischer Gruppen.

Die $R$-lineare Abbildung $R^l \to R^k$ beschriebung durch Darstellungsmatrizen in $R^{l\times k}$. Der Elementarteiler-Algorithmus liefert Mittel $\kernn(f)$ und $\bild(f)$ explizit zu beschreiben.

\chapter{Ganzzahlige quadratische Formen}

\section{Grundbegriffe und Bezeichnungen}

\paragraph{Problem:} Man diskutiert die diophantische Gleichung
\[ k = ax^2 + bxy + cy^2 \quad (*)\]
Gegeben sind $a,b,z,k \in\MdZ$, gesucht ist ein $\underline{x} = (x,y)\in\MdZ^2$, für die $(*)$ gilt.

Gegeben $Q=aX^2+bXY+cY^2 \in \MdZ[X,Y]$, $a,b,c \ne 0$, mit Kurzbezeichnung $Q=[a,b,c]$. Dieses $Q$ heißt ganzzahlige binäre (wegen den 2 Variablen) quadratische ($\grad q=2$) Form.

Nun betrachtet man $Q$ als Abbildung $\MdZ^2 \to \MdZ^2$, $\underline x = (x,y) \mapsto Q(x,y)$.

\begin{definition}
\begin{enumerate}
\item $\underline x$ primitiv $\iff \ggt(x,y)=1$
\item $Q$ primitiv $\iff \ggt(a,b,c) = 1$
\item $Q$ stellt $k\in\MdZ$, $k\ne 0$ (primitiv) da $\iff \exists \underline x \in \MdZ^3$ ($\underline x$ primitiv), mit $Q(\underline x) = k$
\end{enumerate}
\end{definition}

\paragraph{Problem:} Welche Formen stellen welche Zahlen dar? $Q(\MdZ^2) =$ ?

Falls $k\in Q(\underline x)$, welche weiteren $\underline x'$ erzeugen $k=Q(\underline x')$? $Q^{-1}(\{k\})=$ ?

\begin{bemerkung}
\begin{enumerate}
\item $z\in\MdZ$, so $Q(z\cdot \underline x)= z^2 \cdot Q(\underline x)$
\item Mit $Q$ ist auch $mQ$ eine Quadratische Form ($m\in\MdZ$, $m\ne 0$)
\end{enumerate}
Wegen (1) genügt es meist, primitive Darstellungen zu betrachten.
\end{bemerkung}

Aus der Linearen Algebra ist über reelle Quadriken bekannt: Es gibt Darsellungsmatrixen $A_Q = \MdR^{2\times 2}$ mit $Q(x) = xA_Q x^\top$, wobei 
\[ A_Q = \begin{pmatrix} a & \frac b 2 \\ \frac b 2 & c \end{pmatrix}\]

\paragraph{Idee} (Gauß?) Wegen $\MdZ^2 U = \MdZ^2$ für $U\in GL_2(\MdZ)$ gilt $Q(\MdZ^2) = Q\cdot (\MdZ^2 U)$. $Q(\underline xU) = \underline xU \cdot A_Q \cdot (x U)^\top = \underline x (U A_Q U^\top) x^\top$

\begin{definition}
\begin{enumerate}
\item Zu $Q$ sei $U.Q$ die Quadratische Form mit Darstellungsmatrix $UA_QU^\top$
\item $Q$ und $Q^\top$ heißen (eigentlich) äquivalent ($Q \sim Q'$ bzw $Q \approx Q'$) $\iff \exists U\in GL_2(\MdZ)$ (bzw. $\exists I \in SL_2(\MdZ)$, wobei $SL_2(\MdZ) = \{ U\in\MdZ^{2\times 2}\mid \det U = 1\}$) mit $Q' = U.Q$.
\end{enumerate}
\end{definition}

$\sim$, $\approx$ unterscheiden sich wenig, sozusagen höchstens um eine Matrix $\begin{pmatrix} 0 & 1 \\ 1 & 0\end{pmatrix}$.

\begin{bemerkung}
\begin{enumerate}
\item $1_2.Q = Q$, $U,V \in GL_2(\MdZ)$. $(UV).Q = U.(V.Q)$.\\
"`$GL_2(\MdZ)$ bzw. $SL_2(\MdZ)$ operiert auf der Menge der Quadratischen Formen"'
\item $\sim$, $\approx$ sind Äquivalenzrelationen
\item Äquivalente Formen stellen die selben Zahlen dar.
\end{enumerate}
\end{bemerkung}

\begin{beweis}
\begin{enumerate}
\item $UV.Q$: $UVA_Q(UV)^\top = U(VA_QV^\top)U^\top: U.(V.Q)$.

Folgt $Q'=U.Q$, so $U^{-1}.Q' = U^{-1}.(U.Q)=(U^{-1}U).Q = 1_2. Q =Q$.

Also ist $\sim$ symetisch: $Q\sim Q$.

Transitivität: $Q\sim Q'$, $Q'=U.Q$ und $Q'\sim Q''$, $Q'' = V.Q$, mit $U,V\in GL_2(\MdZ)$, so ist $Q''=V.(U.Q)=(VU).Q \folgt Q''\sim Q$
\end{enumerate}
\end{beweis}

\section{Die Diskriminante}

Sei $Q=[a,b,c]$ eine Quadratische Form.

\begin{definition}
$\Delta =-4 \cdot \det A_Q = b^2 - 4ac = \dis(Q) \in \MdZ$ heißt Diskriminante von $Q$.
\end{definition}

Bemerkung aus der Linearen Algebra: $\mathcal{V} = \mathcal{V}_{Q-k}(\MdR) = \{\underline x \in \MdR^2 \mid Q(\underline x) = k\}$ ist reelle Quadrik, abgesehen von ausgearteten Fällen gilt: $\Delta <0$: $\mathcal{V}$ Ellipse, $\Delta >0$, $\mathcal{V}$ Hyperbel.

\begin{beispiel}
$X^2 + 5Y^2$ Ellipse: $\Delta = 0-4\cdot 5 = -20 < 0$\\
$X^2 + -2Y^2$ Hyperbel: $\Delta = 0-4\cdot (-2) = 8 > 0$
\end{beispiel}

\paragraph{Problem:} Welche $(x,y)\in\MdZ^2$ (Gitterpunkte) liegen auf $\mathcal{V}$.

\begin{satz}[Diskriminantensatz]
Sei $Q$ eine Quadratische Form.
\begin{enumerate}
\item Ist $Q\sim Q'$, so gilt $\dis(Q) = \dis(Q')$.
\item Ist $\Delta = \dis Q$ ein Quadrat in $\MdZ \iff$ "`$Q$ zerfällt über $\MdZ$"', also $\exists u,v,w,z\in\MdZ$ mit $Q=(uX+vY)(wX+zY)$
\item Ist $\dis Q \ne 0$, so gilt 
\begin{align*}
 Q \text{ definit } &\iff \dis Q <0 \\
 Q \text{ indefinit } &\iff \dis Q > 0
 \end{align*}
\item $0\ne d\in \MdZ$ ist Diskriminante $\iff d\equiv 0,1 \mod 4$
\end{enumerate}
\end{satz}

Anwendung: $\Delta = \dis Q$ sei ein Quadrat
$Q(\underline x) = k \ne 0 \iff \exists d\in\MdZ, d\mit k$: $ux+vy=d$, $wx+zy=\frac kd$. Die Frage nach den darstellbaren $k$ läuft zurück auf a) Bestimmung aller Teiler von $k$, b) Diskussion eines ganzzahligen LSG.

Ab jetzt interessieren nur noch nichtquadratische Diskriminanten.

\begin{beweis}
\begin{enumerate}
\item[(4)] $\delta = \dis Q = b^2 - 4ac \equiv b^2 \equiv 0,1 \mod 4$. \\
$d\equiv 0 \mod 4$: $Q=[1,0,-\frac d 4]$ \\
$d\equiv 1 \mod 4$: $Q=[1,1,-\frac {1-d} 4]$ \\
Für diese Formen gilt $\dis Q = d \equiv \Delta$. Diese Form heißt "`Hauptform"' der Diskriminante.
\item[(1)] $\det UA_QA^\top = \det U \cdot \det U^\top \cdot \det A_Q = (\det U)^2 \cdot \det A_Q = \det A_Q \folgt$ Behauptung.
\item[(2)] (Skizze)
\begin{enumerate}
\item["`$\Leftarrow$"'] Nachrechnen
\item["`$\Rightarrow$"'] $\Delta = \dis Q = q^2$. Sei $t=\ggt(a, \frac {b-a} 2)$, dann (Übung):
\[ Q=(\frac a t X + \frac {b-q} {2t} Y)(t X + \frac{b+q}{2\frac at} Y) \]
\end{enumerate}
\item[(3)] $a=0\folgt \Delta > 0$, $Q=bXY + cY^2 = (bX + cY) Y$ indefinit\\
$a\ne 0$: $aQ = (aX+bY)^2 - \frac 14 \Delta Y^2$. Offensichtlich: $\Delta < 0$: definit, $\Delta > 0$: indefinit
\end{enumerate}
\end{beweis}<++>

\section{Darstellung von Zahlen durch QFen}
Vor. $Q$ QF, $\dis Q = \Delta$ sei kein Quadrat.\\
$U.Q$ QF mit Matrix $UA_qU^T, U \in GL_2(\MdZ)$\\
$U = \begin{pmatrix}r & s \\ u & v\end{pmatrix} \Rightarrow U.Q = [Q(r,s), 2rU\cdot a + (rv + su)b + 2sv\cdot c, Q(u,v)]$

Spezialfälle:\\
$Q' = \begin{pmatrix}1 & 0\\t & 1\end{pmatrix}.Q = [a, t \cdot 2a + b, at^2 + bt + c]$\\
$Q' = \begin{pmatrix}\cdot & 1\\ -1 & t\end{pmatrix}.Q = [c, -b + 2ct, ct^2 - bt + a]$\\
$Q' = \begin{pmatrix}\cdot & 1\\ -1 & \cdot\end{pmatrix}.Q = [c, -b, a]$\\
$Q' = \begin{pmatrix}1 & \cdot\\1 & 1\end{pmatrix}.Q = [a, 2a + b, a + b + c]$

Wunsch:\\
Algorithmus der feststellt, ob $Q$ $k$ darstellt oder nicht.

\begin{satz}[1. Darstellungssatz]
$Q$ stellt $0 \not= k \in \MdZ$ genau dann primitiv dar, wenn: $\exists Q' = [k,l,m]$ mit $Q' \approx Q \wedge -|k| < l \le |k|$.
\end{satz}
Hat man also einen Algorithmus, der feststellt, ob $Q \approx Q' \vee Q \not\approx Q'$, so hat man einfach $2k$ Formen zu testen (auf Äquivalenz zu $Q$). ($m = \frac{l^2 - \Delta}{4k}$)

Spezialfall:\\
$k = 1, Q$ stellt $1$ dar $\Leftrightarrow Q \approx [1, 0, \frac{-\Delta}{4}]$ (für $\Delta \equiv 0 \mod 4$)\\
--HIER FEHLT NOCH EINE ZEILE, WELCHE NICHT RICHTIG KOPIERT WURDE --

$Q \approx [1, 1, \frac{1 - \Delta}{4}]$ (für $\Delta \equiv 1 \mod 4$).\\
Ergebnis: Genau die zur Hauptform äquivalenten Formen stellen $1$ dar.

\begin{beweis}
\begin{itemize}
\item[\underline{"`$\Leftarrow$"':}] $Q'(1,0) = k$. Hat man $Q' \approx Q \Rightarrow Q$ stellt $k$ dar
\item[\underline{"`$\Rightarrow$"':}] $k = Q(x,y), \ggt (x,y) = 1$. LinKomSatz liefert $u,v \in \MdZ$ mit $xv-yu = 1 \Rightarrow U := \begin{pmatrix}x & y\\u & v\end{pmatrix} \in Sl_2(\MdZ)$\\
$Q_1 := U.Q = [\underbrace{Q(x,y)}_{=k}, l', \text{irgendwas}]$, $l := l' \text{ mods } 2|k|, \exists t: l = l' + 2tk \Rightarrow Q' = \begin{pmatrix}1 & \cdot \\ t & 1\end{pmatrix}.Q_1$ wie verlangt.
\end{itemize}
\end{beweis}

\begin{satz}[2. Darstellungssatz]
Sei $k \in \MdZ, k \not= 0$. Genau dann gibt es eine Form $Q$ mit $\dis Q = \Delta$, die $k$ primitiv darstellt, wenn die Kongruenz $l^2 \equiv \Delta \mod 4k$ so lösbar ist, dass $\ggt (k, l, \frac{l^2 - \Delta}{4k}) = 1$.
\end{satz}

\begin{beweis}
\begin{itemize}
\item[\underline{"`$\Leftarrow$"':}] Einfach, die Form $[k, l, \frac{l^2 - \Delta}{4k}]$ tut es
\item[\underline{"`$\Rightarrow$"':}] $k$ so darstellbar $Q \approx Q' = [k, l, \frac{l^2 - \Delta}{4k}]$ nach \emph{1. Darstellungssatz} (für (mindestens) ein $l$) $\Rightarrow \frac{l^2 - \Delta}{4k} \in \MdZ \Rightarrow l^2 \equiv \Delta \mod 4k$ [ggT stimmt auch]
\end{itemize}
\end{beweis}

Spezialfälle:\\
Sei $k = p \in \MdP$
\begin{itemize}
\item $p \nmid \Delta, p \not= 2: p$ so darstellbar $\Leftrightarrow (\frac{\Delta}{p}) = 1$
\item $p \mid \Delta, p \not= 2: p$ so darstellbar $\Leftrightarrow v_p(\Delta) = 1$
\item $p = 2 \mid \Delta$: $2$ so darstellbar $\Leftrightarrow \Delta \equiv 8, 12 \mod 16$
\end{itemize}

Zu den Spezialfällen
\begin{itemize}
\item $p \nmid \Delta: (\frac{\Delta}{p}) = 1$ lösbar, $l_1^2 \equiv \Delta \mod p \Leftrightarrow l_1^2 \equiv \Delta \mod 4p \leadsto ChRs$
\item $2 \not= p \mid \Delta$: Löse $l \equiv 0 \equiv \Delta \mod p (\ast)$, $l^2 \equiv \mod 4 \Rightarrow l^2 \equiv \Delta \mod 4p$\\
$\ggt (\underbrace{p, l}_{\ggt = p}, \frac{l^2 - \Delta}{4p}) = 1 \Leftrightarrow p \nmid \frac{l^2 - \Delta}{4p} \Leftrightarrow p^2 \nmid l^2 - \Delta \Leftrightarrow p^2 \nmid \Delta$, da $p^2 \mid l^2$ nach ($\ast$). ($\Rightarrow v_p(\Delta) = 1$)
\item $p = 2 \mid \Delta$: Ü.
\end{itemize}

\begin{definition}
Die \underline{Klassenzahl} $h(\Delta)$ ist die Anzahl der Klassen eigentlich äquivalenter Formen mit Diskriminante $\Delta$. "`Schöne Resultate"', falls $h(\Delta) = 1$.\\
$\Rightarrow$ Alle Formen der Diskriminante $\Delta$ stellen $k$ dar $\Leftrightarrow$ Bed. 2. DarstSatz.
\end{definition}

Später. $h(-4) = 1, Q = [1,0,1]$
Ergebnis: $2 \not= p \in \MdP$ wird durch $Q = x^2 + y^2$ dargestellt $\Leftrightarrow 1 = (\frac{-4}{p}) = \frac{-1}{p} = (-1)^{\frac{p-1}{2}} \Leftrightarrow p \equiv 1 \mod 4$
Andere Beispiele, etwa $\Delta = -164$ (Klassenzahl 1, betragsmäßig größte negative Zahl. Im positiven unbekannt)

\section{Reduktion der definiten Formen}

Sei $\Delta < 0$ [und damit "`Nicht-Quadrat"'], $\Delta = b^2 - 4ac \Rightarrow ac > 0$. Ohne Einschränkung positiv definit, d.h. $a > 0, c > 0$.
\begin{definition}[Gauß]
$Q$ (mit Diskr $\Delta$) heißt \underline{reduziert} $\Leftrightarrow |b| \le a \le c$
\end{definition}

In dieser Vorlesung:\\
$Q$ heißt \underline{vollreduziert} $\Leftrightarrow Q$ ist reduziert und falls $(c = 0 \wedge b \not= 0) \vee (|b| = a)$ auch noch $b > 0$ ist.

Idee (Gauß):\\
Setzte $|Q| := a + |b|$. Versuche $Q' \approx Q$ zu finden mit $|Q'| < |Q|$. Das geht, solange $Q$ nicht reduziert ist.
\begin{itemize}
\item[Fall I:] $a > c, Q' := \begin{pmatrix}\cdot & 1 \\ -1 & \cdot\end{pmatrix}, Q = [\underbrace{c}_{-a'}, \underbrace{-b}_{b'}, \underbrace{a}_{c'}]$. $|Q'| = a' + |b'| = |b| + c < |b| + a = |Q|$
\item[Fall II:] $a \le c, |b| > a$ (da $Q$ nicht-reduziert) Division von $b$ mit Rest durch $2a$: $\exists t \in \MdZ: b = b' - 2ta, -a < b' \le a$. $Q' = \begin{pmatrix}1 & \cdot \\ t & 1\end{pmatrix}.Q = [a, \underbrace{b + 2ta}_{b'}, c']$. $|Q'| = |b'| + a \le a + \underbrace{|a|}_{= a (\text{ da }-a \le a)}$
\end{itemize}

Dies ergibt Vollreduktionsalgorithmus $red(Q)$, der $\tilde Q$ berechnet mit $\tilde Q \approx Q \wedge \tilde Q$ vollreduziert. Wiederholte Anwendung von $Q := Q'$ aus Fall I,II endet nach endlich vielen Schritten mit reduziertem $Q_1 \approx Q$. Falls $Q_1$ vollreduziert, so $\tilde Q := Q_1$.\\
Falls $Q_1$ nicht vollreduziert, so $2$ Fälle für $Q_1 = [a,b,c]$
\begin{itemize}
\item $c = a$, aber $b < 0: \tilde Q := \begin{pmatrix}\cdot & 1 \\ -1 & \cdot\end{pmatrix}.Q_1 = [a,-b,a]$, jetzt $-b > 0$
\item $|b| = a$, also $b = -a < 0$. $\tilde Q = \begin{pmatrix}1 & \cdot\\1 & 1\end{pmatrix}.[a,-a,c] = [a,a,c], c' = a+b+c = c$ ist vollreduziert ($b' = a > 0$).
\end{itemize}
Ziel: $2$ vollreduzierte Formen der Disk $\Delta$ sind äquivalent $\Leftrightarrow$ sie sind gleich. Es folgt:\\
$Q \approx Q' \Leftrightarrow \text{ red }Q = \text{ red }Q'$. Daher gibt es einen Algorithmus, der entscheidet, ob $Q \approx Q' \vee Q \not\approx Q'$

Hilfsatz:\\
$Q = [a, b, c]$ sei reduziert. Dann:
\begin{itemize}
\item[(i)] $a = \min Q(\MdZ^2 \backslash 0)$
\item[(ii)] Für $a < c$ ist $Q^{-1}(\{a\}) = \{\pm(1,0)\}$ (klar: $Q(\underline x) = Q(-\underline x)$)\\
Für $0 \le b < a = c$ ist $Q^{-1}(\{a\}) = \{\pm (1,0), \pm (0,1)\}$. (Für $|b| = a = c$ (=$1$, da $Q$ primitiv) $Q[1, \pm 1, 1] = x^2 \pm yx + y^2 \Rightarrow \# Q^{-1}\{a\} = 6)$\\
\end{itemize}
$|b| \le a \le c$\\
$(\ast)$ $Q(x,y) = ax^2 + bxy + cy^2 \stackrel{(1)}{\ge} ax^2 - |bxy| - ay^2 \ge a(|x|-|y|)^2 + (2a - |b|)|xy| \ge a(\underbrace{(|x|-|y|)^2 + |xy|}_{\in \MdZ, \not= 0, \text{ wenn } (x,y) \not= 0, \text{ also } \ge 1} \stackrel{(4)}{\ge} a$.



Erinnerung:\\
$Q = [a,b,c]$ reduziert $\Leftrightarrow |b| \le a \le c$\\
Vollreduziert: Falls $a = c \wedge b \not= 0 \vee a = c = |b|$, so $b > 0 \leadsto$ Vollreduktionsalgorithmus red.

Sei $Q(x,y) = a \Rightarrow$ in ($\ast$) überall "`c"'\\
$a < c \Rightarrow y = 0$ (sonst bei (1) >)\\
"`="' bei (4) $\Rightarrow (|x|-|y|)^2 + |xy| = 1 \Rightarrow (x,y) \in M = \{\pm (1,0), \pm (0,1), (\pm 1, \pm 1)\}$
\begin{itemize}
\item[Fall I:] $Q^{-1}(a) = \{\pm (1,0)\}, \# Q^{-1}(a) = 2$
\item[Fall II:] $a = c$, aber $|b| < a \Rightarrow 2a-|b| > a \Rightarrow$ "`="' nur für $|xy| = 0$. $Q^{-1}(a) = \{\pm (1,0), \pm (0,1)\}$
\item[Fall III:] $a = c = |b|$, etwa $b > 0$, so $x^2 + xy + y^2 = 1$ von $(\pm 1, \pm 1)$ in $M$ nur $\pm (1, -1)$ [dazu noch $\pm (1,0), \pm(0,1)$] $\Rightarrow \#Q^{-1}(a) = 6$
\end{itemize}

Folgerung: Sei $Q, Q'$ vollständig reduziert und $Q \approx Q'$, so ist $Q = Q'$.
\begin{beweis}
$a = \min (Q(\MdZ^2 \backslash 0)) = \min (Q'(\MdZ^2 \backslash 0)) = a'$.
\begin{itemize}
\item[Fall I:] $a < c \wedge U = \begin{pmatrix}r & s \\ u & v\end{pmatrix}$ mit $U.Q = Q'$. $a = Q(1,0) = Q'(1,0) = Q((1,0)U) = Q(r,s) \Rightarrow (r,s) = \pm (1,0) \Rightarrow s = 0, \pm U = \begin{pmatrix}1 & 0\\ 0(?) & 1\end{pmatrix} = U$.\\
$Q' = (a,b + 2au, \ast (?)), |b| \le a, Q' \text{ red}$. $|b'| = |b + 2au| < a$. Wegen $|b| < a \Rightarrow U = 0, \pm U = \begin{pmatrix}1 & \cdot\\ \cdot & 1\end{pmatrix} \Rightarrow Q = Q'$
\item[Fall II:] $a = c, |b| \not= a$. $\# Q^{-1}(a) = 4 \Rightarrow$ II liegt auch für $Q'$ vor $\Rightarrow a = a' = c' \Rightarrow b^2 = b^{'2} \Rightarrow b' = \pm b$, aber nur $b$ möglich, da $Q'$ vollständig reduziert $\Rightarrow Q' = Q$.
\item[Fall III:] $a = c = |b| = b \Rightarrow$ Fall II auch für $Q'$ $\Rightarrow a = a' = c' = b'$
\end{itemize}
\end{beweis}

\begin{satz}[Hauptsatz über definite QFen]
Sei $\Delta \in \MdZ, \Delta \equiv 0,1 \mod 4, \Delta < 0$.
\begin{itemize}
\item[(i)] Zwei Formen $Q, Q'$ mit Diskriminante $\Delta$ sind nau dann eigentlich äquivalent, wenn $\text{red }(Q) = \text{ red }(Q')$ (mit VollredAlgo $\text{red}$)
\item[(ii)] Die vollreden Formen der Diskriminanten $\Delta$ bilden ein volles Vertretersystem aller eigentlichen Formenklassen, insbesondere ist die Klasse zu $U$ $h(\Delta)$ endlich.
\end{itemize}
\end{satz}

\begin{beweis}
\begin{itemize}
\item [(i)] $\exists U, U'$ mit $\text{red }Q = U.Q, \text{ red }Q' = U'.Q' (U, U' \in Sl_2(\MdZ))$ können in $\text{ red }$ berechnet werden. Multipliziere die Matrizen bei den Reduktionsschritten, $Q \approx \text{ red }Q, Q' \approx \text{ red }Q'$. $Q \approx Q' \Leftrightarrow \text{ red }Q \approx \text{ red }Q' \stackrel{\text{Folgerung}}{\Leftrightarrow} \text{ red}(Q) = \text{ red}(Q')$.
\item[(ii)] $Q$ reduziert $\Leftrightarrow |b| \le a \le c \Rightarrow b^2 \le ac \Rightarrow |\Delta| = -\Delta = -b^2 + 4ac \ge -b^2 + 4b^2 = 3b^2$. Abschätung: $|b| \le \sqrt{\frac{|\Delta|}{3}} \Rightarrow$ Nur endlich viele reduzierte $Q$s.\\
Dies ergibt Algorithmus zur Bestimmung von $h(\Delta)$: $h(\Delta) = \#$ vollreduzierten Formen zu $\Delta$. Reduzierte Form $Q = [a,b,c] \Leftrightarrow |b| \le \sqrt{\frac{|\Delta|}{3}}$, $\equiv \Delta \mod 2$, da $b^2 \equiv \Delta \mod 4$. $|b| \le a \le c \le ac = \frac{b^2 - \Delta}{4}$. Stelle alle diese $(a,b,c)$ auf, streiche die nicht vollreduzierten.
\end{itemize}
\end{beweis}

\begin{satz}[Heegner/Stark (1969)]
Für $\Delta < 0$ gilt: $h(\Delta) = 1 \Leftrightarrow \Delta \in \{-3,-4,-7,-8,-11,-12,-16,-19,-27,-28,-43,-67,-163\}$
\end{satz}
Beweis im Netz!

\begin{satz}[Siegel]
Für negative Diskriminanten $\Delta$ gilt $\lim_{|\Delta| \to \infty} h(\Delta) = \infty$
\end{satz}
($\Rightarrow$ Für jedes feste $\hat h \in \MdN$ gibt es $\infty$ viele $\Delta$ mit $h(\Delta) = \hat h$.)


Gauß definiert eine Verknüpfung (Komposition) zweier Formen $Q_1, Q_2 \Rightarrow Cl(\Delta) =$ Menge aller Formenklassen wird (endliche abelsche Gruppe \underline{"`Klassengruppe"'} genannt.\\
$\leadsto$ viele Vermutungen, wenige Sätze bis heute Gaußsche Geschlechtertheorie ersetzt $h(\Delta) = 1$ durch etwas schwächere Bedingung.

\section{Reduktion indefiniter Formen}

Vor: $Q = [a,b,c], \Delta = b^2 - 4ac > 0, \sqrt{\Delta} \not\in \MdQ$ ($\Delta$ kein Quadrat in $\MdZ$) [aber $a,c \not= 0$]\\
Ärger: Theorie viel komplizierter als bei $\Delta < 0$
\begin{definition}
\begin{itemize}
\item[(i)] $Q$ heißt \underline{halbreduziert} $\Leftrightarrow \sqrt{\Delta} - |2a| < b < \sqrt{\Delta}$
\item[(ii)] $Q$ heißt \underline{reduziert} $\Leftrightarrow 0 < b < \sqrt{\Delta} \wedge \sqrt{\Delta} - b < |2a| < \sqrt{\Delta} + b$
\end{itemize}
\end{definition}

\begin{satz}[Reduktionsungleichungen]
Für eine reduzierte Form $Q = [a,b,c]$ gilt:
\begin{itemize}
\item[] $ac < 0$
\item[] $0 \stackrel{(1)}{<} b \stackrel{(2)}{<} \sqrt{\Delta}$
\item[] $\sqrt{\Delta} - b \stackrel{(3)}{<} |2a| \stackrel{(5)}{<} \sqrt{\Delta} + b$
\item[] $\sqrt{\Delta} - b \stackrel{(4)}{<} |2c| \stackrel{(6)}{<} \sqrt{\Delta} + b$
\end{itemize}
$Q$ ist genau dann reduziert, wenn $(2), (3), (4)$ gelten.
\end{satz}

\begin{beweis}
Abschätzen $\leadsto$ Netz
\end{beweis}

\begin{folgerung}[Reduktionskriterium]
Sei $Q$ halbreduziert. Dann ist $Q$ reduziert, wenn eine der folgenden Ungleichungen gilt:
\begin{itemize}
\item[(i)] $|a| \le |c|$
\item[(ii)] $\sqrt{\Delta} - b < |2c|$
\end{itemize}
\end{folgerung}

\begin{beweis}
$(2), (3)$ ok bei halbreduzierten Formen
\begin{itemize}
\item[(ii)] fordert $(4)$
\item[(i)] Bei $|a| \le |c|: (3) \Rightarrow (4)$
\end{itemize}
\end{beweis}

\begin{bemerkung}
Zu $Q = [a,b,c] \exists ! t \in \MdZ$ mit $Q' = \begin{pmatrix}\cdot & 1 \\ -1 & t\end{pmatrix}.Q$ halbreduziert, denn $Q' = [\underbrace{c}_{=a'}, \underbrace{-b+2ct}_{=b'}, ct^2 - bt + c]$.\\
Zu erreichen. $\sqrt{\Delta} - \underbrace{|2a'|}_{|2c|} < b' < \sqrt{\Delta} \exists ! t$, so dass das stimmt.
\end{bemerkung}

Benennungen:
\begin{itemize}
\item[(i)] $Q' = [a',b',c']$ heißt \underline{rechter} (linker) \underline{Nachbar} von $Q = [a,b,c]$, wenn gilt: $b + b' \equiv 0 \mod 2c$ und $a' = c$ $(a = c')$ und $Q'$ halbreduziert.
\item[(ii)] $T =: T_Q$ aus Bew (oder Bem?) heiße \underline{Nachbarmatrix} (also $Q' = T_Q.Q$)
\end{itemize}

Leicht zu sehen: Jede QF hat je genau einen reuzierten rechten bzw. linken Nachbarn.

Reduktionsalgorithmus:\\
Wiederhole das Bilden des rechten Nachbars so lange, bis reduzierte Form erreicht ist.\\
Wieso terminiert? Ist $Q' = [c, -b+2ct, c']$ nicht-reduziert, so muss $(i)$ im Reduktionskriteriumg nicht vorliegen, d.h. $|a'| = |c| > |c'|$ (für $Q'$). Der Koeffizient $|c|$ kann nicht unendlich oft verkleinert werden.

\begin{satz}[Nachbarreduktionssatz]
\begin{itemize}
\item[(i)] Ist $Q = [a,b,c]$ reduziert, so ist auch der rechte Nachbar $Q'$ von $Q$ reduziert und es ist $\text{sign}(a) = -\text{sign}(a')$
\item[(ii)] Es gibt nur endlich viele reduzierte Formen.
\end{itemize}
\end{satz}

\begin{beweis}
\begin{itemize}
\item[(i)] Abschätzen $\leadsto$ mühsam
\item[(ii)] Klar. Nur endlich viele $b$ zu $\Delta$. Nur endlich viele $a,c$ laut Ungleichungen zu $B \Rightarrow$ Algorithmus zur Aufstellung aller reduzierten Formen.
\end{itemize}
\end{beweis}

$\Delta = -1$ bzw $\Delta = -4m, m \in \MdN, qf, 2 \nmid m$. Dann: Formen zu $\Delta$ stellen $p \in \MdP$ dar mit $p \mid m$ kann zur Faktorisierung von $m$ ausgenutzt werden. Hierzu schneller, hochgezüchteter Algorithmus von Shanks:
\begin{itemize}
\item[WH:] $Q$ indefinit, $\Delta > 0, \sqrt{\Delta} \not\in \MdQ$
\item[1.] $Q = [a,b,c]$ halbreduziert $\Leftrightarrow 0 < b < \sqrt{\Delta}, \sqrt{\Delta} - b < |2a| < \sqrt{\Delta} + b$. Rechter (halbreduzierter) Nachbar von $Q$ ist $Q' = [a',b',c'], Q' = \begin{pmatrix}\cdot & 1\\-1 & t\end{pmatrix}.Q, t$ mit $\sqrt{\Delta} - |2c| < -bt2ct < \sqrt{\Delta}$. Also $t = \text{sign}(c)\cdot \lfloor \frac{\sqrt{\Delta} + b}{|2c|} \rfloor$.
\end{itemize}
Algorithmus: Wiederholtes Nachbarbilden ergibt (irgendwann) reduzierte Form. 

Sei $Q = Q_0$ reduziert.$Q_{j+1} = Q'_j (j \ge 0)$. Da es nur endlich viele reduzierte Formen gibt, muss vorkommen: $\exists k,l \in \MdN, l > 0$ mit $Q_k = Q_{k+l}$. \\
Der reduzierte linke Nachbar ist $Q_{k-1} = Q_{kl-1}$ (da eindeutig bestimmt, usw gibt $Q_0 = Q_l$ (mit $l > 0$)). Ist hier $l$ minimal, so $2 \mid l$ (wegen $\text{sign}(a') = -\text{sign}(a)$)), und $Q_0, ..., Q_{l-1}$ sind alle verschieden.

Benennung:\\
$\zeta(Q) = [Q_0, Q_1,...,Q_{l-1}]$ heißt \underline{Zyklus von $Q$} ($Q$ reduziert)

Klar: Die Menge der reduzierten Formen zerfällt disjunkt in Zyklen.

\begin{satz}[Satz von Mertens]
Sei $U \in Sl_2(\MdZ), U \not= \pm 1_2$. Die Formen $Q$ und $\tilde Q := U.Q$ seien reduziert. Dann ist eine der Matrizen $\pm U, \pm U^{-1}$ ein Produkt von Nachbarmatrizen aufeinanderfolgender rechter Nachbarn. Insbesondere sind $Q$ und $\tilde Q$ im selben Zyklus.
\end{satz}

\begin{folgerung}
Für $2$ definite QFen $Q_1, Q_2$ sei $\Delta > 0$ usw (<- kein Quadrat) und es gilt:\\
$Q_1 \approx Q_2 \Leftrightarrow \text{red}(Q_2)$ ist im Zyklus $\zeta(\text{red}(Q_1)) \Leftrightarrow \zeta(\text{red}(Q_2)) = \zeta(\text{red}(Q_1))$.
\end{folgerung}

Klar:
\begin{itemize}
\item[1.] Es gibt einen Algorithmus, der entscheidet, ob $Q_1 \approx Q_2$ oder nicht
\item[2.] Die Zyklen entsprechen den Formklassen zu $\Delta \Rightarrow$ ist Algorithmus, der $h(\Delta)$ berechnet (stelle alle reduzierten Formen auf, berechne Zyklen!).
\end{itemize}

Zum Beweis des Satzes von Merteus: Viele mühsame Abschätzungen.\\

$U.Q = (-U).Q$, da $U = \begin{pmatrix}r & s\\u & v\end{pmatrix}, -U = \begin{pmatrix}-r & -s\\-u & -v\end{pmatrix}, 1 = \text{det}U =rv - us$. $U^{-1} = \begin{pmatrix}v & -s\\-u & r\end{pmatrix}, -U^{-1} = \begin{pmatrix}-v & s\\u & -r\end{pmatrix}$.

Die richtige Wahl entscheidet sich für passende positive Vorzeichen.\\
Ohne Einschränkung $r > 0, v > 0$, setzte $U' = UT_Q^{-1} = \begin{pmatrix}r' & s'\\u' & v'\end{pmatrix}$. Man zeigt: $IU, IU^{-1}$ keine Nachbarmatrix $\not= \pm 1 \Rightarrow 0 < r' < r$\\
Induktionshypothese für $U', Q' \Rightarrow$ Behauptung.

Über $h(\Delta)$ und Struktur der Klassengruppe bei $\Delta > 0$ "`fast"' keine allgemeine Sätze bekannt. Unbekannt z.B: existieren unendlich viele $\Delta$ mit $h(\Delta) = 1$?

\section{Automorphismengruppen}

\begin{definition}
\begin{itemize}
\item[(i)] $U \in Sl_2(\MdZ)$ heißt \underline{eigentlicher Automorphismus} der QF $Q = [a,b,c] :\Leftrightarrow U.Q = Q$.
\item[(ii)] $Aut_+(Q) = \{U \in Sl_2(\MdZ): U.Q = Q\}$ (ist UGR von $Sl_2(\MdZ) \leadsto$ Untergruppenkriterium) heißt \underline{eigentliche Automorphismengruppe} von $Q$.
\end{itemize}
\end{definition}

\begin{beweis}
\begin{itemize}
\item[(i)] $\Delta > 0 \Rightarrow \text{Aut}_+(Q)$ abelsch und $\#\text{Aut}(Q) = \infty. Q(\Delta) = k, U \in \text{Aut}_+(Q) \Rightarrow k = U.Q(\underline x) = Q(\underline xU)$. Mit $\underline x$ stellt auch $\underline xU$ die Zahl $k$ dar $\Rightarrow$ existieren unendlich viele $\underline y \in \MdZ^2: Q(\underline y) = k$.\\
Man kann zeigen: Es gibt $\underline x_1, ... \underline x_l, l \in \MdN_+$, so dass $\{\underline x \big| Q(\underline x) = k\} = \underline x_1G \dot \cup .. \dot \cup \underline x_lG$ mit $G = \text{Aut}_+(Q)$ (falls $k$ überhaupt darstellbar)
\end{itemize}
\end{beweis}

\begin{definition}
$[Q_0, ..., Q_{2l-1}] = \zeta(Q), Q = Q_0$ reduziert. Die Matrix $-T_Q, T_Q =: R$ heißt \underline{Doppelnachbarmatrix} zu $Q$ ($Q'$ rechter Nachbar). $B: R_{2l-2} \cdot ... \cdot R_2 \dot R_0$ heißt \underline{Grundmatrix} zu $Q$.
\end{definition}

Klar nach Definition: $B.Q = Q$, d.h. $B \in \text{Aut}_+(Q)$. Betrachte $V \in \text{Aut}_+(Q)$, so $\pm V, \pm V^{-1}$ (eines davon) nach Satz von Mertes ein Produkt von Nachbarmatrizen.

$\Rightarrow$ Eine dieser Matrizen ist Potenz von $B$! [würde sonst irgendwo mitten im Zyklus stehenbleiben]

\begin{satz}
$\text{Aut}_+(Q) = \{\pm B^m \big| m \in \MdZ\}$ ist sogar abelsch.
\end{satz}

Wieso unendlich? Man zeigt leichct: $R$ hat alle Koeffizienten $> 0 \Rightarrow B$ auch $\Rightarrow$ Alle Matrizen $\pm B^m$ sind verschieden.

Es gibt auch Aussagen für nicht-reduziertes $Q$. Ist $Q' = V.Q, V \in Sl_2(\MdZ)$, so ist die Abbildung $\phi: \text{Aut}_+(Q) \to \text{Aut}_+(Q'), U \mapsto VUV^{-1} =: \phi(U)$ ein Isomorphismus von Gruppen.

Moderne Theorie: Theorie der QFen zu $\Delta$ weitgehend äquivalent zur algZT in quadratischem "`Zahlkörper"' $K = Q(\sqrt{\Delta})$. Norm $n(a + b\sqrt{\Delta}) = (a + b\sqrt{\Delta})(a-b\sqrt{\Delta}) = a^2 - b^2 \Delta$ ist QF für a,b.


\end{document}
