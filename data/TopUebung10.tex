\documentclass{article}
\usepackage[utf8]{inputenc}
\usepackage{amsmath}
\usepackage{amsfonts}
\usepackage{amssymb}
\usepackage{amsthm}
\usepackage{mathrsfs}
\usepackage{german}
\usepackage{enumerate}
\usepackage{stmaryrd}
\title{7. Topologie Übung}
\author{Ferdinand Szekeresch}

\DeclareMathOperator{\so}{SO}
\DeclareMathOperator{\gl}{GL}
\DeclareMathOperator{\id}{id}

\begin{document}
\maketitle

\textbf{Aufgabe 1}\\
$p$ Primzahl, $\forall n\in\mathbb{N}$ setze $c_n:=\mathbb{Z}/p^n\mathbb{Z}$ mit diskr. Topologie.\\
Betrachte Teilmenge $\mathbb{Z}_p:\subseteq\prod\limits_{n\in\mathbb{N}}x_n$ der Tupel $(x_n)_{n\in\mathbb{N}}$ mit der Eigenschaft
$$\forall m\geq n:x_m\equiv x_n(\mod p^n)$$
\begin{enumerate}[(a)]
\item Beh: $\mathbb{Z}_p$ ist Ring. zeige: $\mathbb{Z}_p$ ist Teilring von $\prod\limits_{n\in\mathbb{N}}x_n$.\\
Nachrechnen: z.\,B. $(x_n),(y_n)\in\mathbb{Z}_p\\
\Rightarrow\forall m\geq n: x_m+y_m\equiv x_n+y_m(\mod p^n)\Rightarrow (x_n)+(y_n)\in\mathbb{Z}_p$
\item $\mathbb{Z}_p$ ist kpompakt. Denn: klar: $\mathbb{Z}/p^n\mathbb{Z}$ ist kompakt (da endlich)\\
$\stackrel{\text{Tichonoff}}{\Longrightarrow} \prod\limits_{n\in\mathbb{N}}(\mathbb{Z}/p^n\mathbb{Z})$ ist kompakt.\\
Zeige: $\mathbb{Z}_p$ ist abgeschlossen in $\prod\limits_{n\in\mathbb{N}}x_n$.\\
Beh: $\prod\limits_{n\in\mathbb{N}}x_n\backslash\mathbb{Z}_p$ ist offen.\\
Bew: Sei $(x_n)_{n\in\mathbb{N}}\in\prod\limits_{n\in\mathbb{N}}x_n\backslash\mathbb{Z}_p$\\
$\Rightarrow\exists n,m\in\mathbb{N}: n\leq m$ und $x_m\neq x_n(\mod p^n)$\\
Setze $U:=\pi_n^-1\left(\{x_n\}\right)$ ist offen in $\prod\limits_{n\in\mathbb{N}}x_n$ und es ist $U\cup\mathbb{Z}_p=\emptyset$\\
$\Rightarrow$ Beh.
\item $\mathbb{Z}\rightarrow\mathbb{Z}_p, a\mapsto\left((a+p^n\mathbb{Z})\right)_{n\in\mathbb{N}}$. Nachrechnen: Das ist ein Ringhomomorphismus.
Sein kern ist $\bigcap\limits_{n=1}^{\infty}p^n\mathbb{Z}=\{0\} \Rightarrow$ der Homomorphismus ist injektiv.
\item Vergleiche Blatt 2, Aufgabe 4. Sei hier $p=5$.\\
Beh: $\exists w=(x_n)_{n\in\mathbb{N}}\in\mathbb{Z}_5$, sodass gilt: $w^5=-1$, d.h. $x_n^2\equiv-1(\mod p^n)$.\\
Setze $x_1:=2 \left(\text{denn } 2^2=4\equiv-1(\mod 5)\right)$\\
Weitere Folgenglieder werden induktiv definiert:\\
Sei $x_n$ gefunden mit $x_n^2\equiv-1(\mod p^n)$\\
Zu zeigen: Es gibt ein $k\in\mathbb{Z}$, sodass $(x_n+kp^n)^2\equiv-1(\mod p^{n+1})$\\
\begin{align*}\leadsto a_p &\stackrel{!}{=}(x_n+kp^n)^2+1=x_n^2+2kx_np^n+k^2p^{2n}+1=(x_n^2+1)+2kx_np^n+k^2p^{2n}\\
&=bp^n+2kp^nx_n+k^2p^{2n}=p^n(2kx_n+b)+p^{2n}k^2\end{align*}
mit $a,b\in\mathbb{Z}$\\
Es muss gelten: $2kx_n+b\equiv0(\mod p)$.
Das geht, denn $\mathbb{Z}/p\mathbb{Z}$ ist Körper, $x_n$ kann nicht kongruent $0(\mod p)$ sein, da $x_n^2\equiv -1(\mod p)$ wäre.
\item Beh: $\mathbb{Z}_p$ ist überabzählbar.\\
Bew: z.\,B. Finde Bijektion von $\mathbb{Z}_p$ nach $[0,1)$, $(x_n)_{n\in\mathbb{N}}\mapsto 0,x_1x_2\ldots$
Oder: Fasse $\mathbb{Z}_p$ auf als \glqq Potenzreihen\grqq der form $\sum\limits_{n=0}^\infty a_np^n\quad a_i\in\mathbb{N}$
\end{enumerate}

\textbf{Aufgabe 3}\\
Sei $\mathcal{A}$ eine kommutative Banachalgebra, $\varphi: \mathcal{A}\rightarrow\mathbb{C}$ ein $\mathbb{C}$ linearer Ringhomomorphismus.\\
Beh: $\varphi$ ist stetige Linearform, d.h. $\exists\delta>0\forall f\in\mathcal{A}: |\varphi(f)|\leq\delta\|f\|$\\
Bew: Es gilt: $f-c=-c\left(1-\frac{f}{c}\right)$. Es ist $\left\|\frac{f}{c}\right\|=\frac{1}{\|c\|}\cdot\|f\|<\frac{1}{|c|}\cdot|c|=1$\\
$\Rightarrow\sum\limits_{n=0}^\infty\left(\frac{f}{c}\right)^n$ konvergiert gegen $\frac{1}{1-\frac{f}{c}}\Rightarrow(-c)^{-1}\cdot\sum\limits_{n=0}^\infty\left(\frac{f}{n}\right)^n$ ist invers zu $f-c$.\\
Beh: Es gilt $|\varphi(f)|\leq\|f\|$ für alle $f\in\mathcal{A}$.\\
Bew: Es gilt: $f-\varphi(f)\in\text{Kern}(\varphi)$ (Klar, da $\varphi$ Ringhomomorphismus ist $\varphi\left(f-\varphi(f)\cdot1_A\right)=\varphi(f)-\varphi(f)\varphi(1_A)=0$\\
$\Rightarrow f-\varphi$ ist nicht invertierbar (da $\text{Kern}(\varphi)$ eis Ideal in $\mathcal{A}$, d.h. wäre $f-\varphi(f)$ invertierbar, wäre $\text{kern}(\varphi)=\mathcal{A}\Rightarrow\varphi=0$.\\
$\stackrel{\text{Beh. 1}}{\Longrightarrow} Beh.$\\

\textbf{Aufgabe 2}\\
Sei $X$ norm. Raum, $X$ seine Stone-Cech-Kompaktifizierung, $K$ ein kompakter, normierter Raum, $\varphi:X\rightarrow K$ eine stetige Abbildung.
Beh: $\varphi$ kann eindeutig fortgesetzt werden zu einer stetigen Abbildung $\bar X\rightarrow K$.\\
Bew: $X$ normal $\Rightarrow \bar X$ ex. und ist Teilmenge von $C_0(x,\mathbb{C})'$\\
$K$ normal und kompakt $\Rightarrow \bar K$ ex. ist $\subseteq C_0(K,\mathbb{c})'$ und ist gleich $K$.\\
\ldots Rest wird ins Netz gestellt

\end{document}
