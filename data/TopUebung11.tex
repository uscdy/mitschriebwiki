\documentclass{article}
\usepackage[utf8]{inputenc}
\usepackage{mathrsfs}
\usepackage{stmaryrd}

\usepackage{mathe}
\usepackage{enumerate}
\usepackage{amscd}

\title{11. Topologie-Übung}
\author{Joachim Breitner}
\date{16. Januar 2008}

\begin{document}
\maketitle

\section*{Aufgabe 1}

Seien $X,Y$ Hausdorffräume.

\paragraph{Behauptung:} Ist $f:X\to Y$ surjektiv, stetig und abgeschlossen, dann trägt $Y$ die Quotiententopologie bezüglich $f$.

Zeige: $U\subseteq Y$ offen $\iff f^{-1}(U)$ ist offen.

„$\Longrightarrow$“: Klar, da $f$ stetig.

„$\Longleftarrow$“: Sei $U\subseteq Y$, so dass $f^{-1}(U)$ offen ist. Also ist $X\setminus f^{-1}(U)$ ist abgeschlossen, damit ist $f(X\setminus f^{-1}(U))$ ebenfalls abgeschlossen in $Y$.

Nach Definition ist $f(X\setminus f^{-1}(U)) \cap f^{-1}(U) = \emptyset\quad (*)$. Es gilt: $X=(X\setminus f^{-1}(U)) \cup U$, also ist  $f(X) = Y = f(X\setminus f^{-1}(U)) \cup f(f^{-1}(U)) = f(X\setminus f^{-1}(U)) \cup U$. Mit $(*)$ folgt dann: $U=Y\setminus f(X\setminus f^{-1}(U))$ und damit offen.


\paragraph{Behauptung:} Ist $X$ kompakt und $f:X\to Y$ surjektiv und stetig, dann trägt $Y$ die Quotiententopologie bezüglich $f$.


„$\Longrightarrow$“: Klar, da $f$ stetig.

„$\Longleftarrow$“: Sei $U\subseteq Y$, so dass $f^{-1}(U)$ offen ist. Es genügt zu zeigen: $f(X\setminus f^{-1}(U))$ ist abgeschlossen, dann folgt die Aussage wie oben.

$X\setminus f^{-1}(U)$ ist abgeschlossen und damit kompakt. Das Bild $f(X\setminus f^{-1}(U))\subseteq Y$ ist auch kompakt und, da $Y$ hausdorff’sch ist, auch abgeschlossen.

\section*{Aufgabe 2}

\paragraph{Behauptung:} Zwei Wege $\gamma,\delta:S^1\to \mathbb C^\times$ sind homotop $\iff \chi(\gamma,0)=\chi(\delta,0)$.

„$\Longrightarrow$“: Siehe Bemerkung 2.4.15 in der Vorlesung.

„$\Longleftarrow$“: Aus $\gamma, \delta: S^1\to \mathbb C^\times$ kann man $\frac\gamma{\|\gamma\|}, \frac{\delta}{\|\delta\|}: S^1\to S^1$ konstruieren.
:
\[
\begin{CD}
[0,1] @>\exists !\tilde\gamma,\tilde\delta>> \mathbb R  \\
@VV\pi V @VV\pi: t\mapsto \left(\begin{smallmatrix}
\cos(2\pi t) \\ \sin(2\pi t)
\end{smallmatrix}\right)V \\
S^1 @>>\frac\gamma{\|\gamma\|}, \frac{\delta}{\|\delta\|}> S^1
\end{CD}
\]
$\tilde\gamma$, $\tilde\delta$ sind homotop, denn
$H:[0,1]\times[0,1] \to \mathbb R$, $H(x,t)\da (1-t)\tilde\gamma(x) + t\tilde\delta(x)$ ist eine Homotopie. Also sind $\frac\gamma{\|\gamma\|}, \frac{\delta}{\|\delta\|}$ homotop, denn $\tilde H: S^1 \times [0,1] \to S^1$, $\pi \circ H(\pi^{-1}(x),t)$ ist Homotopie, denn es ist
$\tilde H(x,0)= \frac\gamma{\|\gamma\|}(x)$, $\tilde H(x,1)= \frac\delta{\|\delta\|}(x)$, per Definition und $\tilde H(0,t) = \tilde H(1,t)$, denn:
\begin{align*}
H(1,t)-H(0,t) &= (1-t)\tilde \gamma(1) + t \tilde \delta(1) - ( (1-t)\tilde\gamma(0) + t\tilde\delta(0))\\
&= (1-t)(\tilde\gamma(1)- \tilde\gamma(0)) + t(\tilde\delta(1)- \tilde\delta(0))\\
&= (1-t)\chi(\gamma,0) + t(\chi(\delta,0))\\
&= \chi(\delta,0) \in \mathbb Z
\end{align*}
Also gilt $\tilde H(0,t) = \tilde H(1,t)$, da $\sin$ und $\cos$ $2\pi$-periodisch sind.

\paragraph{Behauptung:} Es gibt eine bijektive Abbildung $[S^1,S^1]\to \mathbb Z$.

$\chi:[S^1,S^1] \to \mathbb Z$, $[\gamma]\to \chi(\gamma,0)$ ist, wie oben gezeigt, injektiv und wohldefiniert. Surjektivität ist klar.

\section*{Aufgabe 3}

Sei $X$ ein lokalkompakter Hausdorffraum, $\mathcal C(X,y)$ versehen mit der  kompakt-offen-Topologie.

\paragraph{Behauptung:} $H:X\times [0,1]\to Y$ ist stetig $\iff$ $t\mapsto H_t \da (x\mapsto H(x,t))$ definiert eine stetige Abbildung $[0,1]\to \mathcal C(X,Y)$.

Subbasis der kompakt-offenen-Topologie sind die Mengen der Form $V_{K,U}\da \{f\in \mathcal C(X,Y) \mid f(K)\subseteq U$\} für kompakte $K\subseteq X$ und offene $U\subseteq y$.

„$\Longleftarrow$“: Zeige $H$ ist stetig. Sei $(x,t)\in X\times I$ und $U$ eine offene Umgebung von $H(x,t)\ad H_t(x)$. Zeige dazu: Es gibt eine Umgebung $V$ von $(x,t)$ mit $H(V)\subseteq U$.

Denn: Weil $X$ lokalkompakt und hausdorff’sch ist, enthält jede Umgebung von $x\in X$ eine kompakte Umgebung (da der Schnitt einer kompakten Umgebung von $x$ mit einer abgeschlossenen Umgebung von $x$ wieder eine kompakte Umgebung von $x$ ist).
Da $H_t$ stetig ist, hat $x$ eine kompakte Umgebung $K$ mit $H_t(K)\subseteq U$, also ist $H_t \in V_{K,U}$. $(t\mapsto H_t)$ ist stetig, also gibt es ein $\varepsilon >0$, so dass das Bild von $(t-\varepsilon, t+\varepsilon)\subseteq V_{K,U}$. Setzte $V\da K\times (t-\varepsilon, t+\varepsilon)$. Das erfüllt das Gewünschte: Für alle $(\tilde x,t)\in V$ gilt: $H(\tilde x,t) = H_t(\tilde x) \in H_t(K)\subseteq U$. Also ist $H$ stetig.

„$\Longrightarrow$“: Zu zeigen: $\Phi: t\mapsto H_t$ ist stetig.
Zeige: Sei $t\in I$ und o.B.d.A: $V_{K,U}$ eine offene Umgebung von $\Phi(t)$, dann gibt es eine Umgebung $V$ von $t$ mit $\Phi(V)\subseteq V_{K,U}$.

Denn: $\Phi(t)\in V_{K,U}$ heißt: $\Phi(t)(K) = H(K,t) \subseteq U$. $H$ ist stetig, also findet sich für jedes $k\in K$ eine offene Umgebung $W_k$ von $k$ und $\varepsilon_k >0$ mit $H(W_k\times (t-\varepsilon_k, t+\varepsilon_k))\subseteq U$. $(W_k)_{k\in K}$ ist eine offene Überdeckung des Kompaktum $K$, aus der eine Teilüberdeckung $W_{k_1},\ldots,W_{k_n}$ ausgewählt werden kann. Setze $\varepsilon \da \min \{\varepsilon_{k_1},\ldots,\varepsilon_{k_n}\}$.

Es ist $\Phi(t-\varepsilon, t+\varepsilon)\subseteq V_{K,U}$, denn: Sei $r\in(t-\varepsilon, t+\varepsilon)$ und $k\in K$, dann gilt: $H(k,r) \in H( W_{k_i} \times (t-\varepsilon_{k_i}, t+ \varepsilon_{k_i}))\subseteq U$ für ein $i\in\{1,\ldots,n\}$.
Also ist $H(K,r)\subseteq U$, und damit $H_r \in V_{K,U}$. $r$ war beliebig, woraus die Behauptung folgt.

\section*{Aufgabe 4}

Sei $X$ kompakt, $(Y,d)$ ein metrischer Raum.

\paragraph{Behauptung:} Die kompakt-offene-Topologie auf $\mathcal C(X,y)$ wird induziert von der Metrik $d(f,g)\da \sup\{d(f(x),g(X))\mid x\in X\}$.

Zeige zuerst: Jedes $V_{K,U}$ ist offen bezüglich der Metrik $d$. Dazu zeige: Zu jedem $f\in V_{K,U}$ gibt es ein $r>0$: $B_r(f)\subseteq V_{K,U}$.

$f(K)\subseteq U$ und $f(K)$ ist kompakt, also gibt es ein $r>0$, so dass gilt: $f(K)\subseteq U'\da \{y\in Y\mid d(y,f(K)) < r\}\subseteq U$.

Für $g\in B_r(f)$ und $k\in K$ gilt: $d(g(k),f(K))\le d(g(k),f(k))\le d(f,g) < r$. Also ist $g(k)\in U'\subseteq U$, damit ist $g(K)\subseteq U$ und somit $g\in V_{K,U}$. Also ist $V_{K,U}$ offen bezüglich $d$.

Zeige nun: Für jedes $f\in \mathcal C(X,Y)$ und jedes $r>0$ ist $B_r(f)$ offen bezüglich der kompakt-offen-Topologie.

Zeige dazu: Für jedes $g\in B_r(f)$ gibt es eine bezüglich der kompakt-offen-Topologie offene Menge $V$ mit $g\in V\subseteq B_r(f)$.

Es ist $d\da d(f,g)<r$. Setzte $\gamma \da \frac{r-d}2$. Für jedes $x\in X$ ist $B_{\frac 12 \gamma}(g(x))$ offen in $Y$. Damit gibt es eine offene Umgebung $W_x$ von $x$ mit $g(W_x)\subseteq  B_{\frac 12 \gamma}(g(x))$. Es ist $g(\overline{W_x})\subseteq B_\gamma(g(x))$. Da $X$ kompakt ist und $(W_x)_{x\in X}$ eine offene Überdeckung von $X$ sind, gibt es eine offene Teilüberdeckung $\{W_{x_1},\ldots,W_{x_n}\}$ aus $(W_x)_{x\in X}$. Setzte $V\da V_{\overline{W_{x_1}},B_\gamma(g(x_1))} \cap \cdots \cap V_{\overline{W_{x_n}},B_\gamma(g(x_n))}$.

$V\subseteq B_r(f)$, denn: Sei $h\in V$ und $x\in x$. Nach Konstruktion gibt es ein $i\in\{1,\ldots,n\}$, so dass $x\in W_{x_i}$ ist. $g(x)\in g(W_{x_i}) \subseteq B_{\frac12} \gamma(g(x_i))$, also $d(g(x),g(x_i))<\frac\gamma2$.

Wegen $h\in V\subseteq V_{\overline{W_{x_i}}, B_\gamma(g(x_i))}$ gilt $h(x) \in h(\overline{W_{x_i}}) \subseteq B_\gamma(g(x_i))$, also $d(h(x),g(x_i))<\gamma$. Also gilt: $d(h(x),f(x)) \le d(h(X),g(x_i)) + d(g(x_i),g(x)) + d(g(x),f(x)) < \gamma + \frac \gamma2 + d = \frac{r-d}2 + \frac{r-d}4 + d = \frac{3r+d}4 < r$, also $h\in B_r(f)$.



\end{document}
